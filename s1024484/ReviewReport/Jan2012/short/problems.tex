\section{Some Research Problems}
\label{problems}

\subsection{Wadler's Type Pollution Problem in Actor Based Systems}
\label{type_pollution}
Wadler's type pollution problem refers to the situation where one component of a system leaks part of its type information that should not be known by another component.

The type pollution problem exists in most actor based systems.  Let us consider a layered system\cite{DijkstraTHE}, where each layer provides services to the layer immediately above it, and implements those services using services in the layer immediately below it.  If a layer is implemented using an actor, even if in a static typed language, then the actor will receive both messages form the layer above and the layer below via its only channel.  As actors built on current implementations\cite{actor_1, actor_2} use the same interface to communicate with all others, the higher layer could send the middle layer a message that is not expected from the higher layer, so as the lower layer.

One solution to the above problem is using separate channels for communication between different parties.  However, the actor model has to be abandoned in this solution.  So far, we identified two alternatives for the actor model: join calculus \cite{full_join} and session types \cite{Honda93typesfor, Honda_languageprimitives}.  In join calculus, one could define different channels to receive messages for different purposes.  In session types, a communication party can only see the type of communication that it involved in.  The join calculus has been implemented both as language extensions \cite{join_csharp} and libraries \cite{scala_joins, join-fsharp, Russo07thejoins}.  The session type, on the other hand, has only been implemented as prototype languages \cite{Honda_languageprimitives, SJ}.  It might be a challenge to implement session types as a library.

Another solution is using sub-typing.  Let a layer has type Actor[T], denoting that it is an actor that expects messages of type T.  With sub-typing, one could publish a layer as Actor[A] to the layer above it while publish the layer as Actor[B] to the layer below it.  The system could check if both A and B are subtypes of T.  The idea sounds feasible but a few practical problems needs to be solved.  Some problems of implementing the proposed Actor will be discussed in the next section.


\subsection{Actor Parametrised on the Type of Expecting Messages}
\label{typed actor}
The last section proposed an actor that are parametrised on the type of messages it expects to receive.  This proposal sounds like a natural solution to enhance type-safety of communication between actors.  Surprisingly, I do not find such a version in literature nor in popular platforms.  Apart from the type pollution problem, further investigation reveals following problems of implementing the proposed Actor as a compatible alternative to Scala Actor and akka Actor.

\begin{enumerate}
\item The $sender$ field.  To simplify the work of replying to synchronous request, both akka Actor and Scala Actor has a $sender$ field that records the source of the latest received message.  Notice that new values are assigned to the $sender$ field each time a message is received at run-time.  It could be argued what the type parameter of the $sender$ field is, at compile time.

\item Linked actors.  Following the Erlang design, Actors may be linked together so that the exit signal of a dying actor will be broadcasted to other actors.  In this case, only exit signals are transmitted between actors.  Let $E$ be the super type of all exit signals.  Should actor of $Actor[T]$ means an actor expecting message of type $T\vee E$\footnote{$T\vee E$ denotes that a value either has type $T$ or $E$}, or an actor expecting message of type $T :> E $\footnote{$T :> E $ denotes that $T$ is a super type of $E$}?

\item Layers in a supervision tree.  The akka framework provides APIs to enquiry the parent actor and child actors of an actor.  If we would not like to leak type informations between layers, then an actor may need to delegate this kind of enquiries to its parent or child.  Similar to the problem in supporting linked actors, we introduced another type of messages which should be recognised by all actors.

\item The demand for type-safe casting.  To solve the type pollution problem, an actor may be published as if it only expects a sub-set of messages that it is actually expecting.  In those cases, an actor need to $cast$ its type parameters.  This cast should be distinguished from unsafe typecasting, as we have sufficient information to determine whether the type parameter is casting to its subtype.

\end{enumerate}

\subsection{Hot Code Swapping}
Hot code swapping plays an important role in systems running on OTP platform.  At any time, the Erlang/OTP run-time keeps at most two versions of a module running simultaneously.

In an experiment, I implemented a GenServer class, whose instance maintains two pointers to different implementations.  Messages sending to a server is delegated to the $current$ implementation.  The $old$ implementation will not be garbage collected as long as it is referenced.  Comparing to the Erlang platform, this design has the limitation of having a non-swappable GenServer.  I implemented a more general wrapper process to wrap code that potentially be swapped.  This solution improves the situation a bit but does not eliminate the problem of having a non-swappable code.

The akka library \cite{akka} also provides a form of hot swap on the message loop of an actor.  In akka, different versions of message loops are stored at a stack in corresponding ActorContext.  Users could send the actor some special messages, defined in the current message loop, to trigger the context.become/unbecome method.  The downside of this method is three-fold.  Firstly, developers need to specify not only that the code will be potentially hot swapped but also how to trigger the swap.  Secondly, passing new message loop, which is a function, over network is still a problem.  Lastly, different from the behaviour of Erlang programs, only one message loop is ``active'' at a time.

%\subsection{Problems on Serialization}




\subsection{Name Server that Combines Static and Dynamic Typing}
In distributed programming which supports modular compilation, distributed components are implemented according to agreed communication interfaces.  In untyped settings, developers strictly follow the system specification to ensure that one part will be compatible with others.  In typed settings, developers have the advantage of checking the consistency between implementation and the predefined communication interface at compilation time, before the implementation is plugged into the running system.  Unfortunately, most of distributed systems are not static but require dynamic updating individual components, and therefore change communication interfaces now and then.  

To detect incompatible interfaces at the earliest opportunity, we could implement a novel name server that maps a pair of name and type to a channel expecting values of that type.  The novel name server will provide two benefits.  Firstly, ill-typed message sending to a registered channel will be rejected before it is sent.  Secondly, components that relies on a service could be informed when the new communication interface is no longer backward compatible.

Although building a name server described as above will provide promising features, it is not clear yet how many changes are required to embed the name server into standard Scala or the akka framework.

\subsection{Eventual Consistency for Shared States}
Users of a distributed system would expect receiving consistent value when using shared states. Unfortunately, known as the CAP theorem \cite{CAP}, strong consistency, availability, and partition tolerance could not be obtained at the same time in distributed systems.  In practice, developers usually weaken strong consistency to eventually consistency to improve availability and partition tolerance.  Eventually consistency is an active topic both in theoretical study\cite{Eventually_Consistent_Transactions, Vogels_2009} and in practise\cite{Dynamo}.  For our project, we would like to apply the eventually consistency model to build global consistent name server.

