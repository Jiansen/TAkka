\documentclass[coverpage]{inftechrep}

\usepackage{comment}

\usepackage{graphicx}
\usepackage{subfigure}
\usepackage{url}

\usepackage{paralist}
\usepackage{hyperref}
\usepackage[all]{hypcap}
\usepackage{verbatim}
\usepackage{array}
\usepackage{float}
\usepackage{multirow}
% \usepackage{rotating}
\usepackage{multicol}
\usepackage{longtable}
\usepackage{changepage}

\usepackage{color}
\usepackage{listings}   
% "define" Scala
\lstdefinelanguage{scala}{
  morekeywords={abstract,case,catch,class,def,%
    do,else,extends,false,final,finally,%
    for,if,implicit,import,match,mixin,%
    new,null,object,override,package,%
    private,protected,requires,return,sealed,%
    super,this,throw,trait,true,try,%
    type,val,var,while,with,yield},
  otherkeywords={=>,<-,<\%,<:,>:,\#,@},
  sensitive=true,
  morecomment=[l]{//},
  morecomment=[n]{/*}{*/},
  morestring=[b]",
  morestring=[b]',
  morestring=[b]"""
}

% A comment in a draft (shouldn't appear in the final version).
\newcommand{\mycomment}[1]{\(\spadesuit\){\bf #1 }\(\spadesuit\)}
% Comment this out in the draft
% \newcommand{\mycomment}[1]{}
\newcommand{\pmcomment}[1]{\comment{PM}{#1}}
\newcommand{\pwtcomment}[1]{\comment{PT}{#1}}
\newcommand{\rscomment}[1]{\comment{RS}{#1}}

% line break in table.  e.g.
% Foo bar & \specialcell{Foo\\bar} & Foo bar \\    % vertically centered
% Foo bar & \specialcell[t]{Foo\\bar} & Foo bar \\ % aligned with top rule
\newcommand{\specialcell}[2][c]{%
  \begin{tabular}[#1]{@{}c@{}}#2\end{tabular}}

  

\begin{document}
\title{Statically Typed Reliability \\Type-parametrized Actors and Their
Supervision (v.0)
}
\author{Jiansen HE}
% \reportnumber{25}
% \reportyear{1999}
% \reportmonth{5}
\keywords{actor, type, supervision tree, nameserver}
% \published{Presented at the WITCH99 Conference, Debrecen, Hungary, May 1999.}
\institutes{
  Laboratory for Foundations of Computer Science}
\maketitle

\begin{abstract}
Building robust concurrent programs, especially in distributed settings, is
challenging as it often involves complex program structures and dynamic
behaviours.  The robustness of concurrent programs could be improved by
\begin{inparaenum}[(i)]
 \item using static type checking to prevent certain data misuses, or
 \item employing failure recovery mechanisms such as supervision tree.
\end{inparaenum}
An important open question is to what extent those approaches could be merged.

This paper presents the design, implementation and preliminary evaluation
results of the TAkka library, where types of actor related components are
statically typed and operations are type checked at the earliest possibility.
We introduce actors parametrized by the type of messages they expect to
receive. We show that it is straightforward to construct
supervision trees of type-parametrized actors. We minimize the number of system
messages that may be handled by general users by providing standard supervision
strategies.

We evaluate the TAkka library by re-implementing 19 examples from other
OTP-like libraries.  our results show that TAkka has small runtime and
code size overheads compared with Akka. Finally, TAkka programs have similar
scalability comparing
to their Akka equivalent.

\mycomment{rewrite the abstract to reflect the purpose of this technical report.}

\mycomment{Let's show how static and dynamic type checking improve actor
programming}

\end{abstract}

\section{Introduction}

The Actor model defined by \citet{Hewitt:1973} treats actors as
primitive computational components.  Actors collaborate by sending asynchronous
messages to each other.  An actor independently determines its reaction to
messages it receives.  The Actor model was adopted in the Erlang programming 
language \citep{ArmstrongErlang} for building real time telephony applications. 
Erlang developers later propose thed supervision principle \citep{OTP}, which 
suggests that actors should be organized in a tree structure so that any failed 
actor can be properly restarted by its supervisor.  Nevertheless, adopting the 
supervision principle is optional in Erlang.

The notions of actors and supervision trees have been ported to statically 
typed languages including Scala and Haskell.  Scala actor libraries including 
Scala Actors \citep{actor_1, actor_2} and Akka \citep{akka_api,akka_doc} use 
dynamically typed messages even though Scala is a statically typed language.  
Some recent actor libraries, including Cloud Haskell \citep{CloudHaskell}, Lift 
\citep{lift_scala}, and scalaz \citep{scalaz}, support both 
dynamically and statically typed messages, but do not support supervision.
Can actors in supervision trees be statically typed?

The key claim in this paper is that actors in supervision trees can be
typed by parameterizing the actor class with the type of messages it expects to
receive.  Type-parameterized actors benefit both users and developers
of actor-based services. For users, sending ill-typed messages is prevented at 
compile time.  Because messages are usually transmitted asynchronously, it may 
be otherwise difficult to trace the source of errors at runtime, especially in 
distributed environments.  For service developers, since unexpected messages 
are eliminated from the system, they can focus on the logic of the services 
rather than worrying about incoming messages of unexpected types.  


Implementing  type-parameterized actors in a statically-typed language; however, requires 
solving the following challenges identified by the Akka Development Team:

\begin{enumerate}
  \item The Akka developers claim that using a type parameterized actor" is impossible because actor behaviour is dynamic" \citep{Kuhn11}.

  \item The Akka developers claim that ``we need to be able to encode disjoint types" \citep{Kuhn12}, an option which is not provided by Scala. 

  \item The Akka developers claim that ``address lookup will be impossible" \citep{Kuhn12} because actor references will have different types.
  
  \item The Akka developers claim that� ``the way to provide high performance is to provide fewer guarantees, and also let the programmer choose what guarantees he or she needs'' \citep{Kuhn12}.
\end{enumerate}






Continuing a line of work on merging types with actor programming by Haller
and Odersky \citep{actor_1, actor_2} and the Akka developers \citep{akka_doc}, 
along with work on system reliability testing by 
\citet{ChaosMonkey} and \citet{ErlangChaosMonkey}, this paper makes 
following contributions.

\begin{itemize}
 \item It presents the design and implementation of a typed name server
that maps typed names to values of the corresponding type, for example actor 
references.  The typed name server (Section \ref{nameserver}) mixes 
static and dynamic type checking so that type  errors are detected at the earliest 
opportunity.  The implementation requires language support for type reflection 
and first class type descriptors.  In Scala, the {\tt Manifest} class provides 
such facility.

 \item It describes the design of the TAkka library.  Sections \ref{actor} to
\ref{actor_context} illustrate how type parameters are added to actor related
classes to improve type safety.  The TAkka actor supports dynamic backward-compatible 
behaviour upgrading as shown in Section \ref{hot_swapping}.  By separating the handler for system 
messages from the handler for user defined messages, Sections \ref{supervision}
and \ref{systemmessage} show that type-parameterized actors can form
supervision trees in the same manner as untyped actors.  Section 
\ref{evolution} explains how to upgrade Akka programs to their TAkka 
equivalents gradually.
 
 \item It gives a comprehensive evaluation of the TAkka library.  Section 
\ref{type_pollution} shows that the TAkka library can avoid the type pollution 
problem straightforwardly.  Results in
Section \ref{expressiveness} confirm that using type parameterized actors 
sacrifice neither expressiveness nor correctness.  Efficiency and 
scalability test in Section \ref{efficiency} shows that TAkka applications have 
little overhead at the initialization stage but have almost identical run-time 
performance and scalability compared to their Akka equivalents.  In 
addition, two helper libraries explained in Section \ref{reliability} are 
shipped with the TAkka library for assisting reliability evaluation.  The 
first library, Chaos Monkey, is ported to test applications against intensive 
adverse conditions.  The second library, Supervision View, is 
designed and implemented to dynamically monitor changes of supervision tree.
 
\end{itemize}

\section{Background}
\label{background}



\subsection{Actor Model and OTP Design Principles}

The Actor model defined by Carl Hewitt treats actors as primitive computational
components.  Actors collaborate by sending asynchronous messages to each other.
An actor independently determines its reaction to messages it receives.
Furthermore, an actor should confirm its {\it intention} to the context where
it resides.  Carl Hewitt defines intention as ``the contract that the actor
has with the outside world''.\cite{Hewitt:1973}


The Actor model is adopted by Erlang programming language, whose developers
later summarized five OTP design principles to improve the reliability of Erlang
applications\cite{OTP}.  Table-\ref{otp} lists OTP design principles and their
possible implementation methods in an objected-oriented language.  Apart from
the Supervision Tree principle,  an important subject to be discussed later,
other OTP design principles can be straightforwardly implemented in an
objected-oriented language.  For a platform such as Java Virtual Machine (JVM)
where hot swapping is not supported in general, hot swapping a particular
component could be simulated by updating the reference to that component.
\S\ref{code_evolution} explains how hot swapping the behaviour of an actor is
supported in TAkka, which is run on JVM.



\begin{table}[h]
\label{otp}
% \tbl{Implementing OTP Design Principles in an OO Language}{
 \caption{Implementing OTP Design Principles in an OO Language}
 \begin{center}
\begin{tabular}{| l | p{7.8cm} | }
\hline
  OTP Design Principle & Objected-Oriented\\
                       & Implementation \\
\hline
  Supervision Tree  & no direct correspondence  \\
\hline
  Behaviour & defining an abstract class \\
\hline
  Application  & defining an abstract class that has two abstract methods: start and stop \\
\hline
  Release  & packaging related application classes  \\ 
\hline
  Release Handling  & hot swapping support on key modules is required \\
\hline
\end{tabular}
 \end{center}
%}
\end{table}


\subsection{Scala Type Descriptors}
The TAkka library is designed to detect type errors as early as possible.  For
type errors that cannot be statically detected, adequate dynamic type checking
may expose them before the program goes wrong.  To this end, a notion of
run-time type descriptor is required.

In versions since Scala 2.10.0, first class type descriptor for type {\bf T},
which has type {\bf Type}, could be constructed from {\it typeOf[T]}.  In the
case that {\bf T} is a generic type, a {\bf TypeTag} for {\bf T}, constructed
from {\it typeTag[T]} must be provided.  In addition, with {\it implicit
parameter} (e.g. line 4 in the code below) and {\it context bound} (e.g. line 10
in in the code below), library designer could define concise APIs.  For example,
in a Scala interpreter, one could obtain the following results:

\begin{lstlisting}
  scala> typeOf[Int => String]
  res0: Type = Int => String

  scala> def name[T](implicit m: TypeTag[T]) = typeOf[T]
  name:[T](implicit m:TypeTag[T])Type

  scala> name[Int=>String]
  res3: Type = Int => String

  scala> def name2[T:TypeTag] = typeOf[T]
  name2: [T](implicit evidence$1: reflect.runtime.universe.TypeTag[T])Type

  scala> name2[Int => String]
  res2: Type = Int => String
\end{lstlisting}

In TAkka, dynamic type checking is achieved by comparing type descriptors rather
than checking the most precise type of a value.  The adopted approach has two
advantages.  For one thing, it detects type errors otherwise may be left out.
For example, let {\bf S} be a subtype of {\bf T}, the output of function {\it f}
has type {\bf T} and the input of function {\it g} should have type {\bf S}, it
is clear that composing {\it f} and {\it g} is not always type safe.  In an
environment where static type checking is not possible, comparing type
descriptors of {\bf S} and {\bf T} reveals the type error instantly whereas the
alternative approach may admit some test cases.  For another, data security is
improved since values are not disclosed until type safety is confirmed.

\subsection{Type Parameter and Generic Programming}

\mycomment {  Short examples of Generic Collection in Scala, e.g. List[E].  Show how static type-checking avoid type-errors.}


\subsection{Typed Nameserver}
\label{nameserver}

\mycomment{use this example to show the need and benefits of mixing static and dynamic type checking.}

A canonical nameserver maps each registered name, a unique string, to a value,
and to look up a value by giving a name.  A name could be encoded as a {\bf
Symbol} in Scala so that names which represent the same string have the same
value.  We noticed that the widely used canonical namesever has two limitations.
 For one thing, values retrieved from a canonical nameserver are dynamically
typed.  Therefore, circumspect programmers would have to manually insert type
checking code to prevent type errors in later program.  For another, the
canonical nameserver has a weak security barrier in the sense that values will
be returned to an enquirer regardless of his knowledge that how the returned
data values should be used.

To overcome the above limitations of canonical nameserver, we designed and
implemented a typed nameserver which maps each registered typed name to a value
of the corresponding type, and to look up a value by giving a typed name.  A
typed name, {\bf TSymbol}, is a name shipped with a type descriptor.  In Scala,
{\bf TSymbol} could be simply defined as follows:

\begin{figure}[h]
\label{tsymbol}
\begin{lstlisting}
case class TSymbol[T:Type](val s:Symbol) {
    private [takka] def t:Type[T] = typeOf[T]
    override def hashCode():Int = s.hashCode()  
}
\end{lstlisting}
\caption{TSymbol}
\end{figure}

Notice that {\bf TSymbol} is covariant because the same type value is passed to
{\bf Manifest}, another covariant type constructor.  In addition, {\bf TSymbol}
is declared as a {\it case class} in Scala so that the name {\bf TSymbol} could
be used as a data constructor and pattern matching on parameters of {\bf
TSymbol} is supported automatically.  Finally, the manifest is shipped as an
implicit value parameter and given a name so that general users do not need to
manually construct the manifest and the library developer could access the
manifest as a field of {\bf TSymbol}.

With the help of {\bf TSymbol}, we implemented a typed nameserver which provides
following three operations:
\begin{itemize}
  \item {\it set[T:Manifest](name:TSymbol[T], value:T):Unit}

Register a typed name with a value of corresponding type.  A {\bf
NamesHasBeenRegisteredException} will raise if the symbol representation of
$name$ has been registered.  The most precise type of $value$ shall be a subtype
of the intention type {\bf T}.

  \item {\it unset[T](name:TSymbol[T]):Unit}

Cancel the entry $name$ if (i)  its symbol representation is registered and (ii)
the the intention type {\bf T} is a super type of the registered type.

  \item {\it get[T] (name: TSymbol[T]): Option[T]}

Return Some(v:{\bf T}) if (i) $name$ is associated with a value and (ii) {\bf T}
is a supertype of the registered type; Otherwise return None.
\end{itemize}

Notice that $unset$ and $get$ operations permit polymorphism.  To this end, the
{\it hashcode} method of {\bf TSymbol} defined in Figure-{\ref{tsymbol}} does
not take manifest into account.  Instead, type information are checked at run
time. The typed nameserver also prevents users from registering two names with
the same symbol but different manifests, in which case if one is a supertype of
another, the return value of $get$ will be non-deterministic.

\begin{comment}
\subsection{Obligatory Supervision}

System reliability will be improved if failed components are restarted in time.
The Supervision Tree Principle \cite{OTP} suggests that actors should be
organised in a tree structure so that any failed actor could be properly
restarted by its supervisor.  Nevertheless, adopting the Supervision Tree
principle is optional in Erlang.

The Akka library makes supervision obligatory by restricting the way of creating
actors.  The only way to create an actor is using the {\it actorOf} method
provided by {\bf ActorSystem} or {\bf ActorContext}.  Each actor system
provides a guardian actor for all user-created actors.  Calling the {\it
actorOf} method of an actor system creates an actor supervised by the guardian
actor.  Calling the {\it actorOf} method of an actor context creates a child
actor supervised by that actor.  Therefore, all user-created actors in an
actor system, together with the guardian actor of that actor system, form a tree
structure. Obligatory supervision unifies the the structure of actor deployment
and simplifies the work of system maintenance.

In the code below, we define a supervisor and its child.  The Supervisor
forwards all messages to the child.  The child will be restarted by its
supervisor when it raises an exception.  The supervision strategy also specifies
the maximum failures a child may have within a time range.  If a child fails
more frequently than the the allowed frequency, the supervisor will be stopped,
report its failure to its supervisor. In the sample code, the child raises an
exception every time it receives a ``Stop'' string; it prints out the string
representation of other messages. The terminal output indicates that the child
actor is restarted before the second message is delivered.

      \begin{lstlisting}[language=scala]
class Supervisor extends Actor {
  override val supervisorStrategy =
    OneForOneStrategy(maxNrOfRetries = 10, withinTimeRange = 1 minute) {
      case _: Exception                ⇒
        println("Exception Raised to: "+self)
        Restart
    }

  val child:ActorRef = context.actorOf(Props[Child], "child")

  def receive:Any => Unit = {
    case m => child ! m
  }
}

class Child extends Actor {
  def receive:Any => Unit = {
    case "Stop" => throw new Exception("Test Exception")
    case m => println("Received message: "+m)
  }
}


object SupervisorTest extends App{
  val system = ActorSystem("MySystem")
  val actorRef:ActorRef = system.actorOf(Props[Supervisor], "supervisor")

  actorRef ! "Stop"
  actorRef ! "Hello World!"
}

/*
Terminal Output:
Exception Raised to: Actor[akka://MySystem/user/supervisor]
Received message: Hello World!
*/
    \end{lstlisting}


\end{comment}

\section{TAkka Library Design}

This section presents the design of the TAkka library.  We outline how
we add types to actors and how to construct supervision trees of typed actors in
TAkka.  This section concludes with a brief discussion about design alternatives
used by other actor libraries.

\subsection{Type-parameterized Actor}
\label{actor}
A TAkka actor has type {\tt TypedActor[M]}.  It inherits the Akka {\tt
Actor} trait to minimize implementation effort.  Users of the TAkka library,
however, do not need to use any Akka Actor APIs.  Instead, we encourage
programmers to use the typed fields given in Figure \ref{takka_actor_api}. 
 Unlike other actor libraries, every TAkka actor class takes a type parameter 
{\tt M} which specifies the type of messages it expects to receive.  The same 
type parameter is used as the input type of the receive function, the type 
parameter of actor context and the type parameter of the actor self reference.

\begin{figure}[H]
\label{takka_actor_api}
\begin{lstlisting}[language=scala]
package takka.actor
abstract class TypedActor[M:Manifest] extends akka.actor.Actor {
  def typedReceive:M=>Unit
  val typedSelf:ActorRef[M]
  val typedContext:ActorContext[M]
  var supervisorStrategy: SupervisorStrategy
}
\end{lstlisting}
\caption{TAkka Actor API}
\end{figure}

The two immutable fields of {\tt TypedActor}: {\tt typedContext} and 
{\tt typedSelf}, will be initialized automatically when the actor is created.
Library users may override the default supervisor strategy in the
way explained in Section \ref{supervision}.  The implementation of the {\tt
typedReceive} method, on the other hand, is always provided by users.

The limitation of using inheritance to implement TAkka actors is that Akka 
features are still available to library users.  Unfortunately, this limitation 
cannot be overcome by using delegation because, as we have seen in Figure 2, 
a child actor is created by calling the {\tt actorOf} method from its 
supervisor's actor context, which is a private API of the supervisor.  {\tt 
TypedActor} is the only TAkka class that is implemented using inheritance. 
Other TAkka classes are either implemented by delegating tasks to Akka 
counterparts or rewritten in TAkka.  We believe that re-implementing the TAkka 
Actor library requires a similar amount of work for implementing the Akka Actor 
library.


\subsection{Actor Reference}
\label{actor_ref}
A reference to an actor of type {\tt TypedActor[M]} has type {\tt
ActorRef[M]}.  An actor reference provides a {\tt !} method, through which users
can send a message to the referenced actor.  Sending an actor a
message whose type is not the expected type will raise an error at compile
time. By using type-parameterized actor
references, the receiver does not need to worry about unexpected messages, while
senders can be sure that messages will be understood and processed, as long as
the message is delivered.

An actor usually can react to a finite set of different message patterns 
whereas our notion of actor reference only takes one type parameter.  In a 
type system that supports untagged union types, no special extension is
required.  In a type system which supports polymorphism, {\tt ActorRef} should
be contravariant on its type argument {\tt M}, denoted as {\tt ActorRef[-M]}.
Consider rewriting the simple calculator defined in Section \ref{akkasup} using  
TAkka, it is clear that {\tt ActorRef} is contravariant because {\tt 
ActorRef[Operation]} is a subtype of {\tt ActorRef[Division]} though {\tt 
Division} is a subtype of {\tt Operation}. contravariance is crucial to avoid 
the type pollution problem described at Section \ref{type_pollution}.  

\begin{figure}[h]
\label{ActorRef}
      \begin{lstlisting}[language=scala]
abstract class ActorRef[-M](implicit mt:Manifest[M])  {
  def !(message: M):Unit
  def publishAs[SubM<:M](implicit smt:Manifest[SubM]):ActorRef[SubM]
}
    \end{lstlisting}
    \caption{Actor Reference}
\end{figure}

For ease of use, TAkka provides a {\tt publishAs}
method that casts an actor reference to a version that only accepts a 
subset of supported messages.  The {\tt publishAs} method has three 
advantages. Firstly,  explicit type conversion using {\tt publishAs} is always 
type safe because the type of the result is a supertype of the original actor 
reference.  Secondly, the semantics of {\tt publishAs} does not require a deep 
understanding of underlying concepts like contravariance and inheritance.  
Thirdly, with the {\tt publishAs} method, users can give a supertype of an 
actor reference on demand, without defining new types and recompiling affected 
classes in the type hierarchy.

Figure \ref{akkastring} defines the same string processing actor given in
Section \ref{akka actor}.  The {\tt typedReceive} method now has type {\tt 
String$\Rightarrow$Unit}, which is the same as intended.  In this example, the 
types of {\tt typedReceive} and {\tt m} may be omitted because they 
can be inferred by the compiler.  Unlike the Akka example, sending an integer 
to {\tt server} is rejected by the compiler.  Although the type error 
introduced on line 19 cannot be statically detected, it is captured by the 
run-time as soon as the {\tt actorFor} method is called.  In the TAkka version, 
there is no need to define a handler for unexpected messages.

\begin{figure}
\label{takkastring}
      \begin{lstlisting}[language=scala, escapechar=?]
class ServerActor extends ?\bf{TypedActor[String]}?  {
  def ?\bf{typedReceive}?  = {
    case m:String => println("received message: "+m)
  }
}

object ServerTest extends App {
  val system = ActorSystem("ServerTest")
  val server = system.actorOf(Props[?\bf{String}?, ServerActor], "server")
  
  server ! "Hello World"
//  server ! 3
// compile error: type mismatch; found : Int(3)
//   required: String

  val serverString = system.actorFor?\bf{[String]}?
                     ("akka://ServerTest/user/server")
  serverString ! "Hello World"
  val serverInt = system.actorFor?\bf{[Int]}?
                  ("akka://ServerTest/user/server")
  serverInt ! 3
}

/*
Terminal output:
received message: Hello World
received message: Hello World
Exception in thread "main" java.lang.Exception: 
 ActorRef[akka://ServerTest/user/server] does not 
 exist or does not have type ActorRef[Int] at 
 takka.actor.ActorSystem.actorFor(ActorSystem.scala:223)
 ...
*/
    \end{lstlisting}
    \caption{A String Processor in TAkka}
\end{figure}

\subsection{Props and Actor Context}
\label{actor_context}
The type {\tt Props} denotes the properties of an actor.   A Props of type {\tt 
Props[M]} is used when creating an actor of type {\tt TypedActor[M]}.  Say {\tt 
myActor} is of type {\tt MyActor}, which is a subtype of {\tt TypedActor[M]}, a 
Prop of type {\tt Prop[M]} can be created by one of the APIs in Figure 
\ref{takka_props}:

\begin{figure}
\label{takka_props}
\begin{lstlisting}[language=scala]
 val props:Props[M] = Props[M, MyActor]
 val props:Props[M] = Props[M](new MyActor)
 val props:Props[M] = Props[M](myActor.getClass)
\end{lstlisting}
    \caption{Actor Props}
\end{figure}


Contrary to an actor reference, which is the interface for receiving messages, 
an actor context describes the actor's view of the outside world.   Because 
each actor is an independent computational 
primitive, an actor context is private to the corresponding actor.  By using 
APIs in Figure \ref{ActorContext}, an actor can (i) retrieve an actor reference 
corresponding to a given actor path using the {\tt actorFor} method, (ii) 
create a child actor with a system-generated or user-specified name using one 
of the {\tt actorOf} methods, (iii) set a timeout denoting the time within which
a new message must be received using the {\tt setReceiveTimeout} method, and
(iv) update its behaviours using the {\tt become} method.  Comparing 
corresponding Akka APIs, our methods take an additional type parameter whose 
meaning will be explained below.

\begin{figure}[h]
\label{ActorContext}
      \begin{lstlisting}[language=scala]
abstract class ActorContext[M:Manifest] {
  def actorOf [Msg] (props: Props[Msg])(implicit mt: Manifest[Msg]): ActorRef[Msg]
  def actorOf [Msg] (props: Props[Msg], name: String)(implicit mt: Manifest[Msg]): ActorRef[Msg]
  def actorFor [Msg] (actorPath: String)
       (implicit mt: Manifest[Msg]): ActorRef[Msg]
  def setReceiveTimeout(timeout: Duration): Unit

  def become[SupM >: M](
      newTypedReceive: SupM => Unit,
      newSystemMessageHandler:
                         SystemMessage => Unit,
      newSupervisorStrategy:SupervisorStrategy
  )(implicit smt:Manifest[SupM]):ActorRef[SupM]
}
    \end{lstlisting}
    \caption{Actor Context}
\end{figure}

The two {\tt actorOf} methods are used to create a type-parameterized actor supervised 
by the current actor.  Each actor created is assigned to a typed actor path, 
an Akka actor path together with a {\tt Manifest} of the message type.  Each 
actor system contains a typed name server.  When an actor is created inside an 
actor system, a mapping from its typed actor path to its typed actor reference 
is registered to the typed name server.  The {\tt actorFor} method of {\tt 
ActorContext} and {\tt ActorSystem} fetches typed actor reference from the 
typed name server.


\subsection{Backward Compatible Hot Swapping}
\label{hot_swapping}
Hot swapping is a desired feature of distributed systems, whose
components are typically developed separately. Unfortunately, hot  
swapping is not supported by the JVM, the platform on which the TAkka 
library runs.  To support hot swapping on an actor's receive function, 
system message handler, and supervisor strategy, those three behaviour 
methods are maintained as object references.

The {\tt become} method enables hot swapping on the behaviour of an 
actor.  The {\tt become} method in TAkka is different from behaviour 
upgrades in  Akka in two aspects.  Firstly, the supervisor strategy can be
updated.  In Akka, the supervisor strategy is an immutable value of an 
actor.  We believe the supervisor strategy is an important behaviour of an 
actor and it should be as swappable as message handlers.  Secondly, 
hot swapping in TAkka must be backward compatible.  In other words, an actor 
must evolve to a version that is able to handle the original
message patterns.  The above decision is made so that a service published to 
users will not be unavailable later.  

\begin{comment}I suggest to enforce backward compatible upgrades whenever
possible to provide better user experience.  If a bad design has to be deprecated,
an error message should be returned, if the received message is a synchronous 
request.  Otherwise, the user cannot tell whether the message is deprecated or lost.}
\end{comment}

\begin{figure}[h]
\label{become}
\begin{lstlisting}
abstract class ActorContext[M:Manifest] {
  implicit private var mt:Manifest[M] = manifest[M]

  def become[SupM >: M](
      newTypedReceive: SupM => Unit,
      newSystemMessageHandler:
               SystemMessage => Unit
      newSupervisorStrategy:SupervisorStrategy
  )(implicit smtTag:Manifest[SupM]):ActorRef[SupM] = {
     val smt = manifest[SupM]
     if (!(mt <:< smt))
       throw BehaviorUpdateException(smt, mt)

    this.mt = smt
    this.systemMessageHandler =  newSystemMessageHandler
    this.supervisorStrategy = newSupervisorStrategy
  }
}
\end{lstlisting}
\caption{Hot Swapping in TAkka}
\end{figure}

The {\tt become} method is implemented as in Figure \ref{become}.  The
static type {\tt M} should be interpreted as the least general type of
messages addressed by the actor initialized from {\tt TypedActor[M]}.  The
type value of {\tt SupM} will only be known when the {\tt become} method 
is invoked.  When a series of {\tt become} invocations are made at run 
time, the order of those invocations may be non-deterministic.  Therefore, 
performing dynamic type checking is required to guarantee backward
compatibility.  Nevertheless, static type checking prevents some invalid {\tt 
become} invocations at compile time.

\subsection{Supervisor Strategies}
\label{supervision}

The Akka library implements two of the three supervisor strategies in OTP:
{\tt OneForOne} and {\tt AllForOne}.  If a supervisor adopts the
{\tt OneForOne} strategy, a child will be restarted when it fails.  If a 
supervisor adopts the {\tt AllForOne} supervisor strategy, all children will 
be restarted when any of them fails.  The third OTP supervisor strategy, {\tt
RestForOne}, restarts children in a user-specified order, and hence is not
supported by Akka as it does not specify an order of initialization for
children.  Simulating the {\tt RestForOne} supervisor strategy in Akka
requires ad-hoc implementation that groups related children and defines special
messages to trigger actor termination.  None of the Erlang examples in Section 5
uses the {\tt RestForOne} strategy.  It is not clear whether the lack of
the {\tt RestForOne} strategy will result in difficulties when rewriting Erlang
applications in Akka and TAkka.

Figure \ref{super} gives APIs of supervisor strategies in Akka.  As in OTP, for
each supervisor strategy, users can specify the maximum number of
restarts of any child within a period.  The default 
supervisor strategy in Akka is {\tt OneForOne} that permits unlimited 
restarts.  {\tt Directive} is an enumerated type with the following values: the
{\tt Escalate} action which throws the exception to the supervisor of the 
supervisor, the {\tt Restart} action which replaces the failed child with a new 
one, the {\tt Resume} action which asks the child to process the message again, 
and the {\tt Stop} action which terminates the failed actor permanently.

None of the supervisor strategies in Figure \ref{super} requires a 
type-parameterized classes during construction.  Therefore, from the perspective 
of API design, both supervisor strategies are constructed in TAkka in the same 
way as in Akka.


\begin{figure}[h]
\label{super}
    \begin{lstlisting}    
abstract class SupervisorStrategy
case class OneForOne(restart:Int, time:Duration)(decider: Throwable => Directive) extends SupervisorStrategy
case class OneForAll(restart:Int, time:Duration)(decider: Throwable => Directive) extends SupervisorStrategy
    \end{lstlisting}
    \caption{Supervisor Strategies}
\end{figure}


\subsection{Handling System Messages}
\label{systemmessage}
Actors communicate with each other by sending messages.  To maintain a
supervision tree,
a special category of messages should be handled by all actors.  We define
a trait \footnote{A trait in Scala is similar to a JAVA abstract class, but
trait permits multiple inheritance.} {\tt SystemMessage} to be the
supertype of all messages for system maintenance purposes.  The five Akka
system messages retained in TAkka are given as follows:

\begin{itemize}
  \item {\tt ChildTerminated(child: ActorRef[M])}

  A message sent from a child actor to its supervisor before it terminates.

  \item {\tt Kill}

  An actor that receives this message will send an {\tt ActorKilledException} 
to its supervisor.

  \item {\tt PoisonPill}

  An actor that receives this message will be permanently terminated.  The
supervisor cannot restart the killed actor.
  
  \item {\tt Restart}

  A message sent from a supervisor to its terminated child asking the
child to restart.

  \item {\tt ReceiveTimeout}

  A message sent from an actor to itself when it has not received a message
after a timeout.

\end{itemize}

The next question is whether a system message should be
handled by the library or by users.  In Erlang and early versions of Akka, all
system messages can be explicitly handled by users in the {\tt receive}
block.  In recent Akka versions, some system messages are handled
in the library implementation and are not accessible by library users.

As there are only two kinds of supervisor strategies to
consider, both of which have clearly defined operational behaviours, all
messages related to the liveness of actors are handled in the TAkka library. 
Library users may indirectly affect the system message handler via specifying
the supervisor strategies. In contrast, messages related to the behaviour of an
actor, e.g. {\tt ReceiveTimeout}, are better handled by application
developers. In TAkka, {\tt ReceiveTimeout} is the only system message that can
be explicitly handled by users.  Nevertheless, we keep the {\tt SystemMessage}
trait in the library so that new system messages can be included in the future
when required.

A key design decision in TAkka is to separate handlers for the system messages 
and user-defined messages.  The above decision has two benefits. Firstly,
the type parameter of actor-related classes only need to denote
the type of user defined messages rather than the untagged union of user 
defined messages and the system messages.  Therefore, the TAkka design applies
to systems that do not support untagged union type.  Secondly, since 
system messages can be handled by the default handler, which applies
to most applications, users can focus on the logic of handling user
defined messages.


\subsection{Design Alternatives}
\label{alternative designs}


\paragraph{Akka Typed Actor}
In the Akka library, there is a special class called {\tt TypedActor}, which
contains an internal actor and can be supervised.  A service of {\tt 
TypedActor} is invoked by method invocation instead of message exchanging.  
Code in Figure \ref{akka typed actor} demonstrates how to define a simple 
string processor using Akka typed actor.  The Akka {\tt TypedActor} prevent 
some type errors but have two limitations. Firstly, {\tt TypedActor} does not 
permit hot swapping on its behaviours.  Secondly, avoiding type pollution by 
using Akka typed actors is as awkward as using a plain object-oriented 
model, where supertypes need to be introduced.  In Scala and Java, introducing 
a supertype in a type hierarchy requires modification to all affected classes. 
 



\begin{figure}[h]
\label{akka typed actor}
\begin{lstlisting}
trait MyTypedActor{
  def processString(m:String) 
}
class MyTypedActorImpl(val name:String) extends MyTypedActor{
  def this() = this("default")
  
  def processString(m:String) {
    println("received message: "+m) 
  }
}
object FirstTypedActorTest extends App {
  val system = ActorSystem("MySystem")  
  val myTypedActor:MyTypedActor =
    TypedActor(system).typedActorOf(
    TypedProps[MyTypedActorImpl]())
  myTypedActor.processString("Hello World")
}

/*
Terminal output:
received message: Hello World
 */	
\end{lstlisting}
\caption{Akka TypedActor Example}
\end{figure}

\paragraph{Actors with or without Mutable States}
The actor model formalized by Hewitt et al. \cite{Hewitt:1973} does not 
specify its implementation strategy.  In Erlang, a functional programming 
language, actors do not have mutable states.  In Scala, an
object-oriented programming language, actors may have mutable states.
The TAkka library is built on top of Akka and implemented in Scala.  
As a result, TAkka does not prevent users from defining actors with 
mutable states.
Nevertheless, the authors of this paper encourage the use of 
actors in a functional style, for example encoding the {\tt sender} of a 
synchronous message as part of the incoming message rather than a state 
of an actor, because it is difficult to synchronize mutable
states of replicated actors in a cluster environment.

In a cluster, resources are replicated at different locations to provide 
fault-tolerant services.  The CAP theorem \cite{CAP} shows that it is
impossible to achieve consistency, availability, and partition tolerance in a
distributed system simultaneously.  For actors that use mutable state, system 
providers must either sacrifice availability or partition tolerance, or modify 
the consistency model.  For example, Akka actors have mutable state and Akka 
cluster developers spend great effort to implement an eventual consistency 
model \cite{Kuhn12}. In contrast, stateless services, e.g. RESTful web 
services, are more likely to achieve a good scalability and availability.



\paragraph{Bi-linked Actors}
In addition to one-way linking in the supervision tree, Erlang and Akka
provide a mechanism to define a two-way linkage between actors. 
Bi-linked actors are each aware of the liveness of the other.  We believe that
bi-linked actors are redundant in a system where supervision is
obligatory.  Notice that, if the computation of an actor relies on the liveness
of another actor, those two actors should be organized in the same 
supervision tree.

\begin{comment}
\section{Mixing Static and Dynamic Type Checking}

Implementing {\bf ActorContext} described as \S\ref{actor_context} requires a
mixture of static and dynamic type checking.  This section looks into how
static and dynamic type checking are combined to guarantee backward compatible
code evolution and support type safe data retrieval.

\subsection{Backward Compatible Code Evolution}
\end{comment}
\section{Backward Compatible Code Evolution}
\label{code_evolution}

Partial service upgrade is a desired feature of distributed systems, whose
components are typically developed separately. Unfortunately, hot code swap is
not supported by the JVM, the platform where the TAkka library runs on.  To
support hot swapping on an actor's receive function, system message handler, and
supervision strategy, those three behaviour methods are maintained as object
reference.

Different from Erlang and Akka Design, code evolution in TAkka must be backward
compatible.  In another word, an actor must evolve to a version that is able to
handle the same amount of or more message patterns.  The above decision is made
so that a service published to users would not be unavailable later.

\begin{figure}[h]
\label{become}
\begin{lstlisting}
trait ActorContext[M] {
 implicit var mt:Manifest[_] = manifest[M]

 def become[SupM >: M](
      newTypedReceive: SupM => Unit,
      newSystemMessageHandler:
               SystemMessage => Unit
      newSupervisionStrategy:SupervisionStrategy
 )(implicit smt:Manifest[SupM]):ActorRef[SupM] = {
     if (!(smt >:> mt))
       throw BehaviorUpdateException(smt, mt)

    this.mt = smt
    this.systemMessageHandler =  newSystemMessageHandler
    this.timeoutHandler = newTimeoutHandler
    this.supervisionStrategy = newSupervisionStrategy

    // other code
 }
}
\end{lstlisting}
\caption{Code Evolution in TAkka}
\end{figure}

The {\it become} method is implemented as in Figure-\ref{become}.  The manifest
for static type {\bf M} should be interpreted as the least general type of
messages addressed by the actor initialized from {\bf Actor[M]}.  The
manifest for {\bf SupM} will only be known when the {\it become} method is
invoked.  When a series of {\it become} invocations are made at run time, the
order of those invocations may be non-deterministic.  Therefore, performing a
dynamic type checking is required to guarantee backward compatibility.
Nevertheless, static type checking prevents some invalid {\it become}
invocations at compile time.







\section{Library Evaluation}

This section presents the evaluation results of the TAkka library.  
We show that the Wadler's type pollution problem can be avoided 
in a straightforward way by using TAkka. We further assess the TAkka library by 
porting examples written in Erlang and Akka.  Results show that TAkka 
detects type errors without causing obvious runtime and code-size overheads.



\subsection{Wadler\rq{}s Type Pollution Problem}
\label{type_pollution}

Wadler\rq{}s type pollution problem refers to the situation where a
communication interface of a component publishes too much type information to 
another party and consequently that party can send the component a message not
expected from it.  Without due care, actor-based systems constructed using the
layered architecture or the MVC model \begin{comment}citation? MVC is a 
well-known model\end{comment} 
can suffer from the type pollution problem.

One solution to the type pollution problem is using separate channels for
distinct parties.  Programming models that support this solution includes the
join-calculus \citep{full_join} and the typed $\pi$-calculus \citep{pi_book}.

TAkka solves the type pollution problem by using polymorphism.  Take the
code template in Figure \ref{MVC} for example. Let {\tt V2CMessage} and {\tt 
M2CMessage} be the type of messages expected from the View and the Model
respectively.  Both {\tt V2CMessage} and {\tt M2CMessage} are subtypes of {\tt
ControllerMsg}, which is the least general type of messages expected by the 
controller. In the template code, the controller publishes itself as different 
types to the view actor and the model actor.  Therefore, both the view and the
model only know the communication interface between the controller and itself.
The {\tt ControllerMsg} is a sealed trait so that users cannot define a 
subtype of 
{\tt ControllerMsg} outside the file and send the controller a message of 
unexpected type.  Although type convention in line 25 and line 27 can be 
omitted, we explicitly use the {\tt publishAs} to express our intention and 
let the compiler check the type.  The code template is used to implement the 
Tic-Tac-Toe example in the TAkka code repository.
\begin{comment}
\mycomment{I failed to found small-sized example that uses layered
architecture.  Therefore, I use the core structure of the Tik-Tak-Tok example
here. I implement the Tik-Tak-Tok example for demonstration purpose. I want
to keep the code size small, but the Tik-Tak-Tok example still has 291 lines of
code.}
\end{comment}



\begin{figure}[h]
\label{MVC}
\begin{lstlisting}
sealed trait ControllerMsg
class V2CMessage extends ControllerMsg
class M2CMessage extends ControllerMsg

trait C2VMessage
case class ViewSetController(controller:ActorRef[V2CMessage]) extends 
C2VMessage
trait C2MMessage
case class ModelSetController(controller:ActorRef[M2CMessage]) extends 
C2MMessage
class View extends TypedActor[C2VMessage] {
  private var controller:ActorRef[V2CMessage]
  // rest of implementation
}
Model extends TypedActor[C2MMessage] {
  private var controller:ActorRef[M2CMessage]
  // rest of implementation
}
class Controller(model:ActorRef[C2MMessage], view:ActorRef[C2VMessage]) extends 
TypedActor[ControllerMessage] {
  override def preStart() = {
    model ! ModelSetController( typedSelf.publishAs[M2CMessage])
    view ! ViewSetController( typedSelf.publishAs[V2CMessage])
  }
  // rest of implementation
}
\end{lstlisting}
\caption{Template for Model-View-Controller}
\end{figure}



\subsection{Expressiveness and Correctness}
\label{expressiveness}
Table \ref{express} lists the examples used for expressiveness and 
correctness.  We selected examples from Erlang Quiviq \citep{quviq}
and open source Akka projects to ensure that the main requirements for actor 
programming are not unintentionally neglected.  Examples from Erlang 
Quiviq are re-implemented using both Akka and TAkka.  Examples from 
Akka projects are re-implemented using TAkka.  Following the suggestion in  
\citep{HePa06}, we assess the overall code modification and code 
size by calculating the geometric mean of all examples. The evaluation results 
in Table \ref{express} show that when porting an Akka program to TAkka, about 
7.4\% lines of code need to be modified including additional type declarations. 
Sometimes, the code size can be smaller because TAkka code does not 
need to handle unexpected messages.  On average, the total program size 
of Akka and TAkka applications are almost the same.

A type error is reported by the compiler when porting the Socko example 
\citep{SOCKO} from its Akka implementation to equivalent TAkka implementation.
SOCKO is a library for building event-driven web services.  The SOCKO designer
defines a {\tt SockoEvent} class to be the supertype of all events.  One
subtype of {\tt SockoEvent} is {\tt HttpRequestEvent}, representing events
generated when an HTTP request is received. The designer further implements
subclasses of {\tt Method}, whose {\tt unapply} method intends to pattern
match {\tt SockoEvent} to {\tt HttpRequestEvent}.  The SOCKO
designer made a type error in the method declaration so that the {\tt unapply}
method pattern matches {\tt SockoEvent} to {\tt SockoEvent}. The type error is
not exposed in test examples because the those examples always passes instances 
of {\tt HttpRequestEvent} to the {\tt unapply} method and send the returned
values to an actor that accepts messages of {\tt HttpRequestEvent} type.
Fortunately, the design flaw is exposed when upgrading the SOCKO implementation 
using TAkka.




\begin{table*}[t]
\begin{center}
\begin{tabular}{| p{2.3 cm} | p{2.3 cm} | c | c |  c | c | c |}
\hline

Source & Example & \specialcell{Akka Code \\ Lines} &
\specialcell{Modified\\ TAkka Lines} & \specialcell{\% of \\Modified Code} &
\specialcell{TAkka Code\\ Lines}
& \specialcell{\% of \\Code Size} \\
\hline
Quviq   & ATM simulator & 1148 & 199 & 17.3 & 1160 & 101 \\
\cline{2-7}
\citep{quviq}                                    & Elevator Controller &
2850 & 172 & 9.3 & 2878 & 101 \\
\hline


                                                               & Ping Pong & 67
& 13 & 19.4 & 67 & 100 \\
\cline{2-7}
Akka Documentation                 & Dining Philosophers & 189 & 23 & 12.1 &
189 & 100  \\
\cline{2-7}
 \citep{akka_doc}             & Distributed Calculator  & 250 &
43 & 17.2 & 250 & 100 \\
\cline{2-7}
                                                     & Fault Tolerance & 274 &
69 & 25.2 & 274 & 100 \\
\hline

                                                    & Barber Shop \citep{BarberShop}& 754 & 104 & 
13.7 & 751 & 99 \\
\cline{2-7}
Other Open Source             & EnMAS \citep{EnMAS} & 1916 & 213 & 11.1 & 1909 &
100 \\
\cline{2-7}
  Akka Applications                & Socko Web Server \citep{SOCKO}  & 5024 & 227
& 4.5 & 5017 & 100 \\
\cline{2-7}
                                                     & Gatling \citep{Gatling}
& 1635 & 111 & 6.8 & 1623 & 99 \\
\cline{2-7}
                                                     & Play Core \citep{play_doc}
& 27095 & 15 & 0.05 & 27095 & 100 \\
\hline
geometric mean                   & & 991.7 & 71.6 & 7.4 & 992.1 & 100.0 \\
\hline
\end{tabular}
% }
\caption{Results of Correctness and Expressiveness Evaluation}
\end{center}
\label{express}
\end{table*}

\begin{comment}
\begin{table*}[t]
\label{efficiency_description}

\begin{center}
\begin{tabular}{| l | p{10.5 cm} |}
\hline

Example & Description \\
\hline
bang & This benchmark tests many-to-one message passing.  The benchmark
spawns a specified number sender and one receiver.  Each sender sends a
specified number of messages to the receiver.\\
\hline
big & This benchmark tests many-to-many message passing.  The benchmark
creates a number of actors that exchange ping and pong messages. \\
\hline
ehb & This is a benchmark and stress test.  The benchmark is
parameterized by the number of groups and the number of messages sent from each
sender to each receiver in the same group. \\
\hline
mbrot & This benchmark models pixels in a 2-D image.  For each pixel,
the benchmark calculates whether the point belongs to the Mandelbrot set. \\
\hline
%parallel & This benchmark spawns a number of processes, where a list of
%N timestamps is concurrently created.  \\
%\hline
%genstress & This benchmark is similar to the bang test.  It spawns an
%echo server and a number of clients.  Each client sends some dummy messages to
%the server and waits for its response.  The Erlang version of this test can be
%executed with or without using the gen\_server behaviour.  For generality, this
%benchmark only tests the version without using gen\_server. \\

%\hline
ran & This benchmark spawns a number of processes.  Each process
generates a list of ten thousand random integers, sorts the list and sends the
first half of the result list to the parent process. \\
\hline
 serialmsg & This benchmark tests message forwarding through a
dispatcher. \\
\hline
\end{tabular}
% }
\caption{Examples for Efficiency and Scalability Evaluation}
\end{center}
\end{table*}


\hline
timer\_wheel & This benchmark is a modification to the big test.  While
responding to ping messages, a process in this message also waits pong messages.
 If no pong message is received within the specified timeout, the process
terminates itself. \\
\end{comment}





\subsection{Efficiency, Throughput, and Scalability}
\label{efficiency}

The TAkka library is built on top of Akka so that code for shared features 
can be re-used.  The three main source of overheads in the TAkka implementation
are: (i) the cost of adding an additional operation layer on top of Akka 
code, (ii) the cost of constructing type descriptors, and (iii) the cost of 
transmitting type descriptor in distributed settings. We assess the upper bound 
of the cost of the first two factors by a micro benchmark which assesses the 
time of initializing {\it n} instances of {\tt MyActor} defined in Figure
\ref{fig:akkastring} and Figure \ref{takkastring}. When {\it n} ranges from 
$10^4$ to $10^5$, the TAkka implementation is about 2 times slower as
the Akka implementation.  The cost of the last factor is close to the cost of
transmitting the string representation of fully qualified type names.



The JSON serialization example \citep{techempower} is used to compare the 
throughput of 4 web services built using Akka Play, TAkka Play, Akka 
Socko, and TAkka Scoko.  For each HTTP request, the example gives an 
HTTP response with pre-defined content.  All web services are deployed to 
Amazon EC2 Micro instances (t1.micro), which has 0.615GB Memory. The throughput 
is tested with up to 16 EC2 Micro instances.  For each number of EC2 instances, 
10 rounds of throughput measurement are executed to gather the average and 
standard derivation of the throughput. The results reported in Figure 
\ref{throughput} shows that web servers built using Akka-based library and 
TAkka-based library have similar throughput.

We further investigated the speed-up of multi-node TAkka applications by 
porting 6 micro benchmark examples, from the BenchErl benchmarks in the RELEASE 
project \citep{RELEASE}.  Each BenchErl benchmark spawns one master process and 
many child processes for a tested task.  Each child process is asked to perform 
a certain amount of computation and report the result to the master process.  
The benchmarks are run on a 32 node Beowulf cluster at the Heriot-Watt 
University. Each Beowulf node comprises eight Intel 5506 cores running at
2.13GHz. All machines run under Linux CentOS 5.5. The Beowulf nodes are
connected with a Baystack 5510-48T switch with 48 10/100/1000 ports.

Figure \ref{runtime} and \ref{scalability} reports the results of the BenchErl 
benchmarks.   We report the average and the standard deviation 
of the run-time of each example.  Depending on the ratio of the computation 
time and the I/O time, benchmark examples scale at different levels.  In 
all examples, TAkka and Akka implementations have almost identical 
run-time and scalability.


\begin{figure}[h]
     \begin{center}
        \subfigure[]{
            \label{fig:play_throughput}
            \includegraphics[scale=0.25]{Play_throughput.png}
        }
        \subfigure[]{
           \label{fig:socko_throughput}
           \includegraphics[scale=0.25]{Socko_throughput.png}
        }
    \end{center}
     \caption{Throughput Benchmarks}
   \label{throughput}
\end{figure}

\begin{figure}[p]
     \begin{center}
        \subfigure[]{
            \label{fig:2_1}
            \includegraphics[scale=0.16]{Bang_time.png}
        }
        \subfigure[]{
           \label{fig:2_2}
           \includegraphics[scale=0.16]{Big_time.png}
        }
        \subfigure[]{
            \label{fig:2_3}
            \includegraphics[scale=0.16]{EHB_time.png}
        }\\
        \subfigure[]{
            \label{fig:2_5}
            \includegraphics[scale=0.16]{MBrot_time.png}
        }
        \subfigure[]{
            \label{fig:2_7}
            \includegraphics[scale=0.16]{RUN_time.png}
        }
        \subfigure[]{
           \label{fig:2_8}
           \includegraphics[scale=0.16]{SerialMsg_time.png}
        }\\
    \end{center}
    \caption{Runtime Benchmarks 2}
   \label{runtime}
\end{figure}
\begin{figure}[p]
     \begin{center}
        \subfigure[]{
            \label{fig:2_9}
            \includegraphics[scale=0.16]{Bang_speedup.png}
        }
        \subfigure[]{
           \label{fig:2_10}
           \includegraphics[scale=0.16]{Big_speedup.png}
        }
        \subfigure[]{
            \label{fig:2_11}
            \includegraphics[scale=0.16]{EHB_speedup.png}
        }\\
        \subfigure[]{
            \label{fig:2_13}
            \includegraphics[scale=0.16]{MBrot_speedup.png}
        }
        \subfigure[]{
            \label{fig:2_15}
            \includegraphics[scale=0.16]{RUN_speedup.png}
        }
        \subfigure[]{
           \label{fig:2_16}
           \includegraphics[scale=0.16]{SerialMsg_speedup.png}
        }\\        
    \end{center}
     \caption{Scalability Benchmarks 2}
   \label{scalability}
\end{figure}




\section{Conclusion}

Programs written in Actor Model should be easy to reason their intentions
\cite{Hewitt:1973}.  Unfortunately, actors in Erlang and Akka accept
dynamically typed messages and therefore have vague contracts with their outside
world.  The TAkka library introduces a type-parameter for actor-related class.
The additional type-parameter specifies the intention of that actor's solo
communication interface. More importantly, with the help of type-parametrized
actors, the implementation of an actor can be statically checked against the
intention of that actor.

In addition to eliminating programming bugs and type errors, programmers would
like to have reliable failure recovery mechanisms for unexpected run-time
errors.  Supervision tree \cite{OTP} is such a mechanism that has improved the
reliability of a large telecom system written in Erlang.  We are glad to see
that type-parametrized actors can form supervision trees in the same way as
regular actors.

Lastly, tests show that building type-parametrized actor on top of Akka actor
does not introduce significant overheads, respecting to program size,
efficiency, and scalability.  The above test result is encouraging in the sense
that a large amount of previous actor library implementation could be re-used.
We expect similar results could be obtained in other OTP-like libraries such as
the future version of CloudHaskell.


% \acks

% Acknowledgments



\bibliographystyle{plain}
\bibliography{Jiansen}

\appendix
%\chapter{Akka and TAkka API}

\begin{figure}[!h]
\label{akka_api}

\begin{lstlisting}[language=scala, escapechar=?]
package ?\textcolor{blue}{akka}?.actor   
\end{lstlisting}
\vspace{-15pt }
    
\begin{lstlisting}[language=scala, escapechar=?]
abstract class ActorRef
  def !(message: ?\textcolor{blue}{Any}?):Unit
\end{lstlisting}
    \vspace{-15pt }
    
\begin{lstlisting}[language=scala, escapechar=?]
?\textcolor{blue}{trait }? Actor
  def ?\textcolor{blue}{receive:PartialFunction[Any, Unit]}?
  val ?\textcolor{blue}{self: ActorRef}?
  private val ?\textcolor{blue}{context: ActorContext}?
  var supervisorStrategy: SupervisorStrategy
\end{lstlisting}
    \vspace{-15pt }
    
\begin{lstlisting}[language=scala, escapechar=?]
?\textcolor{blue}{trait}? ActorContext
  def actorOf(props: Props): ActorRef
  def actorOf(props: Props, name: String): ActorRef
  def actorFor(path: String): ActorRef
  def setReceiveTimeout(timeout: Duration): Unit
  def become(behavior: ?\textcolor{blue}{PartialFunction[Any, Unit]}?, 
  						?\textcolor{blue}{discardOld: Boolean = true): Unit}?
?\textcolor{blue}{def unbecome(): Unit}?
    \end{lstlisting}
    \vspace{-15pt }
    
    \begin{lstlisting}[language=scala, escapechar=?]
?\textcolor{blue}{final case class Props(deploy: Deploy,}? ?\textcolor{blue}{clazz: Class[\_], }?
                           ?\textcolor{blue}{args: immutable.Seq[Any]) }?
    \end{lstlisting}
    \vspace{-15pt }
        
    \begin{lstlisting}[language=scala, escapechar=?]
object Props extends Serializable
  def apply(creator: =>Actor): Props
  def apply(actorClass: Class[_ <: Actor]): Props  
  def apply[T <: Actor]() (implicit arg0: Manifest[T]): Props  
    \end{lstlisting}
    \vspace{-15pt }
    
    \begin{lstlisting}    
abstract class SupervisorStrategy
case class OneForOneStrategy(restart:Int = -1, 
              time:Duration = Duration.Inf)
              (decider: Throwable =>  Directive) 
              extends SupervisorStrategy
case class OneForAllStrategy(restart:Int = -1, 
              time:Duration = Duration.Inf)
              (decider: Throwable =>  Directive) 
              extends SupervisorStrategy
    \end{lstlisting}
    \vspace{-15pt }
        
    \caption{Akka API}
    \vspace{-20pt }
\end{figure}


\begin{figure}[!h]
\label{takka_api}
    \begin{lstlisting}[language=scala, escapechar=?]
package ?\textcolor{blue}{takka}?.actor   
\end{lstlisting}

    \begin{lstlisting}[language=scala, escapechar=?]
abstract class ActorRef?\textcolor{blue}{[-M](implicit mt:Manifest[M])}?
  def !(message: ?\textcolor{blue}{M}?):Unit
  ?\textcolor{blue}{def publishAs[SubM<:M]}? ?\textcolor{blue}{(implicit smt:Manifest[SubM]):ActorRef[SubM]}?
    \end{lstlisting}
    
    \begin{lstlisting}[language=scala, escapechar=?]
?\textcolor{blue}{abstract class}? Actor?\textcolor{blue}{[M:Manifest]}?  extends  akka.actor.Actor
  def ?\textcolor{blue}{typedReceive:M=>Unit}?
  val ?\textcolor{blue}{typedSelf:ActorRef[M]}?
  private val ?\textcolor{blue}{typedContext:ActorContext[M]}?
  var supervisorStrategy: SupervisorStrategy
\end{lstlisting}

      \begin{lstlisting}[language=scala, escapechar=?]
?\textcolor{blue}{abstract class}? ActorContext?\textcolor{blue}{[M:Manifest]}?
  def actorOf ?\textcolor{blue}{[Msg]}? (props:  Props?\textcolor{blue}{[Msg])}?
  						?\textcolor{blue}{(implicit mt: Manifest[Msg]}?): ActorRef?\textcolor{blue}{[Msg]}?
  def actorOf ?\textcolor{blue}{[Msg]}? (props: Props?\textcolor{blue}{[Msg]}?, name: String)
  						?\textcolor{blue}{(implicit mt: 
Manifest[Msg])}?: ActorRef?\textcolor{blue}{[Msg]}?
  def actorFor ?\textcolor{blue}{[Msg]}? (path: String)
       				?\textcolor{blue}{(implicit mt: 
Manifest[Msg])}?:  ActorRef?\textcolor{blue}{[Msg]}?
  def setReceiveTimeout(timeout: Duration): Unit
  def become?\textcolor{blue}{[SupM >: M]}?(behavior: ?\textcolor{blue}{SupM=>Unit}?)
						?\textcolor{blue}{(implicit smt:Manifest[SupM]):ActorRef[SupM]}?

?\textcolor{blue}{case class BehaviorUpdateException(smt:Manifest[\_],  mt:Manifest[\_]) extends \ 
Exception(smt + "must be a supertype of "+mt+".")}?
    \end{lstlisting}
    
    \begin{lstlisting}    [language=scala, escapechar=?]
?\textcolor{blue}{final case class Props[-T] (props: akka.actor.Props)}?
    \end{lstlisting}
    \begin{lstlisting}        [language=scala, escapechar=?]
object Props extends Serializable
  def apply?\textcolor{blue}{[T]}?(creator: => ?\textcolor{blue}{Actor[T]}?): Props?\textcolor{blue}{[T]}?
  def apply?\textcolor{blue}{[T]}?(actorClass: Class[_<: Actor?\textcolor{blue}{[T]]}?):Props?\textcolor{blue}{[T]}?
  def apply?\textcolor{blue}{[T, A<:Actor[T]]}?  (implicit arg0: Manifest[A]): Props?\textcolor{blue}{[T]}
    \end{lstlisting}
    
    \begin{lstlisting}    
abstract class SupervisorStrategy
case class OneForOneStrategy(restart:Int = -1, 
              time:Duration = Duration.Inf)
              (decider: Throwable =>  Directive) 
              extends SupervisorStrategy
case class OneForAllStrategy(restart:Int = -1, 
              time:Duration = Duration.Inf)
              (decider: Throwable =>  Directive) 
              extends SupervisorStrategy
    \end{lstlisting}
    
    \caption{TAkka API}
\end{figure}
%%\section{Research Plan and Risk Analysis}

\subsection{Plan for the Second Year}
In the rest of the second year, I will port core components of the Riak distributed database\cite{riak}.  I will summarise problems that I encountered when translating the three examples into Scala using akka.  More translation exercises will be done if proper examples are found.

Apart from the above, I will continue the the design and implementation of a library that supports OTP principles.  This work will require iterative and incremental development on requirement analysis, interface design, functionality implementation, and system evaluation.  Pace of progress will be recorded for project evaluation.

\subsection{Plan for the Third Year}
The beginning of the third year will continue the work of library implementation. After that, the library will be published with a technical report that covers:  
\begin{inparaenum}[(i)]
  \item functionalities provided by the library,
  \item alternative choices of library design and reasons for adopting the chosen design,
  \item benchmark results of selected canonical distributed applications.
\end{inparaenum}  

In the meantime of receiving feedbacks from programming communities, I will re-implement the selected medium-sized applications with the new library.  Testing results for the new implementation will be published in an updated version of library documentation.

Finally, a more comprehensive summary of common challenges in distributed programming and their solutions will be given.  The summary will also illustrate how type theories and OTP principles guided the library design.   It may also reveal research opportunities for future study.

\subsection{Potential Risks}
As a novel and ambitious research, some tasks may be tougher than my expectation.  Also, results of some tasks may differ from my intuition.  Risk analyses for planed objectives are given below.

Firstly, sample applications used in the first and second objectives, which are the basis of problems identification and results evaluation, must be carefully selected.  It would be better to select open-source applications implemented in high-quality code.  Fortunately, among five recommended examples, we have identified three appropriate applications: ATM simulator, elevator controller, and Riak core.  The other two examples, Erlang Solutions IMS and EJABBERD, are considered inappropriate for their hight ratio of implementation for communication protocols to implementation for OTP principles.

Secondly, some research problems rooted in the centre of distributed programming, and therefore is inevitably also investigated by other researchers.  Although it seems feasible to re-implement identified applications in other frameworks and then identify problems of using those frameworks, our chosen comparing frameworks are vigorous and evolving quickly.  For example, the new akka library (version 2.0) forces every actor to be part of a supervision tree.  The evolving of comparing frameworks contributes to plausible solutions to some research problems as well as pressures of providing devised simpler solutions for tough problems.  In addition, since there is no uniform model for distributed programming, it would be hard to evaluate to what extend our library interface is a general framework for distributed programming.

Lastly, apart from above general difficulties, I also encountered implementation difficulties listed in \S\ref{problems}.  At the time of this writing, it is not clear to me how those problems could be solved in neat ways.  I anticipate that new problems will be added to the list when I write more programs in type safe-less platforms.

Despite the identified and potential troubles which may distract this research from a perfect result, none of above risks should be considered as a principle hindrance to main objectives listed in \S\ref{objectives}.  With a proper plan, wide sources of suggestions, and the commitment to research objectives, the proposed research should be suitable for a PhD project.

\end{document}
