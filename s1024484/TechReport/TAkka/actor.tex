\section{Adding Type Parameter to Actor Library}

This section examines actor programming in the Akka library and how type parameters are added to actor-related classes to improve type safety.  We show that supervision trees in TAkka are constructed in the same way as in Akka.  This section concludes with a brief discussion about alternative designs used by Akka and other actor libraries.

\mycomment{Explain Actor Programming.  Compare Akka and TAkka side by side.}

\subsection{The Actor Class}

An Actor has four important fields given in Figure-\ref{actor_api}:
(i) a receive function that defines its reaction to incoming messages, (ii)
an actor reference pointing to  itself, (iii) the actor  context representing
the outside world of the actor, and (iv) the supervisor strategy for its
children.


An TAkka {\bf Actor} has Akka equivalent fields as shown in
Table \ref{actor_api}.  Different from other actor libraries, every TAkka
actor class takes a type parameter {\bf M} which specifies the type of messages
it expects to receive.  The same type parameter is used as the input type of the
receive function, the type parameter of actor context and the type parameter of
the actor reference pointing to itself.  We introduce a new field $mt$ to inform the compiler that the type parameter of the {\bf Actor} class should be recorded.

\begin{table}[h]
\label{actor_api}
\caption{Akka and TAkka Actor}
%  \begin{adjustwidth}{-1.8cm}{}
  \begin{tabular}{ l   l }
\begin{lstlisting}[language=scala]
package akka.actor
trait Actor {

  def typedReceive:Any=>Unit
  val typedSelf:ActorRef
  val typedContext:ActorContext
  var supervisorStrategy:
         SupervisorStrategy
}
\end{lstlisting} &
\begin{lstlisting}[language=scala]
package takka.actor
trait Actor[M] {
  implicit val mt : TypeTag[M]
  def typedReceive:M=>Unit
  val typedSelf:ActorRef[M]
  val typedContext:ActorContext[M]
  var supervisorStrategy: 
         SupervisorStrategy
}
\end{lstlisting}
  \end{tabular}
%  \end{adjustwidth}
\end{table}



The three immutable fields of TAkka {\bf Actor}: {\it mt}, {\it typedContext} and {\it
typedSelf}, will be initialised automatically when the actor is created.
Application developers may override the default supervisor strategy in the way
explained in \S\ref{supervision}.  The implementation of the {\it typedReceive}
method, on the other hand, is left to developers.


\subsection{Actor Reference}
\label{actor_ref}
A reference pointing to an Actor of type {\bf Actor[M]} has type {\bf ActorRef[M]}.  It provides a {\it !} method, through which users could send a message of type {\bf M} to the referenced actor.  Sending an actor reference a message whose type is not the
expected type will raise a compile error.  By using type-parametrized actor
references, the receiver does not need to worry about unexpected messages while
senders can ensure that messages will be understood and processed, as long as
the message is delivered.

In a type system which supports polymorphism, {\bf ActorRef} should be
contravariant.  We further provide a {\it publishAs} method which type-safely
casts an actor reference to a version of its supertype.  In another word, the
output actor reference accepts partial forms of messages that are accepted by
the input actor reference.  The ability of publishing partial services makes
TAkka a good tool for solving security problems such as the type pollution
problem described at \S\ref{type_pollution}.

\begin{table}[h]
\label{ActorRef}
  \begin{tabular}{ l  l }
      \begin{lstlisting}[language=scala]
abstract class ActorRef {

  def !(message: Any):Unit
  
  
}
    \end{lstlisting} 
    &
    \begin{lstlisting}[language=scala]
abstract class ActorRef[-M]
    (implicit mt:TypeTag[M]) {
  def !(message: M):Unit
  def publishAs[SubM<:M]
      (implicit mt:TypeTag[SubM]):ActorRef[SubM]
}
    \end{lstlisting}     
  \end{tabular}
    \caption{Actor Reference}
\end{table}

The code below defines and uses a string processing actor in Akka and TAkka.  The receive
function of Akka Actor has type {\bf Any$\Rightarrow$Unit} but the defined actor
only intends to process strings. Both examples create an actor inside an actor system
and returns a reference pointing to that actor.  The same string message is sent to actor in both examples
and processed in the way defined in the receive function.  Sending an integer message, which is not expected
by both actors; However, is permitted in the Akka version but rejected by the TAkka version.

\begin{table}[h]
\label{ActorRef}
     \begin{adjustwidth}{-0.8cm}{}
  \begin{tabular}{ l   l }
      \begin{lstlisting}[language=scala]
class MyActor extends Actor {
  def receive:Any => Unit = {
    case m:String =>
      println("Received message: " + m)
  }
}

val system = ActorSystem("MySystem")
val actorRef:ActorRef =
    system.actorOf(Props[MyActor])
actorRef ! "Hello World!"
actorRef ! 3



/*
Terminal output:
Received message: Hello World!
*/
    \end{lstlisting}
&
      \begin{lstlisting}[language=scala]
class MyActor extends Actor[String] {
  def typedReceive:String=>Unit = {
    case m:String =>
      println("received message: "+m)
  }
}

val system = ActorSystem("MySystem")
val actorRef:ActorRef[String] = 
    system.actorOf(Props[String, MyActor])
actorRef ! "Hello World!"
// actorRef ! 3 
// compile error: type mismatch; found : Int(3)
//   required: String

/*
Terminal output:
Received message: Hello World!
*/
    \end{lstlisting}

  \end{tabular}
  \end{adjustwidth}
    \caption{A Actor String Processor}
\end{table}

Undefined messages are treated differently in different actor libraries.  In
Erlang, an actor keeps undefined messages in its mailbox, attempts to process
the message again when a new message handler is in use.  In versions prior to
2.0, an Akka actor raises an exception when it processes an undefined message.
In recent Akka versions, undefined message is discarded by the actor and an {\bf
UnhandledMessage} event is pushed to the event stream of the actor system. The
event stream may be subscribed by other actors to which all events will be
published.  In the example code no subscriber of the event stream is defined and
the integer message is simply discarded.


\subsection{Props and Actor Context}
\label{actor_context}
\mycomment{Explain Akka version and TAkka version at the same time}

A Props is the configuration of actor creation.  A Props of type {\bf Prop[M]}
specifies how to create an actor of type {\bf Actor[M]} and returns an actor
reference of type {\bf ActorRef[M]}.  A Prop should be created by one of
following APIs, where {\bf MyActor} should be a subtype of {\bf Actor[M]}:

\begin{table}[h]
\label{Props}
     \begin{adjustwidth}{-1.2cm}{}
  \begin{tabular}{ l  l }

\begin{lstlisting}[language=scala]
 val props:Props = Props[MyActor]
 val props:Props = Props(new MyActor)
 val props:Props = Props(myActor.getClass)
\end{lstlisting}
&
\begin{lstlisting}[language=scala]
 val props:Props[M] = Props[M, MyActor]
 val props:Props[M] = Props[M](new MyActor)
 val props:Props[M] = Props[M](myActor.getClass)
\end{lstlisting}
  \end{tabular}
     \end{adjustwidth}
    \caption{Props Creation API}
\end{table}
  

Contrary to actor reference, an actor context describes the outside world of the
actor.   For security consideration, actor context is only available inside the
actor definition.  By using APIs in Figure \ref{ActorContext}, an actor can (i)
retrieving an actor reference according to a given actor path, (ii) creating a
child actor with system generated or user specified name, (iii) set a timeout
during which period a new message shall be received, and (iv) update its
behaviours.

The {\it ActorFor} method in the {\bf ActorContext} classes returns an actor
reference of the desired type, if the actor located at the specified actor path
has a compatible type.  To implement the {\it ActorFor} method, we would like to
have a more general mechanism that will return a value of the desired type
when the corresponding key is given.  To this end, we designed and implemented a
typed name server which will be explained \S\ref{nameserver}.

The {\it become} method enables hot code swap on the behaviour of an actor.  The
{\it become} method in TAkka is different from code swap in Akka in two aspects.
 Firstly, the supervision strategy could be updated as well.  Secondly, the new
receive function must be more general than the previous version.  As a result,
no stack of receive functions is required.  Interestingly, the implementation of
{\it become} involves a mix of static and dynamic type checks.  Details of the
implementation will be discussed in \S\ref{code_evolution}.

\begin{table}[h]
\label{ActorContext}
   \begin{adjustwidth}{-1.3cm}{}
  \begin{tabular}{ l   l }
      \begin{lstlisting}[language=scala]
trait ActorContext {
  def actorFor(actorPath:String):ActorRef
  
  def actorOf(props: Props):ActorRef
  def actorOf(props: Props, 
              name: String): ActorRef
  def setReceiveTimeout
             (timeout: Duration): Unit

  def become(
      newReceive: Any => Unit,
  ):ActorRef
}



    \end{lstlisting}
&
    \begin{lstlisting}[language=scala]
trait ActorContext[M] {
  def actorFor [Msg] (actorPath: String)
       (implicit mt: TypeTag[Msg]): ActorRef[Msg]
  def actorOf[Msg](props: Props[Msg])
        (implicit mt: TypeTag[Msg]): ActorRef[Msg]
  def actorOf[Msg](props: Props[Msg], name: String)
        (implicit mt: TypeTag[Msg]): ActorRef[Msg]
  def setReceiveTimeout(timeout: Duration): Unit

  def become[SupM >: M](
      newTypedReceive: SupM => Unit,
      newSystemMessageHandler:
                         SystemMessage => Unit,
      newSupervisionStrategy:SupervisionStrategy
  )(implicit smt:TypeTag[SupM]):ActorRef[SupM]
}
    \end{lstlisting}
    \end{tabular}
     \end{adjustwidth}
    \caption{Actor Context}
\end{table}


\subsection{Supervision Strategies}
\label{supervision}

There are three supervision strategies defined in Erlang/OTP: one-for-one,
one-for-all, and rest-for-one\cite{OTP}.  If a supervisor adopts the one-for-one
supervision strategy, a child will be restarted when it fails.  If a supervisor
adopts the one-for-all supervision strategy, all children will be restarted when
any of them fails.  In Erlang/OTP, children are started in a user-specified
order.  If a supervisor adopts the rest-for-one supervision strategy, all
children started after the failed child will be restarted.  For each
supervision strategy, users can further specify the maximum restarts of any
child within a period.

The Akka library only considers the one-for-one strategy and the one-for-all
strategy.  The rest-for-one strategy is not considered because children are not
created according to a user-specified order.  The default supervision strategy
is a one-for-one strategy that permits unlimited restarts.  Users can define
their own supervision strategy by using APIs given in Figure-\ref{super}.  {\bf
OneForOne} corresponds to the one-for-one strategy in Erlang whereas {\bf
OneForAll} corresponds to the all-for-one strategy in Erlang.  Both
strategies are constructed by providing required parameters.  {\bf Directive} is
an enumerable type whose value is {\bf Escalate}, {\bf Restart}, {\bf Resume},
or {\bf Stop}.  Notice that neither supervision strategies require any type-parametrized class. Therefore, both supervision strategies
are constructed in TAkka in the same way as in Akka.


\begin{figure}[h]
\label{super}
    \begin{lstlisting}    
abstract class SupervisorStrategy
case class OneForOne(restart:Int, time:Duration)
                    (decider: Throwable => Directive) extends SupervisorStrategy
case class OneForAll(restart:Int, time:Duration)
                    (decider: Throwable => Directive) extends SupervisorStrategy
    \end{lstlisting}
    \caption{Supervision Strategies}
\end{figure}


\subsection{Handling System Messages}

Actors are communicated with each other by sending messages.  To maintain a
supervision tree, for example to monitor and control the liveness of actors, a
special category of messages should be addressed by all actors.  We define a
trait {\bf SystemMessage} to be the supertype of all messages for system
maintenance purposes.  Based on the design of Erlang and Akka, we consider
following messages should be included in system messages:

\begin{itemize}
  \item {\bf ChildTerminated(child: ActorRef[M])}  

  a message sent from a child actor to its supervisor before it terminates.

  \item {\bf Kill}

  a message sent from a supervisor to its child.

  \item {\bf Restart}
 
  a message sent from a supervisor to its terminated child asking to restart the child.

  \item {\bf ReceiveTimeout}

  a message sent from an actor to itself when it did not receive any message after a timeout.

\end{itemize}

An open question is which system messages are allowed to be handled by users.
In Erlang and early Akka versions, all system messages could be explicitly
handled by users in the {\it receive} block.  In recent Akka versions, there is
a consideration that some system messages should be handled in library
implementation but not be handled by library users.

Considering that there are only two supervision strategies to consider, both of
which have clearly defined operational behaviours, all messages related to the
liveness of actors are handled in the TAkka library implementation.  General
library users may indirectly affect the system message handler via specifying
the supervision strategies.  On the contrary, messages related to the behaviour
of an actor, e.g. ReceiveTimeout, are better to be handled by application
developers.  In TAkka, {\bf ReceiveTimeout} is the only system message that can
be explicitly handled by users.

\subsection{Alternative Designs}
\label{alternative designs}


\paragraph{Akka Typed Actor}
In the Akka library, there is a special class called {\bf TypedActor}, which
contains an internal actor and could be supervised.  Users of typed actor invoke
a service by calling a method instead of sending messages.  The typed actor
prevents some type errors but has two limitations.  For one thing, typed actor
does not permit code evolution.  For the other, avoiding type pollution when
using typed actor would be as awkward as using a plain objected-oriented model.

\paragraph{Actors with or without Mutable States}
The actor model formalised by Hewitt et al. \cite{Hewitt:1973} does not specify
its implementation strategy.  In Erlang, a functional programming language,
actor does not have mutable states.  In Scala, an objected-oriented
programming language, actor may have mutable states.  The TAkka library is
built on top of Akka and implemented in Scala as well.  As a result, TAkka does
not prevent users from defining actors with mutable states.  Nevertheless, the
library designers would like to encourage using actors in a functional
style because actors with mutable states is difficult to synchronize in a
cluster environment.

In a cluster, resources are replicated at different locations to provide
efficient fault-tolerant services.  Known as the CAP theorem \cite{CAP}, it is
impossible to achieve consistency, availability, and partition tolerance in a
distributed system simultaneously.  For actors without mutable state, system
providers do not need to worry about consistency.  For actors that contain
mutable states, system providers have to either sacrifice availability or
partition tolerance, or modify the consistency model.  For example, Akka actor
has mutable states and Akka cluster employs an eventual consistency
model \cite{Kuhn12}.

\paragraph{Linked Actors}
Alternative to forming supervision trees, reliability of actor-based programs
could be improved by linking related actors \cite{ErlangWeb}. Linked actors are
aware of the death of each other.  Indeed, supervision tree is a special
structure for linking actors.  For this reason, we consider actor linking is a
redundant design in a system where supervision is obligatory.  After all, if
the computation of an actor relies on the liveness of another actor, those two
actors should be organised in the same logic supervision tree.





