\section{Background}
\label{background}



\subsection{Actor Model and OTP Design Principles}

The Actor model defined by Carl Hewitt treats actors as primitive computational
components.  Actors collaborate by sending asynchronous messages to each other.
An actor independently determines its reaction to messages it receives.
Furthermore, an actor should confirm its {\it intention} to the context where
it resides.  Carl Hewitt defines intention as ``the contract that the actor
has with the outside world''.\cite{Hewitt:1973}


The Actor model is adopted by Erlang programming language, whose developers
later summarized five OTP design principles to improve the reliability of Erlang
applications\cite{OTP}.  Table-\ref{otp} lists OTP design principles and their
possible implementation methods in an objected-oriented language.  Apart from
the Supervision Tree principle,  an important subject to be discussed later,
other OTP design principles can be straightforwardly implemented in an
objected-oriented language.  For a platform such as Java Virtual Machine (JVM)
where hot swapping is not supported in general, hot swapping a particular
component could be simulated by updating the reference to that component.
\S\ref{code_evolution} explains how hot swapping the behaviour of an actor is
supported in TAkka, which is run on JVM.



\begin{table}[h]
\label{otp}
% \tbl{Implementing OTP Design Principles in an OO Language}{
 \caption{Implementing OTP Design Principles in an OO Language}
 \begin{center}
\begin{tabular}{| l | p{7.8cm} | }
\hline
  OTP Design Principle & Objected-Oriented\\
                       & Implementation \\
\hline
  Supervision Tree  & no direct correspondence  \\
\hline
  Behaviour & defining an abstract class \\
\hline
  Application  & defining an abstract class that has two abstract methods: start and stop \\
\hline
  Release  & packaging related application classes  \\ 
\hline
  Release Handling  & hot swapping support on key modules is required \\
\hline
\end{tabular}
 \end{center}
%}
\end{table}


\subsection{Scala Type Descriptors}
The TAkka library is designed to detect type errors as early as possible.  For
type errors that cannot be statically detected, adequate dynamic type checking
may expose them before the program goes wrong.  To this end, a notion of
run-time type descriptor is required.

In versions since Scala 2.10.0, first class type descriptor for type {\bf T},
which has type {\bf Type}, could be constructed from {\it typeOf[T]}.  In the
case that {\bf T} is a generic type, a {\bf TypeTag} for {\bf T}, constructed
from {\it typeTag[T]} must be provided.  In addition, with {\it implicit
parameter} (e.g. line 4 in the code below) and {\it context bound} (e.g. line 10
in in the code below), library designer could define concise APIs.  For example,
in a Scala interpreter, one could obtain the following results:

\begin{lstlisting}
  scala> typeOf[Int => String]
  res0: Type = Int => String

  scala> def name[T](implicit m: TypeTag[T]) = typeOf[T]
  name:[T](implicit m:TypeTag[T])Type

  scala> name[Int=>String]
  res3: Type = Int => String

  scala> def name2[T:TypeTag] = typeOf[T]
  name2: [T](implicit evidence$1: reflect.runtime.universe.TypeTag[T])Type

  scala> name2[Int => String]
  res2: Type = Int => String
\end{lstlisting}

In TAkka, dynamic type checking is achieved by comparing type descriptors rather
than checking the most precise type of a value.  The adopted approach has two
advantages.  For one thing, it detects type errors otherwise may be left out.
For example, let {\bf S} be a subtype of {\bf T}, the output of function {\it f}
has type {\bf T} and the input of function {\it g} should have type {\bf S}, it
is clear that composing {\it f} and {\it g} is not always type safe.  In an
environment where static type checking is not possible, comparing type
descriptors of {\bf S} and {\bf T} reveals the type error instantly whereas the
alternative approach may admit some test cases.  For another, data security is
improved since values are not disclosed until type safety is confirmed.

\subsection{Type Parameter and Generic Programming}

\mycomment {  Short examples of Generic Collection in Scala, e.g. List[E].  Show how static type-checking avoid type-errors.}


\subsection{Typed Nameserver}
\label{nameserver}

\mycomment{use this example to show the need and benefits of mixing static and dynamic type checking.}

A canonical nameserver maps each registered name, a unique string, to a value,
and to look up a value by giving a name.  A name could be encoded as a {\bf
Symbol} in Scala so that names which represent the same string have the same
value.  We noticed that the widely used canonical namesever has two limitations.
 For one thing, values retrieved from a canonical nameserver are dynamically
typed.  Therefore, circumspect programmers would have to manually insert type
checking code to prevent type errors in later program.  For another, the
canonical nameserver has a weak security barrier in the sense that values will
be returned to an enquirer regardless of his knowledge that how the returned
data values should be used.

To overcome the above limitations of canonical nameserver, we designed and
implemented a typed nameserver which maps each registered typed name to a value
of the corresponding type, and to look up a value by giving a typed name.  A
typed name, {\bf TSymbol}, is a name shipped with a type descriptor.  In Scala,
{\bf TSymbol} could be simply defined as follows:

\begin{figure}[h]
\label{tsymbol}
\begin{lstlisting}
case class TSymbol[T:Type](val s:Symbol) {
    private [takka] def t:Type[T] = typeOf[T]
    override def hashCode():Int = s.hashCode()  
}
\end{lstlisting}
\caption{TSymbol}
\end{figure}

Notice that {\bf TSymbol} is covariant because the same type value is passed to
{\bf Manifest}, another covariant type constructor.  In addition, {\bf TSymbol}
is declared as a {\it case class} in Scala so that the name {\bf TSymbol} could
be used as a data constructor and pattern matching on parameters of {\bf
TSymbol} is supported automatically.  Finally, the manifest is shipped as an
implicit value parameter and given a name so that general users do not need to
manually construct the manifest and the library developer could access the
manifest as a field of {\bf TSymbol}.

With the help of {\bf TSymbol}, we implemented a typed nameserver which provides
following three operations:
\begin{itemize}
  \item {\it set[T:Manifest](name:TSymbol[T], value:T):Unit}

Register a typed name with a value of corresponding type.  A {\bf
NamesHasBeenRegisteredException} will raise if the symbol representation of
$name$ has been registered.  The most precise type of $value$ shall be a subtype
of the intention type {\bf T}.

  \item {\it unset[T](name:TSymbol[T]):Unit}

Cancel the entry $name$ if (i)  its symbol representation is registered and (ii)
the the intention type {\bf T} is a super type of the registered type.

  \item {\it get[T] (name: TSymbol[T]): Option[T]}

Return Some(v:{\bf T}) if (i) $name$ is associated with a value and (ii) {\bf T}
is a supertype of the registered type; Otherwise return None.
\end{itemize}

Notice that $unset$ and $get$ operations permit polymorphism.  To this end, the
{\it hashcode} method of {\bf TSymbol} defined in Figure-{\ref{tsymbol}} does
not take manifest into account.  Instead, type information are checked at run
time. The typed nameserver also prevents users from registering two names with
the same symbol but different manifests, in which case if one is a supertype of
another, the return value of $get$ will be non-deterministic.

\begin{comment}
\subsection{Obligatory Supervision}

System reliability will be improved if failed components are restarted in time.
The Supervision Tree Principle \cite{OTP} suggests that actors should be
organised in a tree structure so that any failed actor could be properly
restarted by its supervisor.  Nevertheless, adopting the Supervision Tree
principle is optional in Erlang.

The Akka library makes supervision obligatory by restricting the way of creating
actors.  The only way to create an actor is using the {\it actorOf} method
provided by {\bf ActorSystem} or {\bf ActorContext}.  Each actor system
provides a guardian actor for all user-created actors.  Calling the {\it
actorOf} method of an actor system creates an actor supervised by the guardian
actor.  Calling the {\it actorOf} method of an actor context creates a child
actor supervised by that actor.  Therefore, all user-created actors in an
actor system, together with the guardian actor of that actor system, form a tree
structure. Obligatory supervision unifies the the structure of actor deployment
and simplifies the work of system maintenance.

In the code below, we define a supervisor and its child.  The Supervisor
forwards all messages to the child.  The child will be restarted by its
supervisor when it raises an exception.  The supervision strategy also specifies
the maximum failures a child may have within a time range.  If a child fails
more frequently than the the allowed frequency, the supervisor will be stopped,
report its failure to its supervisor. In the sample code, the child raises an
exception every time it receives a ``Stop'' string; it prints out the string
representation of other messages. The terminal output indicates that the child
actor is restarted before the second message is delivered.

      \begin{lstlisting}[language=scala]
class Supervisor extends Actor {
  override val supervisorStrategy =
    OneForOneStrategy(maxNrOfRetries = 10, withinTimeRange = 1 minute) {
      case _: Exception                ⇒
        println("Exception Raised to: "+self)
        Restart
    }

  val child:ActorRef = context.actorOf(Props[Child], "child")

  def receive:Any => Unit = {
    case m => child ! m
  }
}

class Child extends Actor {
  def receive:Any => Unit = {
    case "Stop" => throw new Exception("Test Exception")
    case m => println("Received message: "+m)
  }
}


object SupervisorTest extends App{
  val system = ActorSystem("MySystem")
  val actorRef:ActorRef = system.actorOf(Props[Supervisor], "supervisor")

  actorRef ! "Stop"
  actorRef ! "Hello World!"
}

/*
Terminal Output:
Exception Raised to: Actor[akka://MySystem/user/supervisor]
Received message: Hello World!
*/
    \end{lstlisting}


\end{comment}
