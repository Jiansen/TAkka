\begin{comment}
\section{Mixing Static and Dynamic Type Checking}

Implementing {\bf ActorContext} described as \S\ref{actor_context} requires a
mixture of static and dynamic type checking.  This section looks into how
static and dynamic type checking are combined to guarantee backward compatible
code evolution and support type safe data retrieval.

\subsection{Backward Compatible Code Evolution}
\end{comment}
\section{Backward Compatible Code Evolution}
\label{code_evolution}

Partial service upgrade is a desired feature of distributed systems, whose
components are typically developed separately. Unfortunately, hot code swap is
not supported by the JVM, the platform where the TAkka library runs on.  To
support hot swapping on an actor's receive function, system message handler, and
supervision strategy, those three behaviour methods are maintained as object
reference.

Different from Erlang and Akka Design, code evolution in TAkka must be backward
compatible.  In another word, an actor must evolve to a version that is able to
handle the same amount of or more message patterns.  The above decision is made
so that a service published to users would not be unavailable later.

\begin{figure}[h]
\label{become}
\begin{lstlisting}
trait ActorContext[M] {
 implicit var mt:Manifest[_] = manifest[M]

 def become[SupM >: M](
      newTypedReceive: SupM => Unit,
      newSystemMessageHandler:
               SystemMessage => Unit
      newSupervisionStrategy:SupervisionStrategy
 )(implicit smt:Manifest[SupM]):ActorRef[SupM] = {
     if (!(smt >:> mt))
       throw BehaviorUpdateException(smt, mt)

    this.mt = smt
    this.systemMessageHandler =  newSystemMessageHandler
    this.timeoutHandler = newTimeoutHandler
    this.supervisionStrategy = newSupervisionStrategy

    // other code
 }
}
\end{lstlisting}
\caption{Code Evolution in TAkka}
\end{figure}

The {\it become} method is implemented as in Figure-\ref{become}.  The manifest
for static type {\bf M} should be interpreted as the least general type of
messages addressed by the actor initialized from {\bf Actor[M]}.  The
manifest for {\bf SupM} will only be known when the {\it become} method is
invoked.  When a series of {\it become} invocations are made at run time, the
order of those invocations may be non-deterministic.  Therefore, performing a
dynamic type checking is required to guarantee backward compatibility.
Nevertheless, static type checking prevents some invalid {\it become}
invocations at compile time.



