\section{Introduction}

In 1996, the Erlang/OTP (Open Telecom Platform) library\cite{ErlangWeb} was
released for writing Erlang code using the Actor model\cite{Hewitt:1973} and 5
principles derived from ten years successful experience.  Those principles are
later known as OTP Design principles. OTP Design principles, especially the
supervision tree principle, have helped Erlang programmers in building reliable
distributed applications\cite{ArmstrongErlang}.

Recently, the notion of actor and supervision tree has been ported to
statically typed languages including Scala and Haskell.  A Scala library, Akka,
makes actor supervision obligatory \cite{akka_api,akka_doc}. Nevertheless, in
current actor libraries, messages sent to actors are dynamically typed.

Continuing a line of work on merging types and the supervision principle, this
paper makes the following contributions:
\begin{itemize}
 \item it articulates the design of the TAkka library.  Section 3 illustrates
how type parameters are added to actor related classes to improve type safety
and how type-parametrized actors form supervision trees in the same manner as
traditional actors. Section \ref{alternative designs} compares the design of
TAkka with alternatives adopted by other actor libraries.
 \item it demonstrates how static and dynamic type checking could be mixed to
detect type errors at the earliest possibility.  Section 4 explains
implementation techniques used in two TAkka components where both static and
dynamic type checking are employed.
 \item it gives a critical evaluation of the TAkka library.  Section 5.1
defines the type pollution problem and lists possible solutions to that problem.
Among all solutions, employing type-parametrized actors solves the problem
most straightforwardly. The rest of Section 5 reports the overhead of TAkka in
terms of code size, efficiency, and scalability.
\end{itemize}

\mycomment{ contributions needs to be rewritten once the content is finished.}
