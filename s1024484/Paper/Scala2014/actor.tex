\section{TAkka Library Design}

% This section presents the design of the TAkka library.  We outline how
% we add types to actors and how to construct supervision trees of typed actors 
% in TAkka.  This section concludes with a brief discussion about design 
% alternatives used by other actor libraries.

\subsection{Type-parameterized Actor}
\label{actor}
A TAkka actor has type {\tt TypedActor[M]}.  It inherits the Akka {\tt
Actor} trait to minimize implementation effort.  Users of the TAkka library,
however, do not need to use any Akka Actor APIs.  Instead, we encourage
programmers to use the typed interface given in Figure \ref{takka_actor_api}. 
 Unlike other actor libraries, every TAkka actor class takes a type parameter 
{\tt M} which specifies the type of messages it expects to receive.  The same 
type parameter is used as the input type of the receive function, the type 
parameter of the actor context and the type parameter of the actor self reference.

\begin{figure}
\label{takka_actor_api}
\begin{lstlisting}[language=scala, escapechar=?]
package takka.actor
abstract class ?\textcolor{blue}{TypedActor[M:Manifest]}? extends akka.actor.Actor
  def typedReceive:?\textcolor{blue}{M}?=>Unit
  val typedSelf:?\textcolor{blue}{ActorRef[M]}?
  val typedContext:ActorContext?\textcolor{blue}{[M]}?
  var supervisorStrategy: SupervisorStrategy
\end{lstlisting}
\caption{TAkka API: TypedActor}
\end{figure}

The two immutable fields of {\tt TypedActor}: {\tt typedContext} and 
{\tt typedSelf}, will be initialized automatically when the actor is created.
Library users may override the default supervisor strategy in the
way explained in Section \ref{supervision}.  The implementation of the {\tt
typedReceive} method, on the other hand, is always provided by users.

The limitation of using inheritance to implement TAkka actors is that Akka 
features are still available to library users.  Unfortunately, this limitation 
cannot be overcome by using delegation because, as we have seen in Figure 2, 
a child actor is created by calling the {\tt actorOf} method from its 
supervisor's actor context, which is a private API of the supervisor.  {\tt 
TypedActor} is the only TAkka class that is implemented using inheritance. 
Other TAkka classes are either implemented by delegating tasks to Akka 
counterparts or rewritten in TAkka.  Re-implementing the TAkka 
Actor library requires a similar amount of work for implementing the Akka Actor 
library.


\subsection{Actor Reference}
\label{actor_ref}
A reference to an actor of type {\tt TypedActor[M]} has type {\tt
ActorRef[M]}.  The APIs for the {\tt ActorRef} class is given in Figure \ref{ActorRef}. 
An actor reference provides a {\tt !} method, through which users
can send a message to the referenced actor.  Sending an actor a message of unexpected
type will raise an error at compile time. By using type-parameterized actor
references, the receiver does not need to worry about unexpected messages, while
senders can be sure that messages will be understood and processed, as long as
the message is delivered.

An actor usually can react to a finite set of different message patterns 
whereas our notion of actor reference only takes one type parameter.  In a 
type system that supports untagged union types, no special extension is
required.  In a type system which supports polymorphism, {\tt ActorRef} should
be contravariant on its type argument {\tt M}, denoted as {\tt ActorRef[-M]} in Scala.
Consider rewriting the simple calculator defined in Figure \ref{supervisedcalculator} using  
TAkka, it is clear that {\tt ActorRef} is contravariant because {\tt 
ActorRef[Operation]} is a subtype of {\tt ActorRef[Division]} though {\tt 
Division} is a subtype of {\tt Operation}. Contravariance is crucial to avoid 
the type pollution problem described in Section \ref{type_pollution}.  

\begin{figure}
\label{ActorRef}
      \begin{lstlisting}[language=scala, escapechar=?]
abstract class ActorRef?\textcolor{blue}{[-M](implicit mt:Manifest[M])}?
  def !(message: ?\textcolor{blue}{M}?):Unit
  ?\textcolor{blue}{def publishAs[SubM<:M](implicit smt:Manifest[SubM]):ActorRef[SubM]}?
    \end{lstlisting}
    \caption{TAkka API: Actor Reference}
\end{figure}

For ease of use, {\tt ActorRef} provides a {\tt publishAs} method that 
casts an actor reference to a version that only accepts a subset of supported 
messages.  The {\tt publishAs} method encapsulates the process of type cast on 
{\tt ActorRef}, a contravariant type.  We believe that using the notation of the
{\tt publishAs} method can be more intuitive than thinking about contravariance 
and subtyping relationship when publishing an actor reference as different 
types in a complex application.  In addition, type conversion using {\tt 
publishAs} is statically type checked.  More importantly, with the 
{\tt publishAs} method, users can give a supertype of an actor reference on 
demand, without defining new types and recompiling affected classes in the type 
hierarchy.  The last advantage is important in Scala because a library 
developer may not have access to code written by others.


\begin{figure}[!h]
      \begin{lstlisting}[language=scala, escapechar=?]
class StringProcessor extends ?\textcolor{blue}{TypedActor[String]}?  {
  def ?\textcolor{blue}{typedReceive}?  = {
    case m:String => println("received message: "+m)
  }
}
object StringProcessorTest extends App {
  val system = ActorSystem("ServerTest")
  val processor = system.actorOf(Props[?\textcolor{blue}{String}?, StringProcessor], "processor")
  
  processor ! "Hello World"
//  processor ! 1
// compile error: type mismatch; found : Int(3)
//   required: String

  val processorString = system.actorFor?\textcolor{blue}{[String]}?
                     ("akka://ServerTest/user/processor")
  processorString ! "Hello World Again"
  val processorInt = system.actorFor?\textcolor{blue}{[Int]}?
                  ("akka://ServerTest/user/processor")
  processorInt ! 2
}
/*
Terminal output:
received message: Hello World
received message: Hello Worldv Again
Exception in thread "main" java.lang.Exception: 
 ActorRef[akka://ServerTest/user/server] does not 
 exist or does not have type ActorRef[Int] at 
 takka.actor.ActorSystem.actorFor(ActorSystem.scala:223)
 ...
*/
    \end{lstlisting}
    \caption{TAkka Example: A String Processor}
    \label{takkastring}
\end{figure}




Figure \ref{takkastring} defines the same string processing actor given in
Figure \ref{fig:akkastring}.  The {\tt typedReceive} method now has type 
{\tt 
String$\Rightarrow$Unit}, which is the same as intended.  In this example, the 
types of {\tt typedReceive} and {\tt m} may be omitted because they 
can be inferred by the compiler.  Unlike the Akka example, sending an integer 
to {\tt server} is rejected by the compiler.  Although the type error 
introduced on line 19 cannot be statically detected, it is captured by the 
run-time as soon as the {\tt actorFor} method is called.  In the TAkka version, 
there is no need to define a handler for unexpected messages.

\subsection{Props and Actor Context}
\label{actor_context}
The type {\tt Props} denotes the properties of an actor.   An instance  of type {\tt 
Props[M]} is used when creating an actor of type {\tt TypedActor[M]}.  Say {\tt 
myActor} is of type {\tt MyActor}, which is a subtype of {\tt TypedActor[M]}, an instance 
of type {\tt Prop[M]} can be created by one of the three ways shown in Figure 
\ref{takka_props}:

\begin{figure}[h]
\label{takka_props}
\begin{lstlisting}[language=scala, escapechar=?]
 val props:Props[M] = Props[M, MyActor]
 val props:Props[M] = Props[M](new MyActor)
 val props:Props[M] = Props[M](myActor.getClass)
\end{lstlisting}
    \caption{TAkka Example: Creating Actor Props}
\end{figure}


Contrary to an actor reference, which is the interface for receiving messages, 
an actor context describes the actor's view of the outside world.   Because 
each actor is an independent computational 
primitive, an actor context is private to the corresponding actor.  By using 
APIs in Figure \ref{ActorContext}, an actor can (i) retrieve an actor reference 
corresponding to a given actor path using the {\tt actorFor} method, (ii) 
create a child actor with a system-generated or user-specified name using one 
of the {\tt actorOf} methods, (iii) set a timeout denoting the time within which
a new message must be received using the {\tt setReceiveTimeout} method, and
(iv) update its behaviours using the {\tt become} method.  Compared with 
corresponding Akka APIs, our methods take an additional type parameter whose 
meaning will be explained below.

\begin{figure}[h]
\label{ActorContext}
      \begin{lstlisting}[language=scala, escapechar=?]
abstract class ActorContext?\textcolor{blue}{[M:Manifest]}?
  def actorOf ?\textcolor{blue}{[Msg]}? (props: Props?\textcolor{blue}{[Msg])(implicit mt: Manifest[Msg]}?): ActorRef?\textcolor{blue}{[Msg]}?
  def actorOf ?\textcolor{blue}{[Msg]}? (props: Props?\textcolor{blue}{[Msg]}?, name: String)?\textcolor{blue}{(implicit mt: Manifest[Msg])}?: ActorRef?\textcolor{blue}{[Msg]}?
  def actorFor ?\textcolor{blue}{[Msg]}? (actorPath: String)
       ?\textcolor{blue}{(implicit mt: Manifest[Msg])}?: ActorRef?\textcolor{blue}{[Msg]}?
  def setReceiveTimeout(timeout: Duration): Unit
  def become?\textcolor{blue}{[SupM >: M]}?(
      newTypedReceive: ?\textcolor{blue}{SupM}? => Unit,
      newSystemMessageHandler:
                         SystemMessage => Unit,
      newSupervisorStrategy:SupervisorStrategy
  )?\textcolor{blue}{(implicit smt:Manifest[SupM])}?:ActorRef?\textcolor{blue}{[SupM]}?
    \end{lstlisting}
    \caption{TAkka API: Actor Context}
\end{figure}

The two {\tt actorOf} methods are used to create a type-parameterized actor supervised 
by the current actor.  Each actor created is assigned to a typed actor path, 
an Akka actor path together with a {\tt Manifest} of the message type.  Each 
actor system contains a typed name server.  When an actor is created inside an 
actor system, a mapping from its typed actor path to its typed actor reference 
is registered to the typed name server.  The {\tt actorFor} method of {\tt 
ActorContext} and {\tt ActorSystem} fetches a typed actor reference from the 
typed name server.


\subsection{Backward Compatible Behaviour Upgrade}
\label{hot_swapping}

The {\tt become} method enables behaviour upgrade of an 
actor.  The {\tt become} method in TAkka is different from behaviour 
upgrades in  Akka in two aspects.  Firstly, as the handler for system messages 
are separated from the handler for other messages, TAkka users may update the 
system message handler as well.  Secondly, behaviour upgrade in TAkka {\it must} be 
backward compatible and {\it cannot} be rolled back.  In other words, an actor 
must evolve into a version that is able to handle the original
message patterns.  The above decision is made so that a service published to 
users will not be unavailable later.  


\begin{comment}I suggest to enforce backward compatible upgrades whenever
possible to provide better user experience.  If a bad design has to be deprecated,
an error message should be returned, if the received message is a synchronous 
request.  Otherwise, the user cannot tell whether the message is deprecated or lost.}
\end{comment}

\begin{figure}
\label{become}
\begin{lstlisting}
abstract class ActorContext[M:Manifest] 
  implicit private var mt:Manifest[M] = manifest[M]
  def become[SupM >: M](
      newTypedReceive: SupM => Unit,
      newSystemMessageHandler:
               SystemMessage => Unit
  )(implicit smtTag:Manifest[SupM]):ActorRef[SupM] 

case class BehaviorUpdateException(smt:Manifest[_], mt:Manifest[_]) extends Exception(smt + "must be a supertype of "+mt+".")
\end{lstlisting}
\caption{TAkka API: Hot Swapping}
\end{figure}

The {\tt become} method is declared as in Figure \ref{become}.  The
static type {\tt M} should be interpreted as the least general type of
messages addressed by the actor initialized from {\tt TypedActor[M]}.  The
type value of {\tt SupM} will only be known when the {\tt become} method 
is invoked.  When a series of {\tt become} invocations are made at run 
time, the order of those invocations may be non-deterministic.  Therefore, 
performing dynamic type checking is required to guarantee backward
compatibility.  Nevertheless, static type checking prevents some invalid {\tt 
become} invocations at compile time.

\subsection{Reusing Akka Supervisor Strategies in TAkka}
\label{supervision}

The Akka library implements two of the three supervisor strategies in OTP:
{\tt OneForOne} and {\tt AllForOne}.  If a supervisor adopts the
{\tt OneForOne} strategy, a child will be restarted when it fails.  If a 
supervisor adopts the {\tt AllForOne} supervisor strategy, all children will 
be restarted when any of them fails.  The third OTP supervisor strategy, {\tt
RestForOne}, restarts children in a user-specified order, and hence is not
supported by Akka as it does not specify an order of initialization for
children.  Simulating the {\tt RestForOne} supervisor strategy in Akka
requires ad-hoc implementation that groups related children and defines special
messages to trigger actor termination.  The {\tt RestForOne} strategy is not 
implemented in Akka.  The Akka library does not implement it neither because it 
is not required for applications that we have considered.

Figure \ref{super} gives APIs of supervisor strategies in Akka.  As in OTP, for
each supervisor strategy, users can specify the maximum number of
restarts of any child within a period.  The default 
supervisor strategy in Akka is {\tt OneForOne} that permits unlimited 
restarts.  {\tt Directive} is an enumerated type with the following values: the
{\tt Escalate} action which throws the exception to the supervisor of the 
supervisor, the {\tt Restart} action which replaces the failed child with a new 
one, the {\tt Resume} action which asks the child to process the message again, 
and the {\tt Stop} action which terminates the failed actor permanently.

None of the supervisor strategies in Figure \ref{super} requires a 
type-parameterized classes during construction.  Therefore, from the perspective 
of API design, both supervisor strategies are constructed in TAkka in the same 
way as in Akka.


\begin{figure}
\label{super}
    \begin{lstlisting}    
abstract class SupervisorStrategy
case class OneForOne(restart:Int, time:Duration)(decider: Throwable => Directive) extends SupervisorStrategy
case class OneForAll(restart:Int, time:Duration)(decider: Throwable => Directive) extends SupervisorStrategy
    \end{lstlisting}
    \caption{Supervisor Strategies}
\end{figure}


\subsection{Handling System Messages}
\label{systemmessage}
Actors communicate with each other by sending messages.  To organize actors, a 
special category of messages should be handled by all actors.  In Akka, those 
messages are subclasses of the {\tt PossiblyHarmful} trait.  The TAkka library 
retains some of them as subclasses of the {\tt SystemMessage} trait.

Actors communicate with each other by sending messages.  To maintain a
supervision tree, a special category of messages should be handled by all 
actors.  We define a type {\tt SystemMessage} to be the supertype of all 
messages for system maintenance purposes.  The three Akka system messages 
retained in TAkka are listed below.  Other Akka system messages that are not 
related to supervision are not included in TAkka.


\begin{description}
  \item[ReceiveTimeout] A message sent from an actor to itself when it has not received a message
after a timeout.
  \item[Kill] An actor that receives this message will send an {\tt ActorKilledException} to its supervisor.
  \item[PoisonPill] An actor that receives this message will be 
permanently terminated.  The supervisor cannot restart the killed actor.
\end{description}


The next question is whether a system message should be handled by the library 
or by application developers.  In Erlang and early versions of Akka, all
system messages can be explicitly handled by developers in the {\tt receive}
block.  In recent Akka versions, some system messages become private to library 
developers and some can be still handled by application developers.

As there are only two kinds of supervisor strategies to
consider, both of which have clearly defined operational behaviours, all
messages related to the liveness of actors are handled in the TAkka library. 
Application developers may indirectly affect the system message handler via 
specifying
the supervisor strategies. In contrast, messages related to the behaviour of an
actor, e.g. {\tt ReceiveTimeout}, are better handled by application
developers. In TAkka, {\tt ReceiveTimeout} is the only system message that can
be explicitly handled by users.  Nevertheless, we keep the {\tt SystemMessage}
trait in the library so that new system messages can be included in the future
when required.

A key design decision in TAkka is to separate handlers for the system messages 
and user-defined messages.  The above decision has two benefits. Firstly,
the type parameter of actor-related classes only need to denote
the type of user defined messages rather than the untagged union of user 
defined messages and the system messages.  Therefore, the TAkka design applies
to systems that do not support untagged union type.  Secondly, since 
system messages can be handled by the default handler, which applies
to most applications, users can focus on the logic of handling user
defined messages.

\begin{comment}
\subsection{Design Alternatives}
\label{alternative designs}


\paragraph{Akka Typed Actor}
In the Akka library, there is a special class called {\tt TypedActor}, which
contains an internal actor and can be supervised.  A service of {\tt 
TypedActor} is invoked by method invocation instead of message exchanging.  
Code in Figure \ref{akka typed actor} demonstrates how to define a simple 
string processor using Akka typed actor.  The Akka {\tt TypedActor} prevent 
some type errors but have two limitations. Firstly, {\tt TypedActor} does not 
permit hot swapping on its behaviours.  Secondly, avoiding type pollution by 
using Akka typed actors is as awkward as using a plain object-oriented 
model, where supertypes need to be introduced.  In Scala and Java, introducing 
a supertype in a type hierarchy requires modification to all affected classes. 
 



\begin{figure}
\label{akka typed actor}
\begin{lstlisting}
trait MyTypedActor{
  def processString(m:String) 
}
class MyTypedActorImpl(val name:String) extends MyTypedActor{
  def this() = this("default")
  
  def processString(m:String) {
    println("received message: "+m) 
  }
}
object FirstTypedActorTest extends App {
  val system = ActorSystem("MySystem")  
  val myTypedActor:MyTypedActor =
    TypedActor(system).typedActorOf(
    TypedProps[MyTypedActorImpl]())
  myTypedActor.processString("Hello World")
}

/*
Terminal output:
received message: Hello World
 */	
\end{lstlisting}
\caption{Akka TypedActor Example}
\end{figure}

\paragraph{Actors with or without Mutable States}
The actor model formalized by Hewitt et al. \citep{Hewitt:1973} does not 
specify its implementation strategy.  In Erlang, a functional programming 
language, actors do not have mutable states.  In Scala, an
object-oriented programming language, actors may have mutable states.
The TAkka library is built on top of Akka and implemented in Scala.  
As a result, TAkka does not prevent users from defining actors with 
mutable states.
Nevertheless, the authors of this paper encourage the use of 
actors in a functional style, for example encoding the {\tt sender} of a 
synchronous message as part of the incoming message rather than a state 
of an actor, because it is difficult to synchronize mutable
states of replicated actors in a cluster environment.

In a cluster, resources are replicated at different locations to provide 
fault-tolerant services.  The CAP theorem \citep{CAP} shows that it is
impossible to achieve consistency, availability, and partition tolerance in a
distributed system simultaneously.  For actors that use mutable state, system 
providers must either sacrifice availability or partition tolerance, or modify 
the consistency model.  For example, Akka actors have mutable state and Akka 
cluster developers spend great effort to implement an eventual consistency 
model \citep{Kuhn12}. In contrast, stateless services, e.g. RESTful web 
services, are more likely to achieve a good scalability and availability.



\paragraph{Bi-linked Actors}
In addition to one-way linking in the supervision tree, Erlang and Akka
provide a mechanism to define a two-way linkage between actors. 
Bi-linked actors are each aware of the liveness of the other.  We believe that
bi-linked actors are redundant in a system where supervision is
obligatory.  Notice that, if the computation of an actor relies on the liveness
of another actor, those two actors should be organized in the same 
supervision tree.

\end{comment}