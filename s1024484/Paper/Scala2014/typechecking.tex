\section{Mixing Static and Dynamic Type Checking}
\label{type_checking}
A key advantages of static typing is that it detects some type errors at an 
early stage, i.e., at compile time.  The TAkka library is designed to detect 
type errors as early as possible.  Nevertheless, not all type errors can be 
statically detected, and some dynamic type checks are required. To address this 
issue, a notion of run-time type descriptor is required.  This section 
summarizes the type reflection mechanism in Scala and  explains
how it benefits the implementation of our typed name
server.  Our typed name server can be straightforwardly ported
to other platforms that support type reflection.

\subsection{Scala Type Descriptors}

As in Java, generic types are erased at runtime in Scala.
To record type information that is required at runtime, users can ask Scala to 
keep the type information by using the {\tt Manifest} class.  
A {\tt Manifest[T]} encapsulates the runtime type representation 
of some type {\tt T}.  It provides methods for subtype test ($<:<$).

The code example in Figure \ref{scala_manifest_example} shows common usages of 
Manifest.  Different to the Java class value obtained at line 10, the Scala manifest
value obtained at line 8 records the description of generic types.  
There are three ways of obtaining a manifest: using the Methods {\tt 
manifest} (line 8), using an implicit parameter of type {\tt Manifest[T]} ]
(line 13), or using a {\tt context bound} of a type parameter (line 18).  
The {\tt manifest} method is defined in the Scala 
{\tt Predef} object.  Values for implicit parameters are provided by 
the run-time system if qualified values can be inferred.  For example, at line 16,
Scala infers that a value for the implicit parameter {\tt m} is {\tt mainifest[Int]}.  
Details of using implicit parameter can be found in the Scala language 
specification \citep{scala_specification}.  Context bound is a syntactic sugar for implicit parameters without a 
user-specified parameter name.  
The {\tt isSubType} method defined at line 23
tests if the first {\tt Manifest} represents a type that is a subtype of the 
type represented by the second {\tt Manifest}.

\begin{figure}[h]
\begin{lstlisting}[language=scala]
import scala.reflect._

object ManifestExample extends App {
  assert(List(1,2.0,"3").isInstanceOf[List[String]])
  // Compiler Warning :non-variable type argument String in type 
  // List[String] is unchecked since it is eliminated by erasure
  
  assert(manifest[List[Int]].toString.equals(
            "scala.collection.immutable.List[Int]"))
  assert(classOf[List[Int]].erasure.toString.equals(
            "class scala.collection.immutable.List"))
  
  def typeName[T](x: T)(implicit m: Manifest[T]): String  = {
    m.toString
  }
  assert(typeName(2).equals("Int"))
    
  def boundTypeName[T:Manifest](x: T):String = {
    manifest[T].toString
  }
  assert(boundTypeName(2).equals("Int"))
  
  def isSubType[T: Manifest, U: Manifest] = manifest[T] <:< manifest[U]
  assert(isSubType[List[String], List[AnyRef]])
  assert(! isSubType[List[String], List[Int]]) 
}
\end{lstlisting}
\caption{Scala Example: Manifest Example}
\label{scala_manifest_example}
\end{figure}







\subsection{Typed Name Server}
\label{nameserver}

In distributed systems, a name server maps each registered name, usually a
unique string, to a dynamically typed value, and provides a function to look up 
a value for a
given name. A name can be encoded as a {\tt Symbol} in Scala so that names
which represent the same string have the same value.  As a value retrieved from 
a name server is {\it dynamically typed}, it needs to be checked against and be 
cast to the expected type at the client side before using it.
To overcome the limitations of the untyped name server, we design and implement
a typed name server which maps each registered typed name to a value of the
corresponding type, and allows look-up of a value by giving a typed name.

\begin{figure}[!h]
\label{tsymbol}
\begin{lstlisting}
case class TSymbol[T:Manifest](val s:Symbol) {
    private [takka] val t:Manifest[_] = manifest[T]
    override def hashCode():Int = s.hashCode()  
    override def equals(that: Any) :Boolean = {
      case ts: TSymbol[_] => ts.t.equals(this.t) && ts.s.equals(this.s)
      case _ => false
    }
}
case class TValue[T:Manifest](val value:T){
  private [takka] val t:Manifest[_] = manifest[T]
}
\end{lstlisting}
\caption{TSymbol and TValue}
\end{figure}

A typed name, {\tt TSymbol}, is a name shipped with a type descriptor.  A 
typed value, {\tt TValue}, is a value shipped with a type descriptor, which
describes a super type of the most precise type of that value.  
In Scala, {\tt TSymbol} and {\tt TValue} can be simply defined as in Figure
\ref{tsymbol}.  {\tt TSymbol} is declared as a {\it case class} in Scala so 
that it can be used as a data constructor and for pattern matching.  In 
addition, the type descriptor, {\tt t}, is constructed automatically and is 
private to the {\tt takka} package so that it can only be accessed by TAkka 
library developers. {\tt TValue} is implemented similarly for the same reason.

With the help of {\tt TSymbol}, {\tt TValue}, and a hashmap, a typed name 
server provides the following three operations:




\begin{itemize}
  \item {\tt set[T:Manifest](name:TSymbol[T], value:T):Boolean}

The {\tt set} operation registers a typed name with a value of corresponding 
type and returns true if the symbolic representation of {\it name} has {\it 
not} been registered; otherwise the typed name server discards the request and
returns false.

  \item {\tt unset[T](name:TSymbol[T]):Boolean}

The {\tt unset} operation removes the entry {\it name} and returns true if (i) 
its symbolic representation is registered and (ii) the type {\tt T} is a 
supertype of the registered type; otherwise the operation returns false.

  \item {\tt get[T] (name: TSymbol[T]): Option[T]}

The {\tt get} operation returns Some(v:{\tt T}), where {\tt v} is the value 
associated with {\it name}, if (i) {\it name} is a registered name and (ii) 
{\tt T} is a supertype of the registered type; otherwise the operation returns 
None.

\end{itemize}

Notice that {\tt unset} and {\tt get} operations succeed as long as the 
associated type of the input name is the supertype of the associated type of 
the registered name.  To permit polymorphism, the {\tt hashcode} method of {\tt 
TSymbol} defined in  Figure {\ref{tsymbol}} does not take type values into 
account.  Equivalence comparison on {\tt TSymbol} instances, however, should 
consider the type.  Although the {\tt TValue} class does not appear in the
APIs for library users, it is required for an efficient library implementation 
because the type 
information in {\tt TSymbol} is ignored in the hashmap.  Overriding the
hash function of {\tt TSymbol} also prevents the situation where users accidentally
register two typed names with the same symbol but different types,
in which case if one type is a supertype of the other, the return value of
{\tt get} may be non-deterministic.  Last but not least, when an operation 
fails, the name server returns {\tt false} or {\tt None} rather than raising an
exception so that the name server is always available.

In general, dynamic type checking can be carried out in two ways.  The first 
method is to check whether the most precise type of a value conforms to the
structure of a data type.  Examples of this method include dynamically typed
languages and the {\tt instanceof} method in Java and other languages.  The
second method is to compare two type descriptors at run time.  The
implementation of our typed name server employs the second method because  
it detects type errors which may otherwise be left out.  Our
implementation requires the runtime type reification feature provided by Scala.
In a system that does not support type reification, implementing typed name 
server can be difficult.


