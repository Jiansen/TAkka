\section{Introduction}

The Actor model defined by \citet{Hewitt:1973} treats actors as
primitive computational components.  Actors collaborate by sending asynchronous
messages to each other.  An actor independently determines its reaction to
messages it receives.  The Actor model was adopted in the Erlang programming 
language \citep{ArmstrongErlang} for building real time telephony applications. 
Erlang developers later propose thed supervision principle \citep{OTP}, which 
suggests that actors should be organized in a tree structure so that any failed 
actor can be properly restarted by its supervisor.  Nevertheless, adopting the 
supervision principle is optional in Erlang.

The notions of actors and supervision trees have been ported to statically 
typed languages including Scala and Haskell.  Scala actor libraries including 
Scala Actors \citep{actor_1, actor_2} and Akka \citep{akka_api,akka_doc} use 
dynamically typed messages even though Scala is a statically typed language.  
Some recent actor libraries, including Cloud Haskell \citep{CloudHaskell}, Lift 
\citep{lift_scala}, and scalaz \citep{scalaz}, support both 
dynamically and statically typed messages, but do not support supervision.
Can actors in supervision trees be statically typed?

The key claim in this paper is that actors in supervision trees can be
typed by parameterizing the actor class with the type of messages it expects to
receive.  Type-parameterized actors benefit both users and developers
of actor-based services. For users, sending ill-typed messages is prevented at 
compile time.  Because messages are usually transmitted asynchronously, it may 
be otherwise difficult to trace the source of errors at runtime, especially in 
distributed environments.  For service developers, since unexpected messages 
are eliminated from the system, they can focus on the logic of the services 
rather than worrying about incoming messages of unexpected types.  


Implementing  type-parameterized actors in a statically-typed language; however, requires 
solving the following challenges identified by the Akka Development Team:

\begin{enumerate}
  \item The Akka developers claim that using a type parameterized actor" is impossible because actor behaviour is dynamic" \citep{Kuhn11}.

  \item The Akka developers claim that ``we need to be able to encode disjoint types" \citep{Kuhn12}, an option which is not provided by Scala. 

  \item The Akka developers claim that ``address lookup will be impossible" \citep{Kuhn12} because actor references will have different types.
  
  \item The Akka developers claim that� ``the way to provide high performance is to provide fewer guarantees, and also let the programmer choose what guarantees he or she needs'' \citep{Kuhn12}.
\end{enumerate}






Continuing a line of work on merging types with actor programming by Haller
and Odersky \citep{actor_1, actor_2} and the Akka developers \citep{akka_doc}, 
along with work on system reliability testing by 
\citet{ChaosMonkey} and \citet{ErlangChaosMonkey}, this paper makes 
following contributions.

\begin{itemize}
 \item It presents the design and implementation of a typed name server
that maps typed names to values of the corresponding type, for example actor 
references.  The typed name server (Section \ref{nameserver}) mixes 
static and dynamic type checking so that type  errors are detected at the earliest 
opportunity.  The implementation requires language support for type reflection 
and first class type descriptors.  In Scala, the {\tt Manifest} class provides 
such facility.

 \item It describes the design of the TAkka library.  Sections \ref{actor} to
\ref{actor_context} illustrate how type parameters are added to actor related
classes to improve type safety.  The TAkka actor supports dynamic backward-compatible 
behaviour upgrading as shown in Section \ref{hot_swapping}.  By separating the handler for system 
messages from the handler for user defined messages, Sections \ref{supervision}
and \ref{systemmessage} show that type-parameterized actors can form
supervision trees in the same manner as untyped actors.  Section 
\ref{evolution} explains how to upgrade Akka programs to their TAkka 
equivalents gradually.
 
 \item It gives a comprehensive evaluation of the TAkka library.  Section 
\ref{type_pollution} shows that the TAkka library can avoid the type pollution 
problem straightforwardly.  Results in
Section \ref{expressiveness} confirm that using type parameterized actors 
sacrifice neither expressiveness nor correctness.  Efficiency and 
scalability test in Section \ref{efficiency} shows that TAkka applications have 
little overhead at the initialization stage but have almost identical run-time 
performance and scalability compared to their Akka equivalents.  In 
addition, two helper libraries explained in Section \ref{reliability} are 
shipped with the TAkka library for assisting reliability evaluation.  The 
first library, Chaos Monkey, is ported to test applications against intensive 
adverse conditions.  The second library, Supervision View, is 
designed and implemented to dynamically monitor changes of supervision tree.
 
\end{itemize}
