\section{Actors and there Supervision in Akka}
\label{background}

The Akka library \citep{akka_doc} implements Actor programming in 
static typed language, Scala, and makes supervision obligatory.  Figure 
\ref{fig:akkastring} gives an example actor defined in Akka.  The {\tt receive} 
function of the Akka actor has type {\tt Any$\Rightarrow$Unit} but the 
defined actor, {\tt StringProcessor}, is only intended to process strings.  At 
Line 14, a {\tt Props}, an abstraction of actor creation, is initialized and 
passed to an actor system, which creates an actor with name  {\it processor} and 
returns a reference pointing to that actor.  Another way to obtain 
an actor is using the {\tt actorFor} method as shown in line 20.  We then use 
actor references to send string messages and integer messages to the 
referenced actor.  String messages are processed in the way defined by the 
receive function.

\begin{figure}[!h]
      \begin{lstlisting}[language=scala]
class StringProcessor extends Actor {
  def receive = {
    case m:String => println("received message: "+m);
  }
}
class MessageHandler extends Actor {
  def receive = {
    case akka.actor.UnhandledMessage(message, sender, recipient) =>
      println("unhandled message:"+message);
  }
}
object StringCounterTest extends App {
  val system = ActorSystem("StringProcessorTest")
  val processor = system.actorOf(Props[StringProcessor], "processor")
  
  val handler = system.actorOf(Props[MessageHandler]))
  system.eventStream.subscribe(handler,classOf[akka.actor.UnhandledMessage]);
  processor ! "Hello World"
  processor ! 1
  val processorRef = system.actorFor("akka://StringCounterTest/user/processor")
  processorRef ! "Hello World Again"
  processorRef ! 2
}
/*
Terminal output:
received message:	Hello World
received message:	Hello World Again
unhandled message:1
unhandled message:2
*/
    \end{lstlisting}
    \caption{Akka Example: A String Processor}
 \label{fig:akkastring}    
\end{figure}

Undefined messages, such as the two integer messages in the String Processor 
example, are treated differently in different actor libraries.  In
Erlang, an actor keeps undefined messages in its mailbox.  It attempts to 
process the message again when a new message handler is in use.  In versions 
prior to
2.0, an Akka actor raises an exception when it processes an undefined message.
In recent Akka versions, an undefined message is discarded by the actor and an
{\tt UnhandledMessage} event is pushed to the event stream of the actor system.
The event stream may be subscribed by other actors who are interested in
particular event messages.  To handle the unexpected integer message in the
above short example, an event handler is defined and created with 6 lines of 
code.

The Akka library makes supervision obligatory by restricting the way of 
creating actors. Actors can only be initialized by using the {\tt actorOf}
method provided by {\tt ActorSystem} or {\tt ActorContext}.  Each actor system
provides a guardian actor for all user-created actors.  Calling the {\tt
actorOf} method of an actor system creates an actor supervised by the 
guardian actor.  Calling the {\tt actorOf} method of an actor context 
creates a child actor supervised by that actor.  Therefore, all user-created  
actors in an actor system, together with the guardian actor of that actor 
system, form a tree structure.  Obligatory supervision unifies the 
structure of actor deployment and simplifies the work of system maintenance.

\begin{comment}
Each actor in Akka is associated with an actor path.  The string representation
of the actor path of a guardian actor has format 
{\it akka://mysystem@IP:port/user}, where {\it mysystem} is the name of the 
actor system, {\it IP} and {\it port} are the IP address and the 
port number which the actor system listens to, and {\it user} is the name of 
the guardian actor.  The actor path of a child actor is actor path of its 
supervisor appended by the name of the child actor, either a user specified 
name or a system generated name. 
\end{comment}

Figure \ref{supervisedcalculator} defines a simple calculator which supports
multiplication and division. The simple calculator does not consider the
problematic case of dividing a number by 0, where an {\tt
ArithmeticException} will be raised.  We then define a safe calculator as the 
supervisor of the simple calculator.  The safe calculator delegates 
calculation tasks to the simple calculator and restarts the simple calculator 
when an {\tt ArithmeticException} is raised.  The supervisor strategy of
the safe calculator also specifies that its child may not fail more than twice 
within a minute.  If the child fails more frequently, the safe calculator 
itself will stop, and the failure will be reported to its supervisor, the 
system guardian actor in this example.  The terminal output shows that the 
simple calculator is restarted before the third message and the fifth message 
are delivered.  The last message is not processed because both calculators 
have been terminated since the simple calculator fails more frequently than 
allowed.

\begin{figure}[!p]

  \begin{lstlisting}[language=scala]
trait Operation
case class Multiplication(m:Int, n:Int) extends Operation
case class Division(m:Int, n:Int) extends Operation
class Calculator extends Actor {
  def receive = {
    case Multiplication(m:Int, n:Int) =>
      println(m +" * "+ n +" = "+ (m*n))
    case Division(m:Int, n:Int) =>
      println(m +" / "+ n +" = "+ (m/n))
  }
}
class SafeCalculator extends Actor {
  override val supervisorStrategy =
    OneForOneStrategy(maxNrOfRetries = 2, withinTimeRange = 1 minute) {
      case _: ArithmeticException  =>
        println("ArithmeticException Raised to: "+self)
        Restart
    }
  val child:ActorRef = context.actorOf(Props[Calculator], "child")
  def receive = {
    case m => child ! m
  }
}
object SupervisedCalculator extends App {
  val system = ActorSystem("MySystem")
  val actorRef:ActorRef = system.actorOf(Props[SafeCalculator],
"safecalculator")

  calculator ! Multiplication(3, 1)
  calculator ! Division(10, 0)
  calculator ! Division(10, 5)
  calculator ! Division(10, 0)
  calculator ! Multiplication(3, 2)
  calculator ! Division(10, 0)
  calculator ! Multiplication(3, 3)
}
/*
Terminal Output:
3 * 1 = 3
java.lang.ArithmeticException: / by zero
ArithmeticException Raised to: Actor[akka://MySystem/user/safecalculator]

10 / 5 = 2
java.lang.ArithmeticException: / by zero
ArithmeticException Raised to: Actor[akka://MySystem/user/safecalculator]
java.lang.ArithmeticException: / by zero

3 * 2 = 6
ArithmeticException Raised to: Actor[akka://MySystem/user/safecalculator]
java.lang.ArithmeticException: / by zero
*/
    \end{lstlisting}
  \caption{Akka Example: Supervised Calculator}
  \label{supervisedcalculator}
\end{figure}