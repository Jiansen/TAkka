\section{Mixing Static and Dynamic Type Checking}
\label{type_checking}
A key advantages of static typing is that it detects some type errors at an 
early stage, i.e., at compile time.  The TAkka library is designed to detect 
type errors as early as possible.  Nevertheless, not all type errors can be 
statically detected, and some dynamic type checks are required. To address this 
issue, a notion of run-time type descriptor is required.  This section 
summarizes the type reflection mechanism in Scala and  explains
how it benefits the implementation of our typed name
server.  Our typed name server can be straightforwardly ported
to other platforms that support type reflection.

\subsection{Scala Type Descriptors}

As in Java, generic types are erased by the Scala compiler.
To record type information that is required at runtime but might be
erased, Scala users can ask the compiler to keep the type information
by using the {\tt Manifest} class. The Scala standard library contains four
manifest classes as shown in Figure \ref{scala_api_manifest}.
A {\tt Manifest[T]} encapsulates the runtime type representation 
of some type {\tt T}.   {\tt Manifest[T]} a subtype of {\tt ClassManifest[T]}, 
which declares methods for subtype ($<:<$) test and supertest ($>:>$).  
The object {\tt NoManifest} represents type information that is required
by a paraterized type but is not available in scope.  {\tt OptManifest[+T]} is 
the supertype of {\tt ClassManifest[T]} and {\tt OptManifest}.

\begin{figure}[!h]
\begin{lstlisting}[language=scala, escapechar=?]
package scala.reflect
trait OptManifest[+T] extends Serializable
object NoManifest extends OptManifest[Nothing] with Serializable
trait Manifest[T] extends ClassManifest[T] with Serializable
trait ClassManifest[T] extends OptManifest[T] with Serializable
  def <:<(that: ClassManifest[_]): Boolean
  def >:>(that: ClassManifest[_]): Boolean
  erasure: Class[_]
\end{lstlisting}
\caption{Scala API: Manifest Type Hierarchy}
\label{scala_api_manifest}
\end{figure}

\begin{comment}
\subsection{Scala Type Descriptors}
Scala 2.8 defines a {\tt Manifest} class whose instance is a serializable 
first class type descriptor used at runtime.  With the help of 
the {\tt Manifest} class, users can record the type information, including 
generic types, which may be erased by the Java compiler.  

In the Scala interactive session below, we obtain a Manifest value at Line 5 and 
test a subtype relationship at Line 8.  To define a method that obtains type 
information of a generic type, Scala requires a type tag as an implicit 
argument to the method.  To simplify the API, Scala further provides a form of
syntactic sugar called context bounds.  We define a method using context
bounds at Line 11, which is compiled to the version using implicit 
arguments as shown at Line 12.


\begin{lstlisting}
scala> class Sup; class Sub extends Sup
defined class Sup
defined class Sub

scala> manifest[Sub]
res0: Manifest[Sub] = Sub

scala> manifest[Sub] <:< manifest[Sup]
res1: Boolean = true

scala> def getType[T:Manifest] = {manifest[T]}
getType: [T](implicit evidence$1: Manifest[T])Manifest[T]

scala> getType[Sub => Sup => Int]
res2: Manifest[Sub => (Sup => Int)] = scala.Function1[Sub, scala.Function1[Sup, Int]]

\end{lstlisting}

 
\end{comment}

\begin{figure}[!h]
\label{tsymbol}
\begin{lstlisting}
case class TSymbol[T:Manifest](val s:Symbol) {
    private [takka] val t:Manifest[_] = manifest[T]
    override def hashCode():Int = s.hashCode()  
}
case class TValue[T:Manifest](val value:T){
  private [takka] val t:Manifest[_] = manifest[T]
}
\end{lstlisting}
\caption{TSymbol and TValue}
\end{figure}

\subsection{Typed Name Server}
\label{nameserver}

In distributed systems, a name server maps each registered name, usually a
unique string, to a dynamically typed value, and provides a function to look up 
a value for a
given name. A name can be encoded as a {\tt Symbol} in Scala so that names
which represent the same string have the same value.  As a value retrieved from 
a name server is {\it dynamically typed}, it needs to be checked against and be 
cast to the expected type at the client side before using it.
To overcome the limitations of the untyped name server, we design and implement
a typed name server which maps each registered typed name to a value of the
corresponding type, and allows to look up a value by giving a typed name.

A typed name, {\tt TSymbol}, is a name shipped with a type descriptor.  A 
typed value, {\tt TValue}, is a value shipped with a type descriptor, which
describes a super type of the most precise type of that value.  
In Scala, {\tt TSymbol} and {\tt TValue} can be simply defined as in Figure
\ref{tsymbol}.  {\tt TSymbol} is declared as a {\it case class} in Scala so 
that it can be used as a data constructor and for pattern matching.  In 
addition, the type descriptor, {\tt t}, is constructed automatically and is 
private to the {\tt takka} package so that only it can only be accessed by TAkka 
library developers. {\tt TValue} is implemented similarly for the same reason.

With the help of {\tt TSymbol}, {\tt TValue}, and a hashmap, a typed name 
server provides the following three operations:




\begin{itemize}
  \item {\tt set[T:Manifest](name:TSymbol[T], value:T):Boolean}

The {\tt set} operation registers a typed name with a value of corresponding 
type and returns true if the symbolic representation of {\it name} has {\it 
not} been registered; otherwise the typed name server discards the request and
returns false.

  \item {\tt unset[T](name:TSymbol[T]):Boolean}

The {\tt unset} operation removes the entry {\it name} and returns true if (i) 
its symbolic representation is registered and (ii) the type {\tt T} is a 
supertype of the registered type; otherwise the operation returns false.

  \item {\tt get[T] (name: TSymbol[T]): Option[T]}

The {\tt get} operation returns Some(v:{\tt T}), where {\tt v} is the value 
associated with {\it name}, if (i) {\it name} is a registered name and (ii) 
{\tt T} is a supertype of the registered type; otherwise the operation returns 
None.

\end{itemize}

Notice that {\tt unset} and {\tt get} operations succeed as long as the 
associated type of the input name is the supertype of the associated type of 
the registered name.  To permit polymorphism, the {\tt hashcode} method of {\tt 
TSymbol} defined in  Figure {\ref{tsymbol}} does not take type values into 
account.  Equivalence comparison on {\tt TSymbol} instances, however, should 
consider the type.  Although the {\tt TValue} class does not appear in the
APIs for library users, it is required for an efficient library implementation 
because the type 
information in {\tt TSymbol} is ignored in the hashmap.  Overriding the
hash function of {\tt TSymbol} also prevents the case where users accidentally
register two typed names with the same symbol but different types,
in which case if one type is a supertype of the other, the return value of
{\tt get} may be non-deterministic.  Last but not least, when an operation 
fails, the name server returns {\tt false} or {\tt None} rather than raising an
exception so that the name server is always available.

In general, dynamic type checking can be carried out in two ways.  The first 
method is to check whether the most precise type of a value conforms to the
structure of a data type.  Examples of this method include dynamically typed
languages and the {\tt instanceof} method in Java and other languages.  The
second method is to compare two type descriptors at run time.  The
implementation of our typed name server employs the second method because  
it detects type errors which may otherwise be left out.  Our
implementation requires the runtime type reification feature provided by Scala.
In a system that does not support type reification, implementing typed name 
server can be difficult.


