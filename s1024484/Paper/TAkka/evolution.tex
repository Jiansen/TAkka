\section{Evolution, Not Revolution }
\label{evolution}

Akka systems can be smoothly migrated to TAkka systems. In other words, 
existing systems can evolve to introduce more types, rather than requiring a 
revolution where all actors and interactions must be typed.

The above property is analogous to adding generics to Java programs.  Java 
generics are carefully designed so that programs without generic types can be 
partially replaced by an equivalent generic version (evolution), rather than 
requiring generic types everywhere (revolution) \cite{JGC}.

In previous sections, we have seen how to use Akka actors in an Akka 
system (Figure \ref{akkastring}) and how to use TAkka actors in a TAkka 
system (Figure \ref{takkastring}).  In the following, we will explain how to 
use TAkka actors in an Akka system and how to use an Akka actor in a TAkka 
system.


\subsection{TAkka actor in Akka system}

It is often the case that an actor-based library is implemented by one 
organization but used in a client application implemented by another 
organization.  If a developer decides to upgrade the library implementation 
using TAkka actors, for example, by upgrading the Socko Web Server 
\cite{SOCKO}, the Gatling \cite{Gatling} stress testing tool, or the core 
library of the Play framework \cite{play_doc} as we have done in Section 
\ref{expressiveness}, how the upgrade will affect client code, especially 
legacy applications built using the Akka library?  TAkka actors and actor 
references are implemented using inheritance and delegation respectively so 
that no changes are required for legacy applications.

TAkka actors inherits Akka actors.  In Figure \ref{takkaINakka}, 
the actor implementation is upgraded to the TAkka version as in Figure 
\ref{takkastring}.  The client code, line 15 through line 25, is the same as the 
old Akka version given in Figure \ref{akkastring}.  That is, no changes are 
required for the client application.

TAkka actor reference delegates the task of message sending to an 
Akka actor reference, its {\tt untypedRef} field.  In line 31 in Figure 
\ref{akkastring}, we get an untyped actor reference from {\tt typedserver} and 
use the untyped actor reference in code where an Akka actor reference is 
expected.  Because an untyped actor reference accepts messages of any type, 
messages of unexpected type may be sent to TAkka actors if an Akka actor 
reference is used.  As a result, users who are interested in the {\tt 
UnhandledMessage} event may subscribe to the event stream as in line 33.


\begin{figure}
\label{takkaINakka}
      \begin{lstlisting}[language=scala, escapechar=?]
class TAkkaServerActor extends ?\bf{takka.actor.TypedActor[String]}? {
  def ?\bf{typedReceive}? = {
    case m:String => println("received message: "+m)
  }
}

class MessageHandler(system: akka.actor.ActorSystem) extends akka.actor.Actor {
  def receive = {
    case akka.actor.UnhandledMessage(message, sender, recipient) =>
      println("unhandled message:"+message);
  }
}

object TAkkaInAkka extends App {
  val akkasystem = akka.actor.ActorSystem("AkkaSystem")
  val akkaserver = akkasystem.actorOf(
    akka.actor.Props[TAkkaServerActor], "server")

  val handler = akkasystem.actorOf(
    akka.actor.Props(new MessageHandler(akkasystem)))
  
  akkasystem.eventStream.subscribe(handler,
     classOf[akka.actor.UnhandledMessage]);
  akkaserver ! "Hello Akka"
  akkaserver ! 3
  
  val takkasystem = ?\bf{takka}?.actor.ActorSystem("TAkkaSystem")
  val typedserver = takkasystem.actorOf(
     takka.actor.Props[?\bf{String,}? ServerActor], "server")
  
  val untypedserver = takkaserver.untypedRef
  
  takkasystem.system.eventStream.subscribe(
    handler,classOf[akka.actor.UnhandledMessage]);
  
  untypedserver ! "Hello TAkka"
  untypedserver ! 4
}

/*
Terminal output:
received message: Hello Akka
unhandled message:3
received message: Hello TAkka
unhandled message:4
*/
    \end{lstlisting}
    \caption{TAkka actor in Akka application}
\end{figure}



\subsection{Akka Actor in TAkka system}

Sometimes, developers want to update the client code or the API before upgrading 
the actor implementation. For example, a developer may not have access to 
the actor code; or the library may be large, so the developer may want to 
upgrade the library gradually.

Users can initialize a TAkka actor reference by providing an Akka actor 
reference and a type parameter.  In Figure \ref{akkaINtakka}, we re-use the 
Akka actor, initialize the actor in an Akka actor system, and obtain an Akka 
actor reference as in Figure \ref{akkastring}.  Then, we initialize a TAkka 
actor reference, {\tt takkaServer}, which only accepts {\tt String} messages.


\begin{figure}
\label{akkaINtakka}
      \begin{lstlisting}[language=scala, escapechar=?]
class AkkaServerActor extends akka.actor.Actor {
  def receive = {
    case m:String => println("received message: "+m)
  }
}

object AkkaInTAkka extends App {
  val system = akka.actor.ActorSystem("AkkaSystem")
  val akkaserver = system.actorOf(
       akka.actor.Props[AkkaServerActor], "server")
  
  val takkaServer = new takka.actor.ActorRef?\bf{[String]}?{
    val untypedRef = akkaserver
  }
  
  takkaServer ! "Hello World"
//  takkaServer ! 3
}

/*
Terminal output:
received message: Hello World
*/
    \end{lstlisting}
    \caption{Akka actor in TAkka application}
\end{figure}


