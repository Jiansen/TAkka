\section{Actor Programming}
\label{background}

\subsection{Actor Model and OTP Design Principles}

The Actor model defined by Hewitt et al.  \cite{Hewitt:1973} treats actors as
primitive computational components.  Actors collaborate by sending asynchronous
messages to each other.  An actor independently determines its reaction to
messages it receives.

The Actor model is adopted by the Erlang programming language, whose 
developers later summarized 5 OTP design principles to improve the 
reliability of Erlang applications \cite{OTP}.  We notice that the Behaviour 
principle, the Application principle, and the Release principle coincide with 
3 Java and Scala Programming practices listed in Table \ref{otp}.  The 
Release Handling principle requires runtime support for hot swapping.  In a 
platform where hot swapping is not supported in general, e.g. Java Virtual 
Machine (JVM), hot swapping a particular component can be simulated by updating 
the reference to that component. For example, Section \ref{hot_swapping} 
explains how to hot swap the behaviour of an actor in TAkka, which runs on the 
JVM. The Supervision principle, which is the central topic of this paper, 
has no direct correspondence in native JVM based systems.

\begin{table}[t]
\label{otp}
% \tbl{Implementing OTP Design Principles in an OO Language}{
 \begin{center}
\begin{tabular}{| l | p{4.5 cm} | }
\hline
  OTP Design Principle & JAVA/Scala Programming \\
\hline
  Behaviour & defining an abstract class \\
\hline
  Application  & defining an abstract class that has two abstract methods: start and stop \\
\hline
  Release  & packaging related application classes  \\ 
\hline
  Release Handling  & hot swapping support on key modules is required \\
\hline
  Supervision  & no direct correspondence  \\
\hline
\end{tabular}
 \caption[]{Using OTP Design Principles in JAVA/Scala\\  Programming}
\end{center}
%}
\end{table}




\subsection{Akka Actor}
\label{akka actor}


An Akka Actor has four essential components as shown in Figure 
\ref{fig:akka_actor_api}: (i) a receive function that defines its reaction to 
incoming messages, (ii) an actor reference pointing to  itself, (iii) the actor 
 context representing the outside world of the actor, and (iv) the supervisor 
strategy for its children.

\begin{figure}[h]
\label{fig:akka_actor_api}
\begin{lstlisting}[language=scala]
package akka.actor
trait Actor{
  def receive:Any=>Unit
  val self:ActorRef
  val context:ActorContext
  var supervisorStrategy: SupervisorStrategy
}
\end{lstlisting}
\caption{Akka Actor API}
\end{figure}

Figure \ref{akkastring} shows an example actor in Akka.  The {\tt receive} 
function of the Akka actor has type {\tt Any$\Rightarrow$Unit} but the 
defined actor, {\tt ServerActor}, is only intended to process strings.  At Line 
16, a {\tt Props}, an abstraction of actor creation, is initialized and passed 
to an actor system, which creates an actor with name \textcolor{mauve}{\tt 
server} and returns a reference pointing to that actor.  Another way to obtain 
an actor is using the {\tt actorFor} method as shown in line 24.  We then use 
actor references to send the actor string messages and integer messages.  
String messages are processed in the way defined by the receive function.

\begin{figure}[t]
 \label{akkastring}
      \begin{lstlisting}[language=scala]
class ServerActor extends Actor {
  def receive = {
    case m:String => println("received message: "+m)
  }
}

class MessageHandler(system: ActorSystem) extends Actor {
  def receive = {
    case akka.actor.UnhandledMessage(message, sender, recipient) =>
      println("unhandled message:"+message);
  }
}

object ServerTest extends App {
  val system = ActorSystem("ServerTest")
  val server = system.actorOf(Props[ServerActor], "server")
  
  val handler = system.actorOf(Props(new MessageHandler(system)))
  system.eventStream.subscribe(handler,
                     classOf[akka.actor.UnhandledMessage]);
  server ! "Hello World"
  server ! 3
  
  val serverRef = system.actorFor("akka://ServerTest/user/server")
  serverRef ! "Hello World"
  serverRef ! 3
}


/*
Terminal output:
received message: Hello World
unhandled message:3
received message: Hello World
unhandled message:3
*/
    \end{lstlisting}
    \caption{A String Processor in Akka}
\end{figure}

Undefined messages are treated differently in different actor libraries.  In
Erlang, an actor keeps undefined messages in its mailbox, attempts to process
the message again when a new message handler is in use.  In versions prior to
2.0, an Akka actor raises an exception when it processes an undefined message.
In recent Akka versions, an undefined message is discarded by the actor and an
{\tt UnhandledMessage} event is pushed to the event stream of the actor system.
The event stream may be subscribed by other actors who are interested in
particular event messages.  To handle the unexpected integer message in the
above short example, an event handler is defined and created with 6 lines of 
code.


\subsection{Supervision}
\label{akkasup}



Reliable Erlang applications typically adopt the Supervision Principle
 \cite{OTP}, which suggests that actors should be organized in a tree structure 
so that any failed actor can be properly restarted by its supervisor.  
Nevertheless, adopting the Supervision principle is optional in Erlang.

The Akka library makes supervision obligatory by restricting the way of 
creating actors. Actors can only be initialized by using the {\tt actorOf}
method provided by {\tt ActorSystem} or {\tt ActorContext}.  Each actor system
provides a guardian actor for all user-created actors.  Calling the {\tt
actorOf} method of an actor system creates an actor supervised by the 
guardian actor.  Calling the {\tt actorOf} method of an actor context 
creates a child actor supervised by that actor.  Therefore, all user-created  
actors in an actor system, together with the guardian actor of that actor 
system, form a tree structure.  Obligatory supervision unifies the 
structure of actor deployment and simplifies the work of system maintenance.

Each actor in Akka is associated with an actor path.  The string representation
of the actor path of a guardian actor has format 
{\it akka://mysystem@IP:port/user}, where {\it mysystem} is the name of the 
actor system, {\it IP} and {\it port} are the IP address and the 
port number which the actor system listens to, and {\it user} is the name of 
the guardian actor.  The actor path of a child actor is actor path of its 
supervisor appended by the name of the child actor, either a user specified 
name or a system generated name.

Figure \ref{supervisedcalculator} defines a simple calculator which supports
multiplication and division. The simple calculator does not consider the
problematic case of dividing a number by 0, where an {\tt
ArithmeticException} will be raised.  We then define a safe calculator as the 
supervisor of the simple calculator.  The safe calculator delegates 
calculation tasks to the simple calculator and restarts the simple calculator 
when an {\tt ArithmeticException} is raised.  The supervisor strategy of
the safe calculator also specifies the maximum number of failures its child may 
have within a given time range.  If the child fails more frequently than the 
allowed frequency, the safe calculator will be  stopped, and its failure will be
reported to its supervisor, the system guardian actor in this example.  The
terminal output shows that the simple calculator is restarted before the third 
message and the fifth message are delivered.  The last message is not processed
because both calculators are terminated since the simple calculator fails more 
frequently than allowed.

\begin{figure}
\label{supervisedcalculator}

  \begin{lstlisting}[language=scala]
case class Multiplication(m:Int, n:Int)
case class Division(m:Int, n:Int)

class Calculator extends Actor {
  def receive = {
    case Multiplication(m:Int, n:Int) =>
      println(m +" * "+ n +" = "+ (m*n))
    case Division(m:Int, n:Int) =>
      println(m +" / "+ n +" = "+ (m/n))
  }
}

class SafeCalculator extends Actor {
  override val supervisorStrategy =
    OneForOneStrategy(maxNrOfRetries = 2, withinTimeRange = 1 minute) {
      case _: ArithmeticException  =>
        println("ArithmeticException Raised to: "+self)
        Restart
    }
  val child:ActorRef = context.actorOf(Props[Calculator], "child")
  def receive = {
    case m => child ! m
  }
}

  val system = ActorSystem("MySystem")
  val actorRef:ActorRef = system.actorOf(Props[SafeCalculator],
"safecalculator")

  calculator ! Multiplication(3, 1)
  calculator ! Division(10, 0)
  calculator ! Division(10, 5)
  calculator ! Division(10, 0)
  calculator ! Multiplication(3, 2)
  calculator ! Division(10, 0)
  calculator ! Multiplication(3, 3)

/*
Terminal Output:
3 * 1 = 3
java.lang.ArithmeticException: / by zero
ArithmeticException Raised to: Actor[akka://MySystem/user/safecalculator]

10 / 5 = 2
java.lang.ArithmeticException: / by zero
ArithmeticException Raised to: Actor[akka://MySystem/user/safecalculator]
java.lang.ArithmeticException: / by zero

3 * 2 = 6
ArithmeticException Raised to: Actor[akka://MySystem/user/safecalculator]
java.lang.ArithmeticException: / by zero
*/
    \end{lstlisting}
  \caption{Supervised Calculator}
\end{figure}