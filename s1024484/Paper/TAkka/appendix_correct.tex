\section{Examples for Expressiveness and Correctness Test}
\label{app_correct}

\begin{description}
  \item[ATM simulator\cite{quviq} ] A bank ATM simulator with backend database
and frontend GUI.
  \item[Elevator Controller \cite{quviq}] A system that monitors and schedules a
number of elevators. 
  \item[Ping Pong \cite{akka_doc} ] A simple message passing application.
  \item[Dining Philosophers \cite{akka_doc}] A application that simulates the
dining philosophers problem using Finite State Machine (FSM) model.
  \item[Distributed Calculator \cite{akka_doc}] An application that examines
distributed computation and hot code swap.
  \item[Fault Tolerance \cite{akka_doc}] An application that demonstrates how
system responses to component failures.
  \item[Barber Shop\cite{BarberShop} ] A application that simulates the Barber
Shop problem. 
  \item[EnMAS \cite{EnMAS}] An environment and simulation framework for
multi-agent and team-based artificial intelligence research

  \item[Socko Web Server \cite{SOCKO} ]  lightweight Scala web server that can
serve static files and support RESTful APIs
  \item[Gatling \cite{Gatling}] A stress testing tool.
\end{description}


\section{Examples for Efficiency and Scalability Test}

\begin{description}
  \item [bang] This benchmark tests many-to-one message passing.  The benchmark
spawns a specified number sender and one receiver.  Each sender sends a
specified number of messages to the receiver.
  \item [big] This benchmark tests many-to-many message passing.  The benchmark
creates a number of actors that exchange ping and pong messages.
  \item [ehb] This is a benchmark and stress test.  The benchmark is
parameterized by the number of groups and the number of messages sent from each
sender to each receiver in the same group.
  \item [mbrot] This benchmark models pixels in a 2-D image.  For each pixel,
the benchmark calculates whether the point belongs to the Mandelbrot set.
  \item [parallel] This benchmark spawns a number of processes, where a list of
N timestamps is concurrently created.
  \item [genstress] This benchmark is similar to the bang test.  It spawns an
echo server and a number of clients.  Each client sends some dummy messages to
the server and waits for its response.  The Erlang version of this test can be
executed with or without using the gen\_server behaviour.  For generality, this
benchmark only tests the version without using gen\_server.
  \item [serialmsg] This benchmark tests message forwarding through a
dispatcher.
  \item [timer\_wheel] This benchmark is a modification to the big test.  While
responding to ping messages, a process in this message also waits pong messages.
 If no pong message is received within the specified timeout, the process
terminates itself.
  \item [ran] This benchmark spawns a number of processes.  Each process
generates a list of ten thousand random integers, sorts the list and sends the
first half of the result list to the parent process.
\end{description}