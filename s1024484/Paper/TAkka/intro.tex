\section{Introduction}

The Erlang/OTP (Open Telecom Platform) library \cite{ErlangWeb}
was released in 1996 for writing Erlang code using the Actor
model \cite{Hewitt:1973} together with five OTP design principles derived from
ten years' experience of Erlang programming. The OTP Design principles, 
especially the supervision principle, made it easier to build reliable 
distributed applications \cite{ArmstrongErlang}.

The notions of actors and supervision trees have been ported to statically 
typed languages including Scala and Haskell.  Scala actor libraries including 
Scala Actors \cite{actor_1, actor_2} and Akka \cite{akka_api,akka_doc} use 
dynamically typed messages even though Scala is a statically typed language.  
Cloud Haskell \cite{CloudHaskell}, a recent actor library, supports both 
dynamically and statically typed messages, but does not support supervision. 
Can actors in supervision trees be statically typed?

The key claim in this paper is that actors in supervision trees can be
typed by parameterizing the actor class with the type of messages it expects to
receive.  Type-parameterized actors benefit both users and developers
of actor-based services. For users, sending ill-typed messages is prevented at 
compile time.  Because messages are usually transmitted asynchronously, it may 
be otherwise difficult to trace the source of errors at runtime, especially in 
distributed environments.  For service developers, since unexpected messages 
are eliminated from the system, they can focus on the logic of the services 
rather than worrying about incoming messages of unexpected types.  Implementing 
type-parameterized actors in a statically-typed language; however, requires 
solving following three problems.

\begin{enumerate}
  \item A typed name server is required to retrieve actor references of
specific types.  A distributed system usually requires a name server that
maps names of services to processes that implement that service.  If processes
are dynamically typed, this is usually implemented as a map from names to 
processes. Can this be adapted to cases where processes are statically 
typed?

  \item Supervisor actors must interact with child actors of different types.  
Actors are structured in supervision trees to improve system reliability.  Each 
actor in a supervision tree needs to handle messages within its specific 
interests and also messages from its supervisor.  Is it practical to define a 
supervisor that communicates with children of different type parameters?

  \item Actors which receive messages from distinct parties may suffer from 
the type pollution problem, in which case a party imports too much type  
information about an actor and can send the actor messages not expected from it.
Systems built on layered architecture or the MVC model are often victims of the 
type pollution problem. As an actor receives messages from distinct parties 
using its sole channel, its type parameter is the union type of all expected 
message types.  Therefore, unexpected messages can be sent to actors which 
naively publishes its type parameter or permits dynamically typed messages. 
Can a type-parameterised actor have different types of communication interface, 
i.e. typed actor reference, when published to parties.
\end{enumerate}

Continuing a line of work on merging types with actor programming by Haller
and Odersky \cite{actor_1, actor_2} and Akka developers \cite{akka_doc}, along 
with the work on system reliability test by Netflix, Inc. \cite{ChaosMonkey} 
and Luna \cite{ErlangChaosMonkey}, this paper makes following contributions.

\begin{itemize}
 \item It presents the design and implementation of a novel typed name server
that maps typed names to values, for example actor references, of the 
corresponding type.  The typed name server (Section \ref{nameserver}) mixes 
static and dynamic type checking so that type  errors are detected at the earliest 
opportunity.  The implementation requires runtime support for type reflection and 
a notion of first class type descriptors.  In Scala, the {\tt Manifest} class 
provides such facility.

 \item It describes the design of the TAkka library.  Sections \ref{actor} to
\ref{actor_context} illustrate how type parameters are added to actor related
classes to improve type safety.   By separating the handler for system 
messages from the handler for user defined messages, Sections \ref{supervision}
and \ref{systemmessage} show that type-parameterized actors can form
supervision trees in the same manner as untyped actors. Section \ref{alternative
designs} compares the design of TAkka with alternatives adopted by other actor 
libraries.

 \item It shows that Akka programs can gradually migrate to their TAkka 
equivalents.  Section \ref{evolution} explains strategies of gradually 
upgrading Akka programs to their TAkka equivalents.
 
 \item It gives a straightforward solution to the type pollution problem 
(Section \ref{type_pollution}).  We present a simple API to cast
the type of an actor reference to its super type.  The semantics of the API is 
easy to understand and does not require deep understanding of
underlying concepts such as inheritance, polymorphism, and contravariant 
types.
 
 \item It gives a comprehensive evaluation of the TAkka library.  Results in
Section \ref{expressiveness} confirm that using type parameterized actors 
sacrifice neither expressiveness nor correctness.  Efficiency and 
scalability test in Section \ref{efficiency} shows that TAkka applications have 
little overhead at the initialization stage but have almost identical run-time 
performance and scalability compared to their Akka equivalents.  In 
addition, two helper libraries (Section \ref{reliability}) are shipped with the 
TAkka library.  The Chaos Monkey library are ported to test applications 
against intensive adverse conditions.  The novel Supervision View library is 
designed and implemented to dynamically monitor changes of supervision tree.
 
\end{itemize}
