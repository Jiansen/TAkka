\section{Introduction}


quick read


\begin{itemize}
  \item \S 4.1.5
  \item \S 2 \& \S 3
\end{itemize}

\begin{CJK*}{UTF8}{gbsn}


\end{CJK*}


\begin{comment}


\section{mini outline}

 
 

Expected Contribution:
1. A novel and more precise model for software rejuvenation.  Compared to the original model, the new model
    a. fixes the problem of using mean transition rate between states.
    b. adds the notion of S(t).  Instead of randomly jump to a healthy state (the original model), S(t) will evolve to either S(t+s) or R(0).
    c. merges the highly robust state (the unrealistic state where failure will not occur) and the a failure
probable state (where failures are likely to occur) to W(t), where the failure rate is a function about t.
    d.  split the transaction from the failure state to the highly robust state (i.e. W(0)) to two steps:  W(t) -> S(0) (data restore etc.) and S(t) -> R(0) (cleaning cache etc.). 

2. The approach of availability analysis can be generalized to analyze of the execution cost.
3. An efficient algorithm for approximate estimation on availability and other cost.


The research scope:

The first phrase of this research only aims at modelling a single component or a software system as a whole.  If time permits, we may extend the work to clusters, the widest scope I found in papers on software rejuvenation.

\end{comment}
