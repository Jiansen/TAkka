% \documentclass[10pt, journal, compsoc, authoryear]{IEEEtran}
\documentclass[10pt, journal, compsoc, authoryear]{article}

% \setpagenumber{1}g
\usepackage{CJKutf8}


\usepackage{comment}

\usepackage[numbers]{natbib}

\usepackage{graphicx}
\usepackage{subfigure}
\usepackage{url}

\usepackage{amsthm}
\usepackage{amsmath}
 \numberwithin{equation}{subsection}
\usepackage{paralist}
\usepackage[pdfpagelabels]{hyperref}
\usepackage[all]{hypcap}
\usepackage{verbatim}
\usepackage{array}
\usepackage{float}
\usepackage{multirow}
% \usepackage{rotating}
\usepackage{multicol}
\usepackage{longtable}
\usepackage{graphicx}


\theoremstyle{definition}
\newtheorem{defn}{Definition}
\newtheorem{conj}{Conjecture}
%\newtheorem{exmp}{Example}[section]
\newtheorem{exmp}{Example}
\newenvironment{where}{\noindent{}where\begin{itemize}}{\end{itemize}}
% A comment in a draft (shouldn't appear in the final version).
\newcommand{\mycomment}[1]{\(\spadesuit\){\bf #1 }\(\spadesuit\)}
% Comment this out in the draft
% \newcommand{\mycomment}[1]{}
\newcommand{\pmcomment}[1]{\comment{PM}{#1}}
\newcommand{\pwtcomment}[1]{\comment{PT}{#1}}
\newcommand{\rscomment}[1]{\comment{RS}{#1}}

% line break in table.  e.g.
% Foo bar & \specialcell{Foo\\bar} & Foo bar \\    % vertically centered
% Foo bar & \specialcell[t]{Foo\\bar} & Foo bar \\ % aligned with top rule
\newcommand{\specialcell}[2][c]{%
  \begin{tabular}[#1]{@{}c@{}}#2\end{tabular}}

% \permission {[Copyright notice will appear here once ’preprint’ option is 
% removed.]}

\begin{document}


\title{Towards Precise Analysis for Software Rejuvenation \\ (v 0.1) } % title

\author{Jiansen HE}

% \category{I.3.7}{Computer Graphics}{Three-Dimensional Graphics and Realism}[Animation]
% \category{I.3.5}{Computer Graphics}{Computational Geometry and Object Modeling}[Physically based modeling]

% \terms{Experimentation, Human Factors}



\maketitle


\begin{comment} 
\begin{abstract} 

Software rejuvenation refers to a preventive failure recovery technique, which 
restarts a software regularly to reduce the cost of software ageing.  It has been  
actively studied in the literature and applied in enterprise level applications. 
Traditionally, a software system is modelled as a Semi-Markov process, from 
which a rejuvenation schedule is deducted to optimise the objective function 
(e.g. highest availability, lowest cost) after the system reaches its steady-state.
Unfortunately, it can takes a long time for a system to reach its steady-state.  
We propose a approximate simulation method to study XXX


Continue a line of work by \citet{huang1995software} and 
\citet{dohi2000statistical}, this paper proposes an alternative Markov 
model for software rejuvenate.  Specifically, we expend each state, $S$, in 
traditional models to a group of states, $S(\tau)$, where $\tau$ represents 
the elapsed time that the system has been entered into state $S$.  
The above model is equivalent to a semi-Markov process.  It, however, brings two 
benefits and one limitation.  The first 
benefit is that state transition probabilities can be precisely described.  
Like in general reliability studies, the failure rate in the new model 
can be precisely described by an arbitrary probability density function or 
probability mass function.  In contrast, traditional models either use the 
average Mean Time to Failure (MTTF) for cost analysis or assume that the time 
spent in each state has an exponential distribution.  The second 
benefit is that transition probabilities of the new Markov process has take 
time into account.  The result model is either a simple Markov chain if time is 
a discrete value,  or a general Markov process if time is a continuous value.  
Comparing to traditional models that are semi-Markov processes or Continuous 
Time Markov Chains (CTMT), the Markov Chain used in proposed model is often 
easier to understand and analysis.  The downside of expending the state space 
is the decreased performance of simulation processes.  Fortunately, approximate 
estimations can be given to balance the trade-off between accuracy and 
efficiency.  Most importantly, we show that the classical model given by 
\citep{huang1995software} is the least accurate approximation.  Moreover, 
inspired by economical theories, a notion of {\it utility} function
is proposes paper to unify the cost analysis and the availability 
analysis, which are usually treated separately in previous studies.



\end{abstract}
\end{comment} 

% \begin{IEEEkeywords}
% key1; key2; key3; 

% \end{IEEEkeywords}

% \IEEEpeerreviewmaketitle


\section{Introduction}


quick read


\begin{itemize}
  \item \S 4.1.5
  \item \S 2 \& \S 3
\end{itemize}

\begin{CJK*}{UTF8}{gbsn}


\end{CJK*}


\begin{comment}


\section{mini outline}

 
 

Expected Contribution:
1. A novel and more precise model for software rejuvenation.  Compared to the original model, the new model
    a. fixes the problem of using mean transition rate between states.
    b. adds the notion of S(t).  Instead of randomly jump to a healthy state (the original model), S(t) will evolve to either S(t+s) or R(0).
    c. merges the highly robust state (the unrealistic state where failure will not occur) and the a failure
probable state (where failures are likely to occur) to W(t), where the failure rate is a function about t.
    d.  split the transaction from the failure state to the highly robust state (i.e. W(0)) to two steps:  W(t) -> S(0) (data restore etc.) and S(t) -> R(0) (cleaning cache etc.). 

2. The approach of availability analysis can be generalized to analyze of the execution cost.
3. An efficient algorithm for approximate estimation on availability and other cost.


The research scope:

The first phrase of this research only aims at modelling a single component or a software system as a whole.  If time permits, we may extend the work to clusters, the widest scope I found in papers on software rejuvenation.

\end{comment}

\section{Background}


\subsection{Basic Failure Model in Reliability Study}

In reliability study, the time when the first failure occurs in often assumed 
to follow a probability density function (pdf) if time is a continuous value, or 
a probability mass function (pmf) if time is a discrete value. \citep{MusaBook} 
 
The value of time is a continuous value if its unit is a {\it time unit} (e.g. 
second) and the value could be any real number.  The value of time is 
a discrete value if its unit is a {\it natural unit} (e.g. number of 
operations) or its unit is a {\it time unit} but the user only cares the 
status of a system at discrete time (e.g. every second).

Although the precise definition of failure varies in different systems, 
measures of failures can be consistently defined in terms of a pdf or a
pmf.  Assuming that the pdf of a system is $f(t)$, measures of failures can be 
defined as in Figure \ref{eq:measure_of_failures} 
\citep{MusaBook}.  Measures of failures can be defined in terms of pmf 
similarly.

\begin{figure}
\begin{subequations} 
\begin{align}
    R(t) & = \int_{t}^{\infty} f(x) dx = 1- \int_{0}^{t} f(x) dx  \label{subeq1}\\
   MTTF & = \int_{0}^{\infty} tf(t)dt  = \int_{0}^{\infty} R(t)dt 
\label{mttf}\\
   h(t) & = \lim_{\Delta t\rightarrow 0} \frac{R(t)-R(t+\Delta t)}{\Delta t 
R(t)}  = \frac{f(t)}{R(t)}
\end{align}
\end{subequations}
\caption{Measures of Failures}
\label{eq:measure_of_failures}
\end{figure}

In the above, the reliability during a specific time, $R(t)$, is the 
probability that a system can survive without failure throughout that time.  
The Mean Time to Failure (MTTF) is the average time when the first failure 
occurs.  The hazard rate at time $t$ is the failure rate at that time given 
the condition that the system has survived for $t$ time.  Equation 
\ref{failure_check} checks that the average failure rate ($\lambda$) is the 
reciprocal of MTTF.

\begin{equation}
 \lambda = \int_{0}^{\infty} h(t) dt = \int_{0}^{\infty} \frac{f(t)}{R(t)}  dt 
=  \frac{\int_{0}^{\infty} f(t) dt}{\int_{0}^{\infty} R(t) dt} = \frac{1}{MTTF}
\label{failure_check}
\end{equation}




\subsection{Traditional Software Rejuvenation Models}

\subsubsection{The Basic Model}


The basic software rejuvenation model proposed by
\citep{huang1995software} is modelled by the state transition diagram in 
Figure \ref{fig:model_classic_1}.  Huang et. al defines the following 4 states:

\begin{description}
  \item[S] The robust {\tt start} state in which the system will not fail.
  \item[W] The failure probable {\tt working} state in which the system may 
fail at a certain probability.
  \item[F] The {\tt failure} state when the system reaches its boundary condition.
  \item[R] The {\tt rejuvenation} state when the system is plan to 
restart to a clean state.
\end{description}

\citep{huang1995software} decides the rejuvenation thresholds using the 
following process.  

\begin{enumerate}
  \item compute the probabilistic {\it mean} rates of every state transitions 
somehow.
  \item calculate the probability of the system being in each state, based on 
the state transition rates computed above.
  \item calculate the {\it expected} total time of the system in state $s$ 
during an interval of $L$ time units as $T_s(L) = p_s \times L$, where $p_s$ is 
the probability the system being in state $s$ calculated in the second step.
  \item calculate the cost of downtime during interval $L$ as $Cost(L) = (p_f 
\times c_f + p_r \times c_r) \times L$, where $p_f$ and $p_r$ are probabilities 
the system in state $F$ and $R$ respectively, and $c_f$ and $c_r$ are unit cost 
of the system in state $F$ and $R$ respectively.
  \item determine the optimal rejuvenation thresholds by finding the optimal 
transition rate between state $W$ to state $F$ so that the cost of downtime is 
minimum.
\end{enumerate} 

The original software rejuvenation model, proposed by 
\citet{huang1995software}, has shown its significance in long running billing 
systems, scientific applications, and telecommunication systems 
\citep{huang1995software}. Readers may have noticed that the above estimation 
methodology gives an approximate statistical estimation for a long running 
system because the usage of {\it mean} state transition rates and the {\it 
expected} time of being in each state.  By {\it long running system}, we refer 
to a system whose expected in-service time is a statistically long time regards 
to its MTTF, repair time, and rejuvenation time.


\begin{figure}
  \centering         
        \subfigure[Traditional Model 1]{
            \label{fig:model_classic_1}
            \includegraphics[scale=0.5]{model_1.png}
        }\\
        \subfigure[Traditional Model 2]{
            \label{fig:model_classic_2}
            \includegraphics[scale=0.5]{model_2.png}
        }\\
    \caption{Traditional Software Rejuvenation Models}
   \label{fig:model_traditional}
\end{figure}

\subsubsection{A Modified Model}

\citet{dohi2000statistical} modifies the original model given 
by \citet{huang1995software} with the following modifications:

\begin{enumerate}
  \item As shown in Figure \ref{fig:model_classic_2}, the completion of repair process
  is immediately followed by the software rejuvenation process.
  \item Assuming that every state transition is a stochastic processes.  For each state $S$, define a 
  probability density function $F_s(t)$, which represents the probability that the system will stay in that
  state for time $t$. 
\end{enumerate} 

The first modification is made to distinguish the process of system cleanup and process resuming
from other repair tasks, the former of which is required in both the repair process and the rejuvenation
process in practice.  The second assumption is made so that the model is a Semi-Markov process,
a well studied stochastic process with off-the-shelf analysis techniques.  Although \citet{huang1995software}
do not mention the relationship between their model and Markov process, \citet{dohi2000statistical}
show that if the modified model gives the same analysis result as the one
given in \citet{huang1995software} if (i) remove assumption 1; and (ii) assume 
that the sojourn times in all states are exponentially distributed.

\citet{dohi2000statistical} decides the optimal software rejuvenation thresholds by looking for
the one that maximise the system availability, the ratio of uptime and total time.  Same as the
methodology in \citet{huang1995software}, instead of precisely describe state transition 
probabilities, $mean$ time of staying in the {\tt start} state, the {\tt failure} state, and the
{\tt rejuvenation} state is used.  The expected time of staying the failure probable {\tt working} state
is calculated by equation \ref{mttf}.

The methodology given in \citet{dohi2000statistical}  does not give a general cost
analysis as the one in \citet{huang1995software}.  It also suffers the problem of using
$mean$ times.  Despite those two limitations, \citet{dohi2000statistical} 
derived a non-parametric statistical algorithms to estimate the optimal 
software rejuvenation thresholds that maximise the system availability, based 
on statistical complete sample data of failure times.



\begin{comment}
The basic assumption of software rejuvenation is that software systems will 
become failure-probable after a period of execution.  In other words, failure 
rate is a function about the execution time t.  In contrast, the probabilistic 
state transition model 
\url{
http://citeseerx.ist.psu.edu/viewdoc/download?doi=10.1.1.176.2557&rep=rep1&type=
pdf} (FTCS 1995, citation: 711) employs the statistical mean rate of the 
transition rate between states.  

   The same model is used in the book 
\url{http://onlinelibrary.wiley.com/doi/10.1002/9780470050118.ecse394/full} 
(published in 2007)

   The same problem appears in the model used in 
\url{http://ieeexplore.ieee.org/xpls/abs_all.jsp?arnumber=897287} (HASE 2000, 
citation: 60) and 
\url{http://ieeexplore.ieee.org/stamp/stamp.jsp?tp=&arnumber=895436} (PRDC 2000, 
citation: 90), which extends the original model to be a semi-Markov process, 
where time is a continuous value.
      

Models that use the statistical mean rate can only be used to estimate the 
expected average behaviors (e.g. downtime) over a statistically long time.  A 
more serious problem is that, treating failure rate as a constant in model 
analysis violates the basic assumption of software rejuvenation.  After all, if 
the failure rate is independent of the execution time, restarting the system 
will not help at all but increase the downtime.

The link ( http://srejuv.ee.duke.edu/papers.htm ) gives a list of papers (until 
2010) on software rejuvenation.  I am surprised that the classic general failure 
model used in reliability analysis has not been applied to the model analysis of 
software rejuvenation.
\end{comment}

\section{A New Model}

\subsection{Definitions}

To overcome the limitations of classical models, which focus on the 
equilibrium status that will take a long time to reach, this paper propose a
new model that enables effective simulation on early-stages of a system using 
rejuvenation.  The new model works as the follows. It is initialised to a 
working state and its {\it software rejuvenate schedule} is set to time $T$.  
The failure rate during the working period is a function of the elapsed working 
time $t$.  A failed system takes $t_f$ time to repair.  At the end of a repair 
process or after working for $T$ time, the system enters to the rejuvenation 
process for $t_r$ time before being set to the initial working state.  Figure 
\ref{model_new} gives the transition digram for the new Model, which consists 
of  the following 3 {\it categories} of states:

\begin{itemize}
 \item {\tt W($\tau$), $0 \leq \tau < T$} the state that the system has running 
without failure for $\tau$ time;
 \item {\tt R($\tau$), $0 \leq \tau < t_r$} the state that the system has 
entered 
the rejuvenation process for $\tau$ time.
 \item {\tt F($\tau$), $0 \leq \tau < t_f$} the state that the system has 
failed 
for $\tau$ time, during which period no rejuvenation process has been invoked, 
but other repair processes, such as data restore, maybe performed.
\end{itemize}

For the convince of later discussions, this paper index possible state $x_i$ as 
follows:

\begin{defn}
\[ x_i = \left\{ 
  \begin{array}{l l}
    W(i)    &   0 \leq i < T \\
    R(i-T) &   T \leq i < T+t_r \\ 
    F(i-T-t_r) &   T+t_r \leq i < T+t_r+t_f \\     
  \end{array} \right.\]
\end{defn}

Let $X_t$ be the state of the system at time $t \geq 0$, then the process $X = 
(X_t, t \geq 0)$ is a stochastic process.  If time is a discrete value in the 
model, the process has discrete time and discrete state space.  
If time is a  continuous value in the model, the process has 
continuous time and continuous state space.  




\begin{figure}
  \label{model_new}
  \centering       
  \includegraphics[scale=0.5]{model.png}
  \caption{New Model}
\end{figure}


\subsection{Cast Continues State Space to Discrete State Space}

As Section \ref{discrete_model} will show, a standard Markov Chain can be 
defined if time is treated as a discrete value.  Section 
\ref{discrete_simulation} will give the simulation process corresponding to 
the Markov Chain.  Although a general Markov process can be defined similarly 
for the case where time is a continuous value, unfortunately, the general 
Markov process does not have a precise simulation method.  Therefore, this 
paper converts the continuous case to an approximate discrete case, where a 
simulation method can be applied.  

To convert a model with continuous state space to one with discrete state space,
a new time unit, $\Delta t >  0$, is picked such that 
$T/\Delta t$, $t_r/\Delta t$, and $t_f/\Delta t$ are integers.  Let $pdf$ be 
the failure density function in the continues model, then the failure massive 
function at $T+\Delta t$ in the discrete model is 

\begin{equation}
pmf(T+\Delta t) =  \int_{T}^{T+\Delta t} pdf(t) dt
\end{equation}

Finally, times $t$ in the continuous model can be converted to $t/\Delta t$ in 
the new discrete model.

\begin{comment}
\subsubsection{The Transaction Probability Density}

Let $p_{ij}$ be the probability of moving from $X_i$ to $X_j$ and $0 \le j-i 
\le 
min\{T, t_r, t_f\}$.  
We have:

\begin{equation}
p_{ij} = 
\begin{cases} 
\frac{R(k+j-i)}{R(k)}    &  if\ x_i = W(k), \\ &\hspace{12pt}  x_j = W(k+j-i), 
\\ &\hspace{12pt}  0\leq k <T+i-j; \\

\frac{R(T)}{ R(k)}     &  if\ x_i = W(k),   x_j = F(l), \\ &\hspace{12pt}  
0\leq 
k < T, 0 \leq l < j-i; \\ 

\frac{R(T)}{R(k)}     &   if\ x_i = W(T-k),   x_j = R(k),\\ &\hspace{12pt}   0 
\leq k < j-i; \\

\int_{0}^{w} f(x)dx     &  if\ x_i = R(k),   x_j = F(l), \\ &\hspace{12pt}  
0\leq k < t_r, 0 \leq l < t_f; \\ &\hspace{12pt}  w = (j-i)-(t_r-k)-(t_f-l) \\

1     &  if\ x_i = R(k),  x_j = R(k+j-i),\\ &\hspace{12pt}   0\leq k <t_r+i-j, 
\\
       &  or\ x_i = F(k),   x_j = F(k+j-i),\\ &\hspace{12pt}   0\leq k 
<t_f+i-j, 
\\     
       &  or\ x_i = F(t_f-k),   x_j = R(k),\\ &\hspace{12pt}   0 \leq k < t_r, 
\\      
       &   or\ if\ x_i = R(t_r-k), \\ &\hspace{12pt}  x_j = W(k),   0 \leq k < 
T; \\       
0     &   otherwise
\end{cases}
\end{equation}
\hspace{90pt} where $0 \le j-i \le min{T, t_r, t_f}$
\end{comment}


\subsection{Discrete Model as a Markov Chain}
\label{discrete_model}

\subsubsection{Probability Mass Function and Failure Rate}

Let $f(t)$ be the probability mass function (pmf) which represents the 
probability that, without software rejuvenation, the system will fail when 
being running for exactly $t$.  We can derive its reliability function ($R(t)$) 
and the failure rate function ($\lambda(t)$) as follows:

\begin{subequations}
\begin{align}
R(t) & = 1- \sum \limits_{i=1}^{t-1} f(t) & t > 0 \label{discrete_reliability}\\
\lambda(t) & = f(t)/R(t)  & t>0 \label{discrete_failure_rate}
\end{align}
\end{subequations}

Intuitively, $R(t)$ is the probability that the system can survive for as least 
$t$ time and $\lambda(t)$ is the probability that the system, which has 
survived for time $t$, fails at the next moment.

\subsubsection{The Transaction Matrix}

We define the transaction matrix $P$ whose $(i,j)$th element is the probability 
of moving from $x_i$ to $x_j$ in exactly {\it one} unit of time,  we have:

\begin{equation*}
p_{ij} = 
\begin{cases} 
1 - \lambda(k) & if\ x_i = W(k), x_j = W(k+1),\\ &\hspace{12pt}  0\leq k <T-1; 
\\
\lambda(k)     & if\ x_i = W(k), x_j = F(0),\\ &\hspace{12pt}  0\leq k <T-1,  \\
1              & if\ x_i = W(T-1), x_j = R(0), \\
               & or\ x_i = R(k), x_j = R(k+1),\\ &\hspace{12pt}  0\leq k < 
t_r-1, \\
               & or\ x_i = R(t_r), x_j = W(0), \\
               & or\ x_i = F(k), x_j = F(k+1),\\ &\hspace{12pt}  0\leq k <t_f-1, 
\\
               & or\ x_i = F(t_f), x_j = R(0); \\
0              & otherwise
\end{cases}
\end{equation*}

It is easy to verify that $P$ satisfies the following two properties of {\it stochastic matrix}:
\begin{enumerate}[\itshape a\upshape)]
\item $p_{ij} \ge 0$, and
\item $\sum \limits_{i} p_{ij} = 1$.
\end{enumerate}



\subsubsection{Simulation}
\label{discrete_simulation}

Let $\mu_t$ be a row vector of probabilities so that $\mu_t(i)$ represents the 
probability that the system is at state $x_i$ at time $t$, we can inductively 
simulate $\mu_t$ as the follows:

\begin{subequations}
\label{discrete_simulation_mu}
\begin{align}
\mu_0 & =  \begin{cases}  \mu_0(0) = 1 & \\
    \mu_0(i) = 0 & 0 < i < T+t_r+t_f -1
\end{cases}  \label{mu_0}\\
\mu_t & = \mu_0 P  &  \label{mu_t}
\end{align}
\end{subequations}
Particularly, a row vector $\pi$ is called a stationary distribution if $\pi = 
\pi P$.

On the other hand, the Chapman-Kolmogorov Equation states the follows:

\begin{equation}
\label{discrete_CK}
p_{ij}(m+n) = \sum \limits_k p_{ik}(m)p_{kj}(n)
\end{equation}

where $p_{ij}(n)$ is the probability that the system moves from state $X_i$
to state $X_j$ with in exactly $n$ steps.  

A nice corollary of Equation \ref{discrete_CK} is:
\begin{equation}
\label{discrete_CK_corollary}
P_n = P^n
\end{equation}
where $P_n$ is the transaction matrix whose $(i,j)$th element is the probability 
of moving from $x_i$ to $x_j$ in exactly $n$ unit of time.

From either Equation \ref{discrete_simulation_mu} or Equation 
\ref{discrete_CK_corollary}, we have
\begin{equation}
\label{discrete_CK_corollary}
\mu_t = \mu_0 P^t
\end{equation}
a standard property of (time-homogeneous) Markov Chain.



\subsubsection{Utility Analysis}

Borrowed from economics, the notion of utility function is used to unifies
downtime cost analysis, availability analysis, and other variants in literature.

Let $u(i)$ be {\it utility function} that returns a real value if the system stays
at state $x_i$ for one unit of time, the $expected$ utility gained at time $t$,
and the $total$ utility gained until time $t$ are Equation \ref{discrete_utility_t}
and Equation \ref{discrete_utility_T} respectively.

\begin{subequations}
\label{discrete_utility}
\begin{align}
u_t  & =   \sum \limits_{i=0}^{T+t_r+t_f-1} u(i)\mu_t(i)  \label{discrete_utility_t}\\
U_t  & =  F_{i=0}^{t} u_i  \label{discrete_utility_T}
\end{align}
\end{subequations}
where $F$ is an accumulating function.


\newpage
CASE I: Availability Analysis

The utility function for availability analysis can be defined as:

\begin{equation}
\label{discrete_availability_utiliy}
u(i) =  \begin{cases}  1 & 0 \leq i < T \\
                       0 & T \leq i < T+t_r+t_f -1
\end{cases}
\end{equation}


The target of \citep{dohi2000statistical} is finding a value $T$ so that
$a = \sum \limits_{i=0}^{T+t_r+t_f-1} u(i)\pi(i) $ has a maximum value.
Notice that, the changes of systems availability before the Markov model 
reaches its steady-state $\pi$ is ignored in \citep{dohi2000statistical}.

For a safety crucial system expected to run $t$ time, users may want to use the 
following objective function instead.

\begin{equation}
\label{discrete_availability_minimum}
A_{min}(t) = \min_{i=0}^t u_i
\end{equation} 

The average availability of a system during its first $t$ time is:

\begin{equation}
\label{discrete_availability_average}
\bar{A}(t) = \frac{1}{ t }\sum \limits_{i=0}^{t} u_i
\end{equation}

If the system can reaches its steady-state $\pi$, one may expect that 
\begin{equation}
  \lim_{t\rightarrow \infty}{\bar{A}(t)} = a
\end{equation}

CASE II: Downtime Cost Analysis

As in \citep{huang1995software}, let $c_f$ and $c_r$ be the unit cost 
of the system in state $F$ and $R$ respectively, we have:

\begin{equation}
\label{discrete_downtime_utiliy}
u(i) =  \begin{cases}   0 & 0 \leq i < T \\
                       -c_r & T \leq i < T + t_r \\
                       -c_f & T + t_r \leq i < T + t_r + t_f
        \end{cases}
\end{equation}

The target of \citep{huang1995software} is finding a value $T$ so that
$c = \sum \limits_{i=0}^{T+t_r+t_f-1} u(i)\pi(i) $ has a maximum value.
Notice that, the changes of systems availability before the Markov model 
reaches its steady-state $\pi$ is ignored in \citep{dohi2000statistical}.
Actually, the expected total cost of running the system for $L$ time is:
\begin{equation}
\label{discrete_availability_total}
C(L) = \sum \limits_{t=0}^{L} u_t
\end{equation}

Only when the system can reaches its steady-state $\pi$, , one may expect 
that 
\begin{equation}
  \lim_{L\rightarrow \infty}{C(L)} = cL
\end{equation}

\newpage

CASE III: Benefit Analysis

The utility function in CASE I returns non-negative value while the utility 
function in CASE II returns non-positive value.  In reality, a software 
application gains revenue when it runs and cost resources to restart it when it 
is down.

Now consider an international Voice over Internet Protocol (VoIP) service 
provider, who implements a few independent front-end applications, each of 
which has the following utility function, where the unit of time is hour:

\begin{equation}
\label{discrete_benifit_utiliy}
u(i) =  \begin{cases}     r & 0 \leq i < T \\
                       -c_r & T \leq i < T + t_r \\
                       -c_f & T + t_r \leq i < T + t_r + t_f
        \end{cases}
\end{equation}

The goal is set a proper rejuvenation schedule $T$ to maximize the total 
utility gained during the expect service time $L$ defined as: 
\begin{equation}
\label{discrete_benifi_total}
U(L) = \sum \limits_{t=0}^{L} u_t = \sum \limits_{t=0}^{L}  \sum 
\limits_{i=0}^{T+t_r+t_f-1} u(i)\mu_t(i)
\end{equation}


\subsection{Reflection on Markov Processes}

The model in \citep{huang1995software} has its root in Continuous Time Markov 
Chain.  The model in \citep{dohi2000statistical} is a Continuous Time 
Semi-Markov Chain. Our model is a devised Discrete Time Markov Chain (DTMC) or 
a General Markov Process.

The General Markov Process is a precise model for software rejuvenation, but it 
may not have an analysitcal form for the simulation purpose.  Our devised 
DTMC adds a time parameter to standard DTMC, result in a special form of CTMC 
and SMC.

The problem of our devised DTMC is the expanded state space. Recall that the 
simulation process, $\mu_t = \mu_{t-1} P$, computes the product of a $1 \times 
n$ matrix and a $n \times n$ matrix, where $n$ is the number of states in the 
model.  It appears that the computational cost of the simulation 
process of the devised DTMC is impractical in a general case where $n$ is a 
large number.

Fortunately, the devised DTMC is an acceptable model for the software 
rejuvenation problem discussed in this paper for two reasons.  Firstly, 
non-zero values in the state transition matrix are sparsely distributed, and 
therefore matrix multiplication does not necessary require a long time to 
compute.   Secondly, similar to the process of reducing a continuous model to a 
discrete model, the state space of a descrete model can be further reduced for 
achieving better performance, with the cost of sacrifying accuracy.
\section{Numerical Illustrations}

\subsection{Examples from \citep{huang1995software}}

\subsubsection{Example A}
\begin{itemize}
  \item Mean Time Between Failures (MTBF): $12 \times 30 \times 24$ hours
  \item failure repair time:  30 minutes
  \item base longevity interval:  $7 \times 24$ hours
  \item rejuvenation time: 20 minutes
  \item average cost of unscheduled downtime: \$1700 / hour
  \item average cost of scheduled downtime: \$40 / hour  
\end{itemize}

\begin{table}[h]
\begin{tabular}{ | l || c | c | c | }
  \hline
   \multicolumn{4}{|c|}{For $12 \times 30 \times 24 $ hours } \\
  \hline                       
    & no rejuvenation & once every 3 weeks & once every two weeks \\
   \hline     
  Hours of Down Time  & 0.49 & 5.965 & 8.727 \\
  \hline    
  \$Cost of Down Time & 490  & 554 & 586 \\
  \hline 
\end{tabular}
  \caption{Huang etc's estimation for Example A}
\end{table}  
  


\subsubsection{Example B}
\begin{itemize}
  \item Mean Time Between Failures (MTBF): $3 \times 30 \times 24$ hours
  \item failure repair time:  30 minutes
  \item base longevity interval:  $3 \times 24$ hours
  \item rejuvenation time: 10 minutes
  \item average cost of unscheduled downtime: \$5000 / hour
  \item average cost of scheduled downtime: \$5 / hour  
\end{itemize}

\begin{table}[h]
\begin{tabular}{ | l || c | c | c | }
  \hline
   \multicolumn{4}{|c|}{For $12 \times 30 \times 24 $ hours } \\
  \hline                       
    & no rejuvenation & once every 2 weeks & once a week \\
   \hline     
  Hours of Down Time  & 1.94 & 5.70 & 9.52 \\
  \hline    
  \$Cost of Down Time & 9675.25  & 7672.43 & 5643.31 \\
  \hline  
\end{tabular}
  \caption{Huang etc's estimation for Example B}
\end{table}  
  

\begin{figure}[h]
     \begin{center}
     \subfigure[]{
\includegraphics[scale=0.18]{./plot/Huang1995B2/accumulated_cost_day.png}
        }
\subfigure[]{
\includegraphics[scale=0.18]{./plot/Huang1995B2/accumulated_cost_month.png}
        }\\
        \subfigure[]{            
\includegraphics[scale=0.18]{./plot/Huang1995B3/accumulated_cost_day.png}
        }
	\subfigure[]{            
\includegraphics[scale=0.18]{./plot/Huang1995B3/accumulated_downtime_month.png}
        }\\ 
        
        \subfigure[]{
\includegraphics[scale=0.18]{./plot/Huang1995B2/availability_day.png}
        }
\subfigure[]{
\includegraphics[scale=0.18]{./plot/Huang1995B2/availability_month.png}
        }\\
        \subfigure[]{            
\includegraphics[scale=0.18]{./plot/Huang1995B3/availability_day.png}
        }
	\subfigure[]{            
\includegraphics[scale=0.18]{./plot/Huang1995B3/availability_month.png}
        }\\ 
        
             \subfigure[]{
\includegraphics[scale=0.18]{./plot/Huang1995B2/acc_rej_time_day.png}
        }
\subfigure[]{
\includegraphics[scale=0.18]{./plot/Huang1995B2/acc_rej_time_month.png}
        }\\
        \subfigure[]{            
\includegraphics[scale=0.18]{./plot/Huang1995B3/acc_rej_time_day.png}
        }
	\subfigure[]{            
\includegraphics[scale=0.18]{./plot/Huang1995B3/acc_rej_time_month.png}
        }\\

    \end{center}
     \caption{Example B}
   \label{throughput}
\end{figure}

\subsubsection{Example C}
\begin{itemize}
  \item Mean Time Between Failures (MTBF): $3 \times 30 \times 24$ hours
  \item failure repair time:  $2$ hours
  \item base longevity interval (BLI):  $10 \times 24$ hours
  \item rejuvenation time: 10 minutes
  \item average cost of unscheduled downtime: \$5000 / hour
  \item average cost of scheduled downtime: \$5 / hour  
\end{itemize}

\begin{table}[h]
\begin{tabular}{ | l || c | c | c | }
  \hline
   \multicolumn{4}{|c|}{For $12 \times 30 \times 24 $ hours } \\
  \hline                       
    & no rejuvenation & once every 2 weeks & once a week \\
   \hline     
  Hours of Down Time  & 7019 & 6.83 & 6.36 \\
  \hline    
  \$Cost of Down Time & 3.6K  & 2.48K & 1.11K \\
  \hline  
\end{tabular}
  \caption{Huang etc's estimation for Example C}
\end{table}  
  

\begin{figure}[h]
     \begin{center}
     \subfigure[]{
\includegraphics[scale=0.18]{./plot/Huang1995C2/accumulated_cost_day.png}
        }
\subfigure[]{
\includegraphics[scale=0.18]{./plot/Huang1995C2/accumulated_cost_month.png}
        }\\
%        \subfigure[]{            
%\includegraphics[scale=0.18]{./plot/Huang1995C3/accumulated_cost_day.png}
%        }
%	\subfigure[]{            
%\includegraphics[scale=0.18]{./plot/Huang1995C3/accumulated_downtime_month.png}
%        }\\ 
        
        \subfigure[]{
\includegraphics[scale=0.18]{./plot/Huang1995C2/availability_day.png}
        }
\subfigure[]{
\includegraphics[scale=0.18]{./plot/Huang1995C2/availability_month.png}
        }\\
%        \subfigure[]{            
% \includegraphics[scale=0.18]{./plot/Huang1995C3/availability_day.png}
%        }
%	\subfigure[]{            
% \includegraphics[scale=0.18]{./plot/Huang1995C3/availability_month.png}
%        }\\ 
        
             \subfigure[]{
\includegraphics[scale=0.18]{./plot/Huang1995C2/acc_rej_time_day.png}
        }
\subfigure[]{
\includegraphics[scale=0.18]{./plot/Huang1995C2/acc_rej_time_month.png}
        }\\
%        \subfigure[]{            
%\includegraphics[scale=0.18]{./plot/Huang1995C3/acc_rej_time_day.png}
%        }
%	\subfigure[]{            
%\includegraphics[scale=0.18]{./plot/Huang1995C3/acc_rej_time_month.png}
%        }\\   
    \end{center}
     \caption{Example C}
\end{figure}


\subsubsection{Simulation Results}

\begin{itemize}
  \item 4 experiments for each example: two different rejuvenation schedules, 
two simulated longevity (1 year and 20 years)
  \item basic time unit: 10 minutes
  \item failure distribution: uniform distributio, $U(BLI, 2\times MTBF-BLI)$
\end{itemize}





\subsubsection{Observations}


\begin{itemize}
  \item In each of the above examples, it takes more than 1.5 years to reach steady-state (see availability graph).
  the property of a reliable system 1.5 years from now is less valuable as the its property now.
  \item The cost for the first year is lower than the expected annual cost
  \item For a reliable system, the main cost is the rejuvenation cost.
\end{itemize}



\subsection{Examples from Dohi}

Use the same parameter used in the paper, assume weibull distribution as in the paper







\section{Efficiency}

In Example HuangB \& HuangC.  The time is set to 10 minutes.  It takes 1 hour to
simulate the first year.  Not bad!

The matrix is sparse, space efficiency can be improved.


TODO: measure the complexity 


\section{TODO List}


\begin{itemize}
  \item Examples from Dohi
  \item Space Efficiency improvement
  \item ! compose sub-systems
\end{itemize}



\section{Conclusion}


% \section*{Acknowledgment}


% The authors would like to thank... more thanks here


\bibliographystyle{IEEEtranN}
\bibliography{rejuvenation}

% \appendix
%  \section{Examples for Expressiveness and Correctness Test}
\label{app_correct}

\begin{description}
  \item[ATM simulator\cite{quviq} ] A bank ATM simulator with backend database
and frontend GUI.
  \item[Elevator Controller \cite{quviq}] A system that monitors and schedules a
number of elevators. 
  \item[Ping Pong \cite{akka_doc} ] A simple message passing application.
  \item[Dining Philosophers \cite{akka_doc}] A application that simulates the
dining philosophers problem using Finite State Machine (FSM) model.
  \item[Distributed Calculator \cite{akka_doc}] An application that examines
distributed computation and hot code swap.
  \item[Fault Tolerance \cite{akka_doc}] An application that demonstrates how
system responses to component failures.
  \item[Barber Shop\cite{BarberShop} ] A application that simulates the Barber
Shop problem. 
  \item[EnMAS \cite{EnMAS}] An environment and simulation framework for
multi-agent and team-based artificial intelligence research

  \item[Socko Web Server \cite{SOCKO} ]  lightweight Scala web server that can
serve static files and support RESTful APIs
  \item[Gatling \cite{Gatling}] A stress testing tool.
\end{description}


\section{Examples for Efficiency and Scalability Test}

\begin{description}
  \item [bang] This benchmark tests many-to-one message passing.  The benchmark
spawns a specified number sender and one receiver.  Each sender sends a
specified number of messages to the receiver.
  \item [big] This benchmark tests many-to-many message passing.  The benchmark
creates a number of actors that exchange ping and pong messages.
  \item [ehb] This is a benchmark and stress test.  The benchmark is
parameterized by the number of groups and the number of messages sent from each
sender to each receiver in the same group.
  \item [mbrot] This benchmark models pixels in a 2-D image.  For each pixel,
the benchmark calculates whether the point belongs to the Mandelbrot set.
  \item [parallel] This benchmark spawns a number of processes, where a list of
N timestamps is concurrently created.
  \item [genstress] This benchmark is similar to the bang test.  It spawns an
echo server and a number of clients.  Each client sends some dummy messages to
the server and waits for its response.  The Erlang version of this test can be
executed with or without using the gen\_server behaviour.  For generality, this
benchmark only tests the version without using gen\_server.
  \item [serialmsg] This benchmark tests message forwarding through a
dispatcher.
  \item [timer\_wheel] This benchmark is a modification to the big test.  While
responding to ping messages, a process in this message also waits pong messages.
 If no pong message is received within the specified timeout, the process
terminates itself.
  \item [ran] This benchmark spawns a number of processes.  Each process
generates a list of ten thousand random integers, sorts the list and sends the
first half of the result list to the parent process.
\end{description}

% \received{September 2008}{March 2009}

\end{document}


