\chapter{Evolution, Not Revolution }
\label{evolution}

\begin{center}
(This chapter is expanded from\citep{TAKKA_paper} , Chapter 5)
\end{center}
\vspace{12 pt}

Akka systems can be smoothly migrated to TAkka systems. In other words, 
existing systems can evolve to introduce more types, rather than requiring a 
revolution where all actors and interactions must be typed.
The above property is analogous to adding generics to Java programs.  Java 
generics are carefully designed so that programs without generic types can be 
partially replaced by an equivalent generic version (evolution), rather than 
requiring generic types everywhere (revolution) \citep{JGC}.

Figure~\ref{akka_supervised_calculator} and Figure~\ref{takka_safe_calculator} 
presents how to define and use a safe calculator
in the Akka and TAkka systems respectively.  Think of a {\tt SafeCalculator} actor
as a service and its reference as a client interface.  The following sections show how to upgrade
the Akka version to the TAkka version gradually, either upgrading the 
service implementation first or the client interface.


\section{TAkka Service with Akka Client}

It is often the case that an actor-based service is implemented by one 
organization but used in a client application implemented by another.  
Let us assume that a developer decides to upgrade the service 
using TAkka actors, for example, by upgrading the Socko Web Server 
\citep{SOCKO}, the Gatling stress testing tool \citep{Gatling}, or the core 
library of Play \citep{play_doc}, as we do in Section~\ref{expressiveness}.  
Will the upgrade affect legacy client applications
built using the Akka library?  Fortunately, no changes are required at all.

As the TAkka {Actor} class inherits the Akka {\tt Actor} class, it can be used 
to create an Akka actor.  For example, the object {\tt akkaCal}, created at line 
5 in Figure~\ref{takkaINakka}, is created from a TAkka actor and used as an Akka
actor reference.  After the service developer has upgraded all actors to equivalent 
TAkka versions, the developer may want to start a TAkka actor system.  Until that
time, the developer can create TAkka actor references but publish their untyped
version to users who are working in the Akka environment (line 19).
As a result, no changes are required for a client 
application that uses Akka actor references.  Because an Akka actor reference 
accepts messages of any type, messages of unexpected type may be sent to TAkka actors.  
As a result, handlers for the {\tt UnhandledMessage} event is required in a
careful design (line 10 and 20).


\begin{figure}[h]
      \begin{lstlisting}[language=scala, escapechar=?]
import sample.?\textcolor{blue}{takka}?.SafeCalculator.SafeCalculator      
      
object TSAC extends App {
  val akkasystem = ?\textcolor{blue}{akka}?.actor.ActorSystem("AkkaSystem")
  val akkaCal = akkasystem.actorOf(
    ?\textcolor{blue}{akka}?.actor.Props[SafeCalculator], "acal")
  ?\textcolor{blue}{val handler = akkasystem.actorOf(}?
    ?\textcolor{blue}{akka.actor.Props(new MessageHandler(akkasystem)))}?
  
  ?\textcolor{blue}{akkasystem.eventStream.subscribe(handler,}?
     ?\textcolor{blue}{classOf[UnhandledMessage]);}?
  akkaCal ! Multiplication(3, 1)     
  ?\textcolor{blue}{akkaCal ! "Hello Akka"}?
  
  val takkasystem = ?\textcolor{blue}{takka}?.actor.ActorSystem("TAkkaSystem")
  val takkaCal = takkasystem.actorOf(
     takka.actor.Props[?\textcolor{blue}{String,}? TAkkaStringActor], "tcal")
  
  val untypedCal= takkaCal.?\textcolor{blue}{untypedRef}?  
  ?\textcolor{blue}{takkasystem.system.eventStream.subscribe(}?
    ?\textcolor{blue}{handler,classOf[UnhandledMessage]);}?  
  untypedCal ! Multiplication(3, 2)     
  untypedCal ! "Hello TAkka"
}
/* Terminal output:
3 * 1 = 3
unhandled message:Hello Akka
3 * 2 = 6
unhandled message:Hello TAkka
*/
    \end{lstlisting}
    \caption{TAkka Service with Akka Client}
\label{takkaINakka}    
\end{figure}


% \vspace{-5pt}
\section{Akka Service with TAkka Client}

Sometimes developers want to update the client code or API before 
upgrading the service implementation. For example, a developer may not have access to 
the service implementation; or the service implementation may be large, so the 
developer may want to upgrade the library gradually.

Users can initialize a TAkka actor reference by providing an Akka actor 
reference and a type parameter.  In Figure~\ref{akkaINtakka}, we re-use the 
Akka calculator, initialise it in an Akka actor system, and obtain an Akka 
actor reference.  Then, we wrap the Akka actor reference as a TAkka 
actor reference, {\tt takkaCal}, which only accepts messages of type
{\tt Operation}.

\begin{figure}[h]
      \begin{lstlisting}[language=scala, escapechar=?]
import sample.?\textcolor{blue}{takka}?.SafeCalculator.SafeCalculator      
      
object TSAC extends App {
  val akkasystem = ?\textcolor{blue}{akka}?.actor.ActorSystem("AkkaSystem")
  val akkaCal = akkasystem.actorOf(
    ?\textcolor{blue}{akka}?.actor.Props[SafeCalculator], "acal")
  ?\textcolor{blue}{val handler = akkasystem.actorOf(}?
    ?\textcolor{blue}{akka.actor.Props(new MessageHandler(akkasystem)))}?
  
  ?\textcolor{blue}{akkasystem.eventStream.subscribe(handler,}?
     ?\textcolor{blue}{classOf[UnhandledMessage]);}?
  akkaCal ! Multiplication(3, 1)     
  ?\textcolor{blue}{akkaCal ! "Hello Akka"}?
  
  val takkasystem = ?\textcolor{blue}{takka}?.actor.ActorSystem("TAkkaSystem")
  val takkaCal = takkasystem.actorOf(
     takka.actor.Props[?\textcolor{blue}{String,}? TAkkaStringActor], "tcal")
  
  val untypedCal= takkaCal.?\textcolor{blue}{untypedRef}?  
  ?\textcolor{blue}{takkasystem.system.eventStream.subscribe(}?
    ?\textcolor{blue}{handler,classOf[UnhandledMessage]);}?  
  untypedCal ! Multiplication(3, 2)     
  untypedCal ! "Hello TAkka"
}
/* Terminal output:
3 * 1 = 3
unhandled message:Hello Akka
3 * 2 = 6
unhandled message:Hello TAkka
*/
    \end{lstlisting}
    \caption{TAkka Service with Akka Client}
\label{akkaINtakka}
\end{figure}
