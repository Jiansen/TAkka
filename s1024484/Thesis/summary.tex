% !TEX encoding = GBK
\chapter{Future Work}
\label{future_work}

The work presented in this thesis confirms that actors in supervision trees can
be typed by parameterising the actor class with the type of messages it expects
to receive. The results of primary evaluations show that the TAkka library 
can prevent some errors without bringing obvious overheads compared with 
equivalent Akka applications. As actors and supervision trees are widely used 
in the development of distributed applications nowadays, the author believes 
that there is great potential for the TAkka library. Much more can be done to 
make TAkka more usable, as well as to further the goal of making the building 
of reliable distributed applications easier.


\section{Supervision and Typed Actor in Other Systems}

The result of this thesis confirms the feasibility of using type parameterized 
actors in a supervision tree.  The resulting TAkka library is built on top of 
Akka for the following three considerations:  Firstly, both actor and 
supervision have been implemented in Akka.  The legacy work done by the Akka 
developers makes it possible for us to focus on the core research question. 
That is, to what extent can actors in a supervision tree be statically typed?  
Secondly, Akka is 
built in Scala, a language that has a flexible type system.  The flexibility 
provided by Scala allows the author to explore types in a supervision tree.  In 
TAkka, dynamic type checking is only used when static type checking meets its 
limitations. Thirdly, Akka is a popular programming framework.  As part of the 
Typesafe stack, Akka has been used for developing applications in different 
sizes and for different purposes.  If Akka applications can be gradually 
upgraded to TAkka applications, the author believes that the type checking 
feature in TAkka can improve the reliability of existing Akka systems.

Actor programming has been ported to many languages.  The notion of type 
parameterized actors, however, was introduced very recently in libraries such 
as Cloud Haskell \citep{CloudHaskell} and scalaz \citep{scalaz}.  It has been 
proposed to implement a supervision tree in Cloud Haskell 
\citep{OTPCloudHaskell}.  The author believes that the techniques used in this 
thesis canhelp the design of the future version of Cloud Haskell.



\section{Benchmark Results from Large Real Applications}

This thesis compares TAkka with Akka with regards to several dimensions by 
porting small and medium sized applications. Most of selected examples are 
from open source projects. The author gratefully acknowledges the RELEASE team 
for giving us access to the source code of the BenchErl benchmark examplesl; 
Thomas Arts from QuviQ.com and Francesco Cesarini from Erlang Solutions for 
providing the Erlang source code for the ATM simulator example and the Elevator 
Controller example, both of which are used in their commercial training 
courses. Nevertheless, experiments on large real applications are not 
considered due to the restriction of time and other required 
resources. It would be interesting to know whether TAkka can help the 
construction and reliability of large commercial or research applications.



\section{Supervision Tree that Supports Software Rejuvenation}

The core idea of the Supervision Principle is to restart components {\it 
reactively} when they fail. Similarly, software rejuvenation 
\citep{huang1995software, dohi2000statistical} is a  {\it preventive} failure 
recovery mechanism which periodically restarts components with a clean internal 
state. The interval of restarting a component, called  {\it software 
rejuvenate schedule}, is set to a fixed period. Software rejuvenation has been 
implemented in a number of commercial and scientific applications to improve 
their longevity. As a supervisor can restart its children, can  {\it software 
rejuvenate schedule} be set for each actor?

\section{Measuring and Predicting System Reliability}

Due to the nature of software development, the library itself cannot guarantee 
the reliability of applications built using it; nor can the achieved 
high reliability of Erlang applications indicate that a newly 
implemented application using the supervision principle will have desired 
reliability.   To help software developers identify bugs in their applications, 
the ChaosMonkey library and the SupervisionView library are shipped with TAkka. 
 However, a quantitative measurement of software reliability under operational 
environment is still desired in practice.  To solve this problem, two 
approaches are discussed following.

The first approach is measuring the target system as a black-box.  
Unfortunately, \citet{Littlewood93} show that even long term failure-free 
observation itself does not mean that the software system will achieve high 
reliability in the future.  

The second approach is giving a specification of actor-based supervision tree 
and measuring the reliability of a supervision tree as the accumulated result 
of reliabilities of its sub-trees. 
By eliminating language features that are not related to supervision, both the worker 
node and the supervisor node in a supervision tree can be modelled as  
Deterministic Finite Automata.  Analysis shows that various
supervision trees can be modelled by a supervision tree that only contains 
simple worker nodes and simple supervisor nodes. 
To accomplish this study, the following problems need to be solved:
\begin{itemize}
  \item What are possible dependencies between nodes? For each dependency,
what is the algebraic relationship between the reliability of a sub-tree
and reliabilities of individual nodes?
  \item Based on the above result, how are the overall reliabilities
of a supervision tree to be calculated? When will the reliability be improved 
by using a supervising tree, and when will it not be?
  \item Given the reliabilities of individual workers and constraints between
them, is there an algorithm to give a supervision tree with desired 
reliability?  If not, can we determine if the desired reliability is not
achievable?
\end{itemize}


\chapter{Thesis Summary}
\label{summary}

The main goal of this thesis is the development of a library that combines the 
advantages of type checking and the supervision principle.  The aim is to 
contribute to the construction of reliable distributed applications via using 
type-parameterized actors and supervision trees.  Aside from the TAkka library 
itself, this thesis has presented the evaluation results of TAkka.  The 
evaluation metrics in this thesis can be used and further developed for the 
evaluation of other libraries that implement actors and supervision trees.  This 
chapter first reviews the research results presented in the thesis and then 
briefly concludes.


\section{Overview of Contributions}


\subsection{A library for Type-parameterized Actors and Their Supervision}

The key contribution of this thesis is the design and implementation of the 
TAkka library, which is the first programming library where type parameterized 
actors can form a supervision tree.  The TAkka library is built on top of Akka, 
a library which has been used for implementing real world applications.  The 
TAkka library adds type checking features to the Akka library but
delegates tasks such as actor creation and message passing to the underlying
Akka systems. 

The TAkka library uses both static and dynamic type checking so that type 
errors are detected at the earliest opportunity. To enable look up on remote 
actor references, TAkka defines a typed name server that keeps maps from typed 
symbols to values of the corresponding types.

In addition, Akka  programs can gradually migrate to their TAkka equivalents 
(evolution) rather than require providing type parameters everywhere 
(revolution).  The above property is analogous to adding generics to Java 
programs.
  
\subsection{A Library Evaluation Framework}

The second contribution of this thesis is a framework for evaluating the 
TAkka library.  The author believes that the employed evaluation metrics can be 
used and further developed for evaluating other libraries that implement actors 
and the supervision principle.

Compared with Akka, the TAkka library avoids the type pollution 
problem straightforwardly.  The type pollution problem, discussed in 
Section~\ref{type_pollution}, refers to the situation where a user can send a 
service message not expected from him or her because that service publishes too 
much type information about its communication interface. Without due care, 
the type pollution problem may occur in actor-based systems that are 
constructed using the layered architecture \citep{DijkstraTHE, 
buschmann2007pattern} or the MVC pattern \citep{reenskaug1979original, 
reenskaug2003model}, two popular design patterns for constructing modern 
applications.  A demonstration example shows that avoiding type 
pollution problem in TAkka is as simple as publishing a service as having 
different types when it is used by different parties.

By porting existing small and medium sized Erlang and Akka applications, 
results in Section~\ref{expressiveness} and \ref{efficiency} show that 
rewriting Akka programs using TAkka will {\it not} bring obvious runtime and 
code-size overheads. As regards expressiveness, {\it all} Akka applications 
considered in this thesis can be ported to their TAkka equivalents with a small 
portion of code modifications.  The TAkka library is expected to have the {\it 
same} expressiveness as the Akka library.

Finally, the reliability of a TAkka application can be partly assessed by using
the Chaos Monkey library and the Supervision View library.  
The Chaos Monkey library, ported from the work by \citet{ChaosMonkey} and 
\citet{ErlangChaosMonkey}, tests whether an application can survive in an 
adverse environment where exceptions raise randomly.  The Supervision View 
library dynamically captures the structure of supervision trees.  With the help 
of the Chaos Monkey library and the Supervision View library, application 
developers can visualise how the application will behave under the tested 
condition.



\section{Conclusion}

The author believes that the demand for distributed applications will continue 
increasing in the next few years.  The recent trends of emphasis on programming 
for the cloud and mobile platforms all contribute to this direction.  With the 
growing demands and complexity of distributed applications, their reliability 
will be one of the top concerns among application developers. 

The TAkka library introduces a type-parameter for actor-related classes. The 
additional type-parameter of a TAkka actor specifies the communication 
interface of that actor.  The author is glad to see that type-parameterized 
actors can form supervision trees in the same way as untyped actors.
Lastly, test results show that building type-parameterized actors on top of 
Akka does not introduce significant overheads, with respect to program size,
efficiency, and scalability.  In addition, debugging techniques such 
as Chaos Monkey and Supervision View can be applied to applications built 
using actors with supervision trees.  The above results encourage the use of 
types and supervision trees to implement reliable applications and improve the 
reliability of legacy applications with little effort.  The author expects 
similar results can be obtained in other actor libraries.




